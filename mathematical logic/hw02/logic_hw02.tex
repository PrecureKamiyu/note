\documentclass[12pt, a4paper]{ctexart} %中文支持
\usepackage{float}              %防止浮动元素浮动
\usepackage{rotating}           %旋转图片
\usepackage{amsfonts}           %对某一些字体之支持
\usepackage{mathrsfs}           %mathscr e.g.
\usepackage[]{amsmath}          %数学公式
\usepackage{amsthm}             %定义, 定理, 证明, 例子环境的支持
%使用方法:
%\newtheorem{environment name}{caption}
%比如 \newtheorem{example}{这是例子}
%效果 \begin{example} xxx \end{example} -> 这是例子 1 xxx
%proof就不需要了
\usepackage{graphicx}           %插入图片
\usepackage[left=1.60in,right=1.60in,top=1in,bottom=1in]{geometry}   %用来排版的
\usepackage{color}            %给部分文本上色的
\usepackage{algorithm}          %写伪代码的
%\usepackage{algorithmic}       %同上
\usepackage{algorithmicx}
\usepackage{algpseudocode}
%\usepackage{minted}
\usepackage{amssymb}            %用来加入一些数学符号, 比如说 $\varnothing$
\usepackage{titlesec}
\usepackage{fontspec}           %不知道用来干嘛的
\usepackage{hyperref}           %生成可跳转的书签

%\usemintedstyle{vs}    %设置minted插入代码的风格

\titleformat*{\section}{\huge\bfseries}             %管理title的字体和大小
\titleformat*{\subsection}{\Large\bfseries}         %bfseries就是默认的字体.
\titleformat*{\subsubsection}{\large\bfseries}

\newcommand{\N}{\mathbb{N}}
\newcommand{\R}{\mathbb{R}}

\theoremstyle{plain}
\newtheorem{theorem}{Theorem}[section]
\newtheorem{lemma}{Lemma}[section]
\newtheorem{proposition}{Proposition}[section]

\theoremstyle{definition}
\newtheorem{example}{Example}[section]
\newtheorem{definition}{Definition}[section]
\newtheorem*{remark}{Remark}
\newtheorem*{comment}{Comment}
\newtheorem{problem}{Problem}
\newtheorem*{solution}{Solution}
\def\downa{\downarrow}
\def\upa{\uparrow}
\def\lra{\Leftrightarrow}
\title{zuoye2}
\author{210110531\and 毛翰翔}
\date{\today}
\pagestyle{plain}
\begin{document}
\maketitle
\begin{problem}
    使用 $\uparrow$ and $\downarrow$ 表示下面的 proposition.
    \begin{enumerate}
        \item $\neg p \lor q$
        \item $p \wedge \neg q$
        \item $\neg p \lor \neg q$
        \item $p \leftrightarrow q$
    \end{enumerate}
\end{problem}
\begin{solution} to 1.
        \begin{align*}
            \lra & \neg \neg \left( \neg p \lor q\right) \\ 
            \lra & \neg \left( \neg p \downa q\right)\\
            \lra & \neg \left(  \left( p \downa p\right) \downa q\right) \\
            \lra & \left( \left( p \downa p\right) \downa q \right) \downa \left( \left( p \downa p \right) \downa q\right)
        \end{align*}
        而后   
        \begin{align*}
            \lra & \neg p \vee \neg \neg q \\
            \lra & p \upa \left( \neg q\right)\\
            \lra & p \upa \left( q \upa q\right)
        \end{align*}
\end{solution}
\begin{solution} to 2.
        \begin{align*}
            & p \wedge \neg q\\
            \lra & \neg \neg p \wedge \neg q\\
            \lra & \neg p \downa q \\
            \lra & \left( \left( p \downa p\right) \downa q\right)
        \end{align*}
        而后 
        \begin{align*}
            & p \wedge \neg q \\
            \lra & \neg \neg \left( p \wedge \neg q\right) \\
            \lra & \neg \left( p \upa  \neg q\right)\\
            \lra & \neg \left( p \upa \left(q \upa q\right)\right)\\
            \lra & \left(p \upa \left(q \upa q\right)\right) \upa \left( p \upa \left( q \upa q\right)\right)\\
        \end{align*}
\end{solution}
\begin{solution} to 3.
        \begin{align*}
            & \neg p \lor \neg q\\
            \lra & \neg \neg \left( \neg p \lor \neg q\right) \\
            \lra & \neg \left( \neg p \downa \neg q\right) \\
            \lra & \neg \left( \left( p \downa p\right) \downa \left( q \downa q\right) \right)\\
            \lra & \left( \left( p \downa p \right) \downa \left(q \downa q\right) \right) \downa \left( \left( p \downa p \right) \downa \left(q \downa q\right) \right) \\
        \end{align*}
        而后 
        \begin{align*}
            & \neg p \lor \neg q \\
            \lra & \neg \left( p \wedge q\right) \\
            \lra & p \upa q 
        \end{align*}
\end{solution}
\begin{solution} to 4. 
        \begin{align*}
            & p \leftrightarrow q \\
            \lra & \left(\neg p \lor q\right) \wedge \left( \neg q \vee p\right)\\
            \lra & \neg \neg \left(  \left( \neg p \vee q\right) \wedge \left( \neg q \vee p\right) \right) \\
            \lra & \neg \left( \left( \neg \left( \neg p \vee q\right)\right) \vee \left( \neg \left( \neg q \vee p\right)\right)\right) \tag{de Morgan's law}\\ 
            \lra & \neg \left( \left( \neg p \downa q\right) \vee \left( \neg q \downa p\right)\right)\\
            \lra & \left(\neg p \downa p\right) \downa \left(\neg q \downa p\right)\\
            \lra & \left(\left( p \downa p \right) \downa q  \right) \downa \left( \left( q\downa q\right) \downa p\right)
        \end{align*}
        而后
        \begin{align*}
            & p \leftrightarrow q \\
            \lra & \left(\neg p \wedge  \neg q \right) \vee \left( p \wedge q\right)  \tag{经过简单的化简}\\
            \lra & \left( \neg \left(\neg p \wedge \neg q\right)\right) \upa \left(\neg \left( p \wedge q\right)\right) \\
            \lra & \left( \neg p \upa \neg q\right) \upa \left( p \upa q\right) \\
            \lra & \left(\left( p \upa p \right) \upa \left( q \upa q\right)\right) \upa \left( p \upa q\right)
        \end{align*}
\end{solution}
\begin{problem}
    在 Propositonal Calculus 中证明下面这些事实.
    \begin{enumerate}
        \item $\vdash  \left(A \to  \left(A \to B \right)\right) \to \left(A \to B \right)$ 
        \item $A \to B , \neg \left( B \to C \right) \to \neg  A \vdash A \to C$ 
        \item $\vdash \left(A \to \left(B \to C\right) \right) \to ((C \to D)\to (A \to  ( B \to D   )))$
        \item $\vdash \left( \left(A \to B \right) \to \left( B \to A\right) \right) \to \left(B \to A\right)$
        \item $\vdash \left( \left(A \to B \right) \to A    \right) \to A $
        \item $\vdash \left( \left(A \to B \right) \to C\right) \to \left( \left(A \to C\right) \to C    \right) $ 
        \item $\vdash \left(A \to C\right) \to \left(\left(B \to C\right) \to \left(\left(\left(A \to B\right) \to B    \right) \to C    \right)\right)$
    \end{enumerate}
\end{problem}
\begin{comment}
    形如 $A \to  B \to C \to D$ 的式子, 默认从右往左加括号, viz. 原本这个式子应为
    \begin{align*}
        A \to \left(B \to \left(C \to D\right)\right) 
    \end{align*}
\end{comment}
\begin{solution} to 1. 
    \begin{align}
        &\left(A  \to B\right) \to \left(A \to B \right)                                  \tag{thm 1} & (1) \\
        &A \to \big(  \left(A \to B\right) \to B    \big)                                 \tag{前件互换, thm 8} & (2) \\
        &A \to \big( \left(A \to B \right) \to B    \big) \to \big( A \to \left(A \to B\right) \big) \to \left(A \to B\right) \tag{A2}& \left(3\right) \\
        &\left(A \to \left(A \to B\right) \right) \to \left(A \to B\right) \tag{(2) (3) thm 8} & \left(4\right)
    \end{align}
\end{solution}
\begin{solution} to 2.
    \begin{align*}
         &\ \big( \neg \left(B \to C\right) \to \neg A \big) \to \left(A \to \left(B \to C\right) \right)&\quad \left(1\right)\tag{A3}\\
         &\ \left( \neg \left(B \to C\right) \right)                                                     &\quad \left(2\right)        \\
         &\ \left(A \to \left(B \to C\right) \right)                                                     &\quad \left(3\right)    \tag{(1) (2) rmp}\\
         &\ \left(A \to B\right) \to \left(A \to C\right)                                                &\quad \left(4\right)\tag{A2, thm 8, rmp} \\
         &\ A \to B                                                                                      &\quad \left(5\right)\\
         &\ A \to C                                                                                      &\quad \left(5\right)\tag{(4) (5) rmp}
    \end{align*}
\end{solution}

\begin{solution} to 3.
    \begin{align*}
        (1) \ &\neg A \to (A \to \left(\left(C \to D\right) \to \left(B \to D\right)\right)) \tag{thm 6}\\
        (2) \ &\neg A \to \left(\left(C \to D\right) \to \left(A \to \left( B \to D\right)\right)\right) \tag{thm 3, rmp}\\
        (3) \ &\left(B \to C\right) \to \left(\left( C \to D\right) \to \left(B \to D\right) \right) \tag{thm 5} \\
        (4) \ &A \to \left(\left(B \to C\right) \to \left(\left(C \to D\right) \to \left(B \to D\right)\right)\right) \tag{thm 1, rmp}\\
        (5) \ &\left( B \to C \right) \to \left(A \to \left(\left(C \to D\right) \to \left(B \to D\right)\right)\right)\tag{thm 3, rmp}\\
        (6) \ &\left(B \to C\right) \to \left(\left(C \to D\right) \to \left(A \to \left(B \to D\right)\right)\right) \tag{thm 8, thm 3}\\
        (7) \ &\left(A \to \left(B \to C\right) \right) \to \left(\left(C \to D\right) \to \left(A \to \left(B \to D\right)\right)\right) \tag{(2), (6)thm 18}
    \end{align*}
\end{solution}
\begin{solution} to 4.
    \begin{align*}
        (1)\ & \left(B \to A\right) \to \left(B \to A  \right) \tag{thm 1}\\
        (2)\ & \neg A \to \left(A \to B     \right) \tag{thm 6}\\
        (3)\ & B \to \left(\neg A \to \left(A \neg \left(A \to B\right) \right)\right) \tag{thm 1, rmp}\\
        (4)\ & B \to \left(\neg \left( A \to B  \right) \to A  \right) \tag{thm 14, thm 8}\\
        (5)\ & \neg \left(A \to B  \right) \to \left( B \to A  \right) \tag{thm 3, rmp}\\
        (6)\ & \left(\left(A \to B\right) \to \left(B \to A\right) \right) \to \left(B \to A\right) \tag{thm 18, (1)(6)}
    \end{align*}
\end{solution}
\begin{solution} to 5. 
    \begin{align*}
        (1)\qquad \qquad \quad \quad \quad&A \to A \tag{thm 1}\qquad \qquad \qquad \\
        (2)\qquad \qquad \quad \quad \quad&\neg A \to \left( A \to B\right) \tag{thm 6} \qquad \qquad \qquad \\
        (3)\qquad \qquad \quad \quad \quad&\neg \left(A \to B\right) \to A \tag{thm 14, rmp} \qquad \qquad \qquad \\
        (4)\qquad \qquad \quad \quad \quad&\left( \left(A \to B\right) \to A    \right)\to A \tag{thm 18, (1) (3)}\qquad \qquad
    \end{align*}
\end{solution}
\begin{solution} to 6.
    \begin{align*}
        (1) \quad \quad & \neg C \to \left(C \to B\right) \tag{thm 6} \\
        (2) \quad \quad & A \to \left(\neg C \to \left(C \to B\right)\right)\tag{thm 1, rmp}\\
        (3) \quad \quad & \neg C \to \left(A \to \left(C \to B\right)\right) \tag{thm 3}\\
        (4) \quad \quad & \neg C \to \left(\left(A \to C\right) \to \left(A \to B\right)\right) \tag{A2, thm 8} \\
        (5) \quad \quad & \neg C \to \left(\neg \left(A \to B\right) \to \neg \left(A \to C\right)\right) \tag{thm 13, thm 8}\\
        (6) \quad \quad & \Big(\neg C \to \left(\neg \left(A \to B\right)\right)\Big)\to \Big(\neg C \to \neg \left(A \to C\right)\Big) \tag{A2, rmp}\\
        (7) \quad \quad & \Big( \left(A \to B\right) \to C \Big) \to \left(\neg C \to \neg \left(A \to B\right)\right) \tag{thm 13}\\
        (8) \quad \quad & \left(\neg C \to \neg \left(A \to C\right)\right) \to \left(\left(A \to C\right) \to C  \right) \tag{A3}\\
        (9) \quad \quad & \left( \left(A \to B \right) \to C\right) \to \left(\left(A \to C\right) \to C\right) \tag{thm 21, (7)(8)}
    \end{align*}
\end{solution}
\begin{solution} to 7.
    We shall use thm 18 to prove.
    \begin{align*}
        (1) & \quad \quad \neg C \to \left(C \to B\right) \tag{thm 6} \\
        (2) & \quad \quad A \to  \left(\neg C \to \left(C \to B\right)\right)  \tag{thm 1, rmp}\\
        (3) & \quad \quad \neg C \to A \to \left(C \to B\right) \tag{thm 3} \\
        (4) & \quad \quad \neg C \to \left(\left(A \to C\right) \to \left(A \to B\right)\right) \tag{A2, rmp}\\
        (5) & \quad \quad \left(A \to C\right) \to \left(\neg C \to \left(A \to B\right) \right) \tag{thm 3, rmp} \\
        (6) & \quad \quad \left(A \to C\right) \to \left(\neg \left(A \to B\right) \to C\right) \tag{thm 14, thm 8} \\
        (7) & \quad \quad \left(B \to C\right) \to \left(\left(A \to C\right) \to \left(\neg \left(A \to B\right) \to C\right)\right) \tag{thm 1, rmp}\\
        (8) & \quad \quad \neg \left(A \to B\right)  \to \left(\left(A \to C\right) \to \big(\left(B \to C\right) \to C\big)\right) \tag{多用几次thm 3}\\
    \end{align*}
    接下来证明另一半.
    \begin{align*}
        (9) & \quad \quad \left(B \to C\right) \to \left(B \to C\right) \tag{thm 1} \\
        (10) & \quad \quad \left(B \to \left(B \to C\right) \to C\right) \tag{thm 3}\\
        (11) & \quad \quad \left(A \to C\right) \to \left(B \to \left(B \to C\right) \to C\right) \tag{thm 1, rmp} \\
        (12) & \quad \quad B \to \left(A \to C\right) \to \left(B \to C\right) \to C \tag{thm 3, rmp} \\
    \end{align*}
    一上面这两部分的结果作为 thm 18 的理由: 
    \begin{align*}
        (13)& \quad \quad \big(\left(A \to B\right) \to B\big) \to \left((A \to C) \to \left(B \to C\right) \to C\right) \tag{thm 18 (12)(8)}\\ 
        (14)& \quad \quad \left(A \to C\right)   \to \bigl( \left(A \to B\right) \to B     \bigr) \to \left(B \to C\right) \to C\tag{thm 3, rmp}\\
        (15)& \quad \quad \Bigl(( \left(A \to B \right) \to B ) \to \left(B \to C\right) \to C    \Bigr) \to \Bigl( \left(B \to C\right) \to \left( \left(A \to B\right) \to B\right) \to C\Bigr)   \tag{thm 3} \\
        (16)& \quad \quad \left(A \to C\right) \to \left(B \to C\right) \to \left(\left(A \to B\right) \to B   \right) \to C\tag{thm 8} \\
    \end{align*}
\end{solution}
\end{document}