\documentclass[12pt, a4paper]{ctexart} %中文支持
\usepackage{float}              %防止浮动元素浮动
\usepackage{rotating}           %旋转图片
\usepackage{amsfonts}           %对某一些字体之支持
\usepackage{mathrsfs}           %mathscr e.g.
\usepackage[]{amsmath}          %数学公式
\usepackage{amsthm}             %定义, 定理, 证明, 例子环境的支持
%使用方法:
%\newtheorem{environment name}{caption}
%比如 \newtheorem{example}{这是例子}
%效果 \begin{example} xxx \end{example} -> 这是例子 1 xxx
%proof就不需要了
\usepackage{graphicx}           %插入图片
\usepackage[left=1.30in,right=1.30in,top=1in,bottom=1in]{geometry}   %用来排版的
\usepackage{color}            %给部分文本上色的
\usepackage{algorithm}          %写伪代码的
%\usepackage{algorithmic}       %同上
\usepackage{algorithmicx}
\usepackage{algpseudocode}
%\usepackage{minted}
\usepackage{amssymb}            %用来加入一些数学符号, 比如说 $\varnothing$
\usepackage{titlesec}
\usepackage{fontspec}           %不知道用来干嘛的
\usepackage{hyperref}           %生成可跳转的书签

%\usemintedstyle{vs}    %设置minted插入代码的风格

\titleformat*{\section}{\huge\bfseries}             %管理title的字体和大小
\titleformat*{\subsection}{\Large\bfseries}         %bfseries就是默认的字体.
\titleformat*{\subsubsection}{\large\bfseries}

\newcommand{\N}{\mathbb{N}}
\newcommand{\R}{\mathbb{R}}

\theoremstyle{plain}
\newtheorem{theorem}{Theorem}[section]
\newtheorem{lemma}{Lemma}[section]
\newtheorem{proposition}{Proposition}[section]

\theoremstyle{definition}
\newtheorem{example}{Example}[section]
\newtheorem{definition}{Definition}[section]
\newtheorem*{remark}{Remark}
\newtheorem*{comment}{Comment}
\newtheorem{problem}{Problem}
\newtheorem*{solution}{Solution}
\newtheorem*{sketchoftheproof}{Sketch of the Proof}
\newtheorem*{axiom}{Axiom}

\pagestyle{plain}
\title{ND}
\author{Nobody}

\begin{document}
\maketitle
\tableofcontents
\section{definition}
\subsection{Language part}
\begin{align*}
    \Sigma = \left\{ (, ), \leftrightarrow , \rightarrow , \neg , \vee, \wedge , \text{Propositional letters}\right\}
\end{align*}

the definitoin of the Propositions 

the definition is similar to those in the PC. 

\subsection{The 推理过程}
\begin{axiom}
    \begin{equation}
    \Gamma \cup  \left\{A \right\} \vdash A
    \end{equation}
    注意这里只有一条公理. 
\end{axiom}

\begin{definition}[推理的表示]
我们使用类似分号的形式进行表示. 比如说这里是第一条推理规则: 引入规则. 
\begin{equation}
    \frac{\Gamma \vdash  B}{\Gamma \cup \left\{A\right\} \vdash B}
\end{equation}
上面为条件, 下面为结果.
\end{definition}

\begin{proposition}[假设引入]
    \begin{equation}
        \frac{\Gamma \vdash  B   }{ \Gamma \cup \left\{A\right\} \vdash  B}
    \end{equation}
    记为 $\left(+ \right)$
\end{proposition}
\begin{proposition}[假设消除]
    \begin{equation}
        \frac{\Gamma \cup \left\{A\right\}\vdash B \quad \Gamma \cup  \left\{\neg A\right\} \vdash B}{\Gamma \vdash B}
    \end{equation}
    记为 $\left(-\right)$
    \begin{example}[证明反证法]
        需要使用蕴含消除, 等之后就会讲了. 
        \begin{equation}
            \left(\neg A \to A\right) \to A 
        \end{equation}
    \end{example}
\end{proposition}
\begin{proposition}[析取引入]
    \begin{equation}
        \frac{\Gamma \vdash A}{\Gamma \vdash A\vee B}
    \end{equation}
\end{proposition}
\begin{proposition}[析取消除]
\begin{equation}
    \frac{\Gamma \cup A \vdash  C \quad \Gamma \cup  B \vdash C \quad \Gamma \vdash A \vee B}{\Gamma \vdash C}
\end{equation}
\end{proposition}
\begin{proposition}[合取引入]
\begin{equation}
    \frac{\Gamma \vdash A \quad \Gamma \vdash B}{\Gamma \vdash A \wedge B}
\end{equation}
比较简单. 我们使用PC中的东西可以证明这个, 我们已经证明PC的完备性了, 所以说, 可以使用PC来证明这些推理规则是否合理. 
\end{proposition}
\begin{proposition}[合取消除]
\begin{equation}
\frac{\Gamma \vdash A\wedge B   }{\Gamma \vdash  A}
\end{equation}
\end{proposition}
\begin{proposition}[$\rightarrow$ 引入]
\end{proposition}
\begin{proposition}[$\rightarrow$ 消除]
    \begin{equation}
        \frac{\Gamma \vdash A \quad \Gamma \vdash A \to B}{\cdots }
    \end{equation}
\end{proposition}
\begin{proposition}[$\neg$ 引入]
    \begin{equation}
        \frac{\Gamma \cup A \vdash B \quad \Gamma \cup  A \vdash \neg B}{\Gamma \vdash \neg A}
    \end{equation}
\end{proposition}
\begin{proposition}[$\neg$ 消除 ]
    \begin{equation}
        \frac{\Gamma \vdash  A \quad \Gamma \vdash  \neg A}{ \Gamma \vdash B}
    \end{equation}
\end{proposition}
\begin{proposition}[$\neg \neg$ +]
    略
\end{proposition}
\begin{proposition}[$\neg\neg -$]
\end{proposition}
\begin{proposition}[$\leftrightarrow + $]
    略
\end{proposition}
\begin{proposition}[$\leftrightarrow -$]
\end{proposition}

\subsection{演绎序列}
我们说, 演绎序列是下面这样的序列: 
\begin{equation}
    \Gamma_1 \vdash  A_1 , \Gamma_2 \vdash A_2 , \cdots , \Gamma_{m} \vdash A_{m}
\end{equation}
其中 $\Gamma _{i} \vdash A_{i}$ 是 axiom, 或者是 $\Gamma _{i} \vdash A_{i} = \Gamma _{j}  \vdash A_{j}$,
或者是 $\Gamma _{i} \vdash A_{i}$ 是一个推理结果. 

\subsection{例子}
值得注意的是, 上面几个推理规则之中, 有2 4 9 10是比较重要的. 

例子1: 反证法证明 2. 
\subsection{定理}
记得看ppt

\end{document}