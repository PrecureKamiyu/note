\documentclass[a4paper, 12pt]{ctexbook} %中文支持
\usepackage{float}              %防止浮动元素浮动
\usepackage{rotating}           %旋转图片
\usepackage{amsfonts}           %对某一些字体之支持
\usepackage{amsmath}            %数学公式
\usepackage{amssymb}            %用来加入一些数学符号, 比如说 $\varnothing$
\usepackage{amsthm}             %定义, 定理, 证明, 例子环境的支持
\usepackage{graphicx}           %插入图片
\usepackage[left=1.5in,right=1.5in,top=1in,bottom=1in]{geometry}   %用来排版的
\usepackage{color}              %给部分文本上色的
\usepackage{algorithm}          %写伪代码的
\usepackage{algorithmic}        %同上
%\usepackage{minted}             %书写代码
\usepackage{fontspec}           %不知道用来干嘛的
\usepackage{titlesec}           %用来调整section等的大小和字体
\usepackage{hyperref}           %生成可跳转的书签
\usepackage{blindtext}          %for test

\pagestyle{plain}               %这是调整页面的风格, plain的情况下只有页码

%titlesec
\titleformat*{\section}{\huge\bfseries}             %管理title的字体和大小
\titleformat*{\subsection}{\Large\bfseries}         %bfseries就是默认的字体.
\titleformat*{\subsubsection}{\large\bfseries}      

%\setmonofont{vs}                %调整这个minted代码的风格
\theoremstyle{plain}
\newtheorem*{comment}{Comment}

\theoremstyle{definition}
\newtheorem{theorem}{\llap{$\rhd$}\space Theorem}
\newtheorem{example}{\llap{$\bigstar$}\space Example}
\newtheorem{definition}{\llap{Definition}\space}[section]
\newtheorem{lemma}{\llap{Lemma}\space}
\newtheorem{proposition}{\llap{Proposition}\space}
\newtheorem{problem}{}

\theoremstyle{remark}
\newtheorem{remark}{\llap{Remark}\space}
\title{zuoye5}
\author{毛翰翔 \and 210110531}
\begin{document}
\maketitle
\begin{problem}
\begin{enumerate}
    \item 没有无知的教授
    \begin{align*}
        \neg \exists x \left(P _{1} x \wedge P_{2} x\right)
    \end{align*}
    其中 $P_{1} x$ 表示 $x$ 是教授; $P_{2} x$ 表示 $x$ 无知. 
    \item 所有无知者均爱虚荣
    \begin{align*}
        \forall  x \left(P_{3} x \to P_{4} x\right)
    \end{align*}
    $P_{3} x$ 表示 $x$ 无知, $P_{4} x$ 表示 $x$ 爱虚荣. 
    \item 没有爱虚荣的教授
    \begin{align*}
        \neg \exists x \left( P_{5}  x \wedge P_{6} x\right)
    \end{align*}
    其中 $P_{5} x    $ 表示 $x$ 是教授; $P_{6} x$ 表示 $x$ 爱虚荣.
\end{enumerate} 
\end{problem}
\begin{problem}
\begin{enumerate}
\item $\vdash (A \to \exists v B ) \to \exists v \left( A \to B \right)$
\begin{align*}
    (1)  \quad & \neg A \to \left(A \to B \right) \tag{PC 定理6}\\ 
    (2)  \quad & \neg \left(A \to B \right) \to A  \tag{逆否命题} \\
    (3)  \quad & \forall  v \neg \left(A \to B \right) \to \neg (A\to B ) \tag{去全称}\\
    (4)  \quad & \forall v \neg \left(A \to B \right) \to A \tag{(2)(3) 三段论}\\ 
    (5)  \quad & B \to  \left(A \to B \right) \tag{A1} \\
    (6)  \quad & \neg \left(A\to B \right) \to \neg B  \tag{逆否命题}\\
    (7)  \quad & \forall  v \neg \left(A \to B \right) \to \neg B \tag{(3)(6)三段}\\
    (8)  \quad & \forall  v \neg \left(A \to B \right) \vdash \forall v \neg B \tag{全称推广}\\
    (9)  \quad & \forall v \neg \left(A \to B \right) \vdash \neg \left(A \to \neg \forall  x \neg B \right) \tag{(4)(8)} \\
    (10) \quad  & \vdash  \forall  v \neg \left(A \to B \right) \to \neg \left(A \to \neg \forall  v \neg B \right) \tag{演绎定理}\\ 
    (11) \quad  & \vdash  \left(A \to \neg \forall  v \neg B \right) \to \neg  \forall  v \neg \left(A \to B \right)  \tag{逆否}\\
    (12) \quad  & \vdash \left(A \to \neg \forall  v \neg B     \right) \to \exists v \left(A \to B \right)
\end{align*}
\item $\vdash \exists v \left(A \to B \right) \to \left(A \to \exists v B \right) $ 
\begin{align*}
    (1)\quad & \left\{ A , \forall  v \neg B  \right\} \vdash \neg B \tag{去全称}\\  
    (2)\quad & \left\{  A  , \forall  v \neg B \right\} \vdash  \neg \left(A \to B \right) \tag{(1) , 已知条件} \\
    (3)\quad & \left\{ A , \forall v \neg B  \right\} \vdash  \forall  v \neg \left(A \to B \right) \tag{全称推广} \\
    (4)\quad & \left\{ A , \neg \forall  v \neg \left(A \to B \right) \right\} \vdash  \neg \forall  v \neg B  \tag{演绎定理和逆否}\\
    (5)\quad & \left\{ \neg \forall  v \neg \left(A \to B  \right) \right\} \vdash A \to \neg \forall  v \neg B \tag{演绎}\\ 
    (6)\quad & \vdash \neg \forall  v \neg  \left(A \to B \right) \to \left(A \to \neg \forall  v \neg B\right)\tag{演绎}
\end{align*}
\item $\vdash  \left(\forall  v B \to A \right) \to \exists v \left( B \to V\right)$ 
\begin{align*}
(1)\quad & \forall v \neg \left(B \to A \right) \vdash \neg  \left(B \to A \right) \tag{去全称} \\ 
(2)\quad & \forall  v \neg \left( B \to A  \right) \vdash  \neg A \tag{(1)} \\ 
(3)\quad & \forall  v \neg \left( B \to A \right) \vdash B \tag{(1)} \\
(4)\quad & \forall  v \neg \left( B \to A  \right) \vdash \forall v B \tag{全称推广} \\
(5)\quad & \forall  v \neg \left( B\to A \right) \vdash  \neg \left(\forall  v B \to  A     \right) \tag{(2) (4)} \\
(6)\quad & \forall  v B \to A  \vdash \neg \forall  v \neg \left(B \to A \right)\tag{逆否}\\ 
(7)\quad & \forall  v B \to A \vdash \exists v  \left(B \to A \right) 
\end{align*}
\item $\vdash \exists v \left( B \to A \right) \to \left(\forall  v B \to A \right) $ 
\begin{align*}
(1)&\quad      \neg \left(\forall  v B \to A \right) \vdash  \forall  v B \tag{定理}\\
(2)&\quad      \neg \left(\forall  v B \to A  \right) \vdash \neg A \tag{定理} \\
(3)&\quad      \neg \left(\forall  v B \to A  \right) \vdash  \forall v B \to B \tag{去全称} \\
(4)&\quad  \neg \left(\forall  v B \to A  \right) \vdash  B \tag{(1)(3)rmp} \\
(5)&\quad  \neg \left(\forall  v B \to A  \right) \vdash  \neg \left(B \to A \right) \tag{(2)(4)定理} \\ 
(6)&\quad  \neg \left(\forall  v B \to A  \right) \vdash  \forall  v \neg \left(B \to A \right) \tag{全称推广} 
\end{align*}
\end{enumerate}
\end{problem}
\begin{problem}
\begin{enumerate}
    \item $\forall  x \left(A \to B \right) \vdash  \dashv A \to \forall  x B $, $x$ 在 $A$ 中无自由出现. \\
    先是证明 $\vdash  \forall x \left(A \to B \right) \to  \left(A \to \forall  x B \right) $, 使用演绎定理转化一下: 
\begin{align*}
(1)\quad &\forall  x \left(A \to B \right) \vdash \left(A \to \forall  x  B \right) \\
(2)\quad &\left\{ \forall  x \left(A \to  B\right) , A\right\} \vdash \forall  x B
\end{align*} 所以我们要证明 $\left\{ \forall  x \left(A \to  B\right) , A\right\} \vdash \forall  x B $
\begin{align*}
(1)\quad &\left\{ \forall x \left(A \to B \right), A \right\} \vdash  \forall  x A \to \forall  x B \tag{A5} \\
(2)\quad &\left\{ \forall  x  \left(A \to B \right) , A    \right\} \vdash \forall  x A \tag{全称推广} \\ 
(3)\quad &\left\{ \forall  x \left(A \to B \right) , A      \right\} \vdash  \forall  x B \tag{三段}
\end{align*}
然后证明 $\left(A \to \forall  x B \right) \vdash  \forall  x \left(A \to B \right)$
\begin{align*}
(1)\quad &\left\{ A \to \forall  x B  \right\} \vdash  \left(\forall  x B \to B   \right) \tag{去全称}\\
(2)\quad &\left\{ A \to \forall  x B \right\} \vdash  \left(A \to B \right) \tag{(1),已知,三段论} \\
(3)\quad &\left\{ A \to \forall  x B \right\} \vdash  \forall  x \left(A \to B \right) \tag{全称推广}
\end{align*}
    \item $\forall  x \left(A \to B \right) \vdash  \dashv \exists x A \to B    $, $x$ 在 $B $ 中无自由出现. 

先是证明: $\forall  x \left(A \to B \right) \vdash  \exists x A \to B $, 使用演绎定理转化一下: 
\begin{align*}
(1) \quad &\forall  x \left(A \to B \right) \vdash \left(\neg \forall  x \neg A \right) \to B \\
(2) \quad &\forall  x \left(A \to B \right) \vdash  \neg B \to \forall  x \neg A \\
(3) \quad &\left\{\forall  x \left(A \to B\right) , \neg B \right\} \vdash\forall  x \neg A 
\end{align*}
\begin{align*}
(1)\quad &    \left\{\forall  x \left(A \to B \right) , \neg B \right\} \vdash \forall  x \left(A \to B \right) \to \left(A \to B\right) \\
(2)\quad &\left\{\forall  x \left(A \to B \right) , \neg B \right\} \vdash  \left(A \to B \right) \tag{rmp} \\
(3)\quad &\left\{\forall  x \left(A \to B \right) , \neg B \right\} \vdash  \left(A \to B \right) \to \left( \neg B \to \neg A \right) \tag{逆否} \\
(4)\quad &\left\{\forall  x \left(A \to B \right) , \neg B \right\} \vdash  \left(\neg B \to \neg A \right) \tag{rmp} \\
(5)\quad &\left\{\forall  x \left(A \to B \right) , \neg B \right\} \vdash  \neg A \tag{(4) 已知条件 rmp}\\
(6)\quad &\left\{\forall  x \left(A \to B \right) , \neg B \right\} \vdash  \forall  x \neg A \tag{全称推广}
\end{align*}
然后证明: $\exists x A \to B \vdash \forall  x \left(A \to B \right) $ 
\begin{align*}
(1)\quad &\forall  x \neg A \vdash \neg A \\
(2)\quad &\forall  x \neg A \vdash \neg A \to \left(A \to B \right) \tag{PC 定理6}\\
(3)\quad &\forall  x \neg A \vdash \left(A \to B \right) \tag{rmp}\\
(4)\quad &\forall x \neg A \vdash \forall  x \left(A \to B\right) \tag{全称推广}\\ 
(5)\quad &\neg \forall  x \left(A \to B \right) \vdash \neg \forall  x \neg A \tag{逆否}\\ 
(6)\quad &\vdash B \to \left(A \to B \right)  \tag{A1}\\
(7)\quad &B \vdash A \to B \tag{演绎}\\ 
(8)\quad &B \vdash \forall  x \left(A \to B \right) \tag{全称推广}\\ 
(9)\quad &\neg \forall  x \left(A \to B \right) \vdash \neg B \tag{逆否}\\
(10)\quad &\neg \forall  x \left(A \to B \right) \vdash \neg \left(\exists x A \to B \right)\tag{(5) (9)}\\
(11)\quad &\left(\exists x A \to B      \right) \vdash \forall  x \left(A \to B \right) 
\end{align*}
    \item $\forall  \left(A\wedge B \right) \vdash \dashv \forall  x A \wedge \forall  x B $

我们知道 $A \wedge B$ 实际上就是 $\neg \left(A \to \neg B \right) $, 于是我们就是要证明 
$$\forall  x \neg \left(A \to \neg B \right) \vdash \dashv \neg \left(\forall x A \to \neg \forall  x \neg B \right)$$
这里先证明
$\forall  x \neg \left(A \to \neg B \right) \vdash  \neg \left(\forall  x A \to \neg \forall  x B \right) $
\begin{align*}
(1)\quad &\forall x \neg \left(A \to \neg B\right) \vdash  \neg \left(A \to \neg B \right) \tag{去全称}\\ 
(2)\quad &\forall x \neg \left(A \to \neg B \right) \vdash  \neg \left(A \to \neg B \right) \to A \tag{PC 定理6的逆否}\\ 
(3)\quad &\forall x \neg \left(A \to \neg B \right)\vdash  A \tag{rmp} \\
(4)\quad &\forall x \neg \left(A \to \neg B \right) \vdash \forall  x A \tag{全程推广}\\ 
(5)\quad &\forall x \neg \left(A \to \neg B \right) \vdash \neg B \to \left(A \to \neg B \right) \tag{A1} \\
(6)\quad &\forall x \neg \left(A \to \neg B \right) \vdash  \neg \left(A \to \neg B \right) \to B \tag{逆否} \\ 
(7)\quad &\forall x \neg \left(A \to \neg B \right) \vdash  B \tag{rmp, (6), 已知}\\ 
(8)\quad &\forall x \neg \left(A \to \neg B \right) \vdash  \forall x B \\ 
(9)\quad &\forall x \neg \left(A \to \neg B \right) \vdash \neg \left(\forall  x A \to \neg \forall  x B\right) \tag{(4),(8)}
\end{align*}
然后证明另一半: 
\begin{align*}
(1)\quad &\neg \left(\forall  x A \to \neg \forall  x B \right) \vdash\forall  x A \tag{定理6的逆否, rmp}\\ 
(2)\quad &\neg \left(\forall  x A \to \neg \forall  x B \right) \vdash  \neg \neg \forall  x  B \tag{A1的逆否, rmp}\\
(3)\quad &\neg \left(\forall  x A \to \neg \forall  x B \right) \vdash \forall  x A \to A \tag{去全称}\\
(4)\quad &\neg \left(\forall  x A \to \neg \forall  x B \right) \vdash A \tag{rmp}\\ 
(5)\quad &\neg \left(\forall  x A \to \neg \forall  x B \right) \vdash  \neg \neg \forall  x B \to \forall  x  B \tag{否定的否定}\\
(6)\quad &\neg \left(\forall  x A \to \neg \forall  x B \right) \vdash  \forall  x B \tag{rmp}\\ 
(7)\quad &\neg \left(\forall  x A \to \neg \forall  x B \right) \vdash  B \tag{去全称}\\
(8)\quad &\neg \left(\forall  x A \to \neg \forall  x B \right) \vdash  \neg \left(A \to \neg B\right) \tag{(7), (4)}\\
(9)\quad &\neg \left(\forall  x A \to \neg \forall  x B \right) \vdash  \forall  x \neg  \left(A \to \neg B \right)  \tag{全称推广}
\end{align*}
\item $\exists x \left(A \vee  B \right) \vdash \dashv \exists x A \vee \exists x B $

等价于证明: $\vdash  \exists x \left(A \vee B \right) \leftrightarrow \exists x A \vee \exists x B $\\
注意到: 
\begin{align*}
    \neg \left(A \vee B \right) \leftrightarrow \neg A \wedge \neg B 
\end{align*}
下面证明这个: 
\begin{align*}
\vdash     \forall  x \neg \left(A \vee B \right) \leftrightarrow \forall  x \left(\neg A \wedge \neg B \right)
\end{align*}
\begin{align*}
(1) \quad & \forall  x \neg \left(A \vee B \right) \vdash  \neg \left(A \vee B\right) \tag{去全称}\\
(2) \quad & \forall x \neg \left(A \vee B\right) \vdash \neg \left(A \vee B \right) \to \neg A \wedge \neg B \\
(3) \quad & \forall x \neg \left(A \vee B \right) \vdash  \neg A \wedge \neg B \tag{rmp}\\
(4) \quad & \forall x \neg \left(A \vee B \right) \vdash \forall  x (\neg A \wedge \neg B )\tag{全称推广}\\
(5) \quad & \vdash \forall  x \neg \left(A \vee B \right) \to \forall x \left( \neg A \wedge \neg B \right)\tag{演绎} \\
(6) \quad & \forall x \left(\neg A \wedge \neg B \right) \vdash \neg A \wedge \neg B \tag{去全称}\\
(7) \quad & \forall x \left(\neg A \wedge \neg B \right) \vdash \neg A \wedge \neg B \to \neg \left(A \vee B\right) \\
(8) \quad & \forall x \left(\neg A \wedge \neg B \right) \vdash \neg \left(A \vee B \right) \tag{rmp} \\
(9) \quad & \forall x \left(\neg A \wedge \neg B \right) \vdash \forall x \neg \left(A \vee B \right) \tag{全称推广}\\
(10) \quad & \vdash  \forall x \left(\neg A \wedge \neg B\right) \to  \forall x \neg \left(A \vee  B\right) \tag{演绎}\\
(11) \quad & \vdash \forall x \left(\neg A \wedge \neg B \right) \leftrightarrow \forall x \neg \left(A \vee B \right) \tag{(5)(10)} 
\end{align*}

接下来证明 $\vdash \exists x A \vee \exists x B \to\exists x \left(A \vee B \right) $
\begin{align*}
(1) \quad &\vdash \forall x \left(\neg A \wedge \neg B \right) \to  \forall  x \neg A \wedge \forall  x \neg B \tag{上一问的结论}\\
(2) \quad &\vdash \forall  x \neg\left(A \vee B\right) \to \forall  x \left(\neg A \wedge \neg B \right) \tag{刚刚证明的结论}\\
(3) \quad &\vdash \forall x \neg\left(A \vee B \right) \to \forall  x \neg A \wedge \forall  x \neg B \tag{(1)(2)三段论}\\
(4) \quad &\vdash \neg \left(\forall  x \neg A \wedge \forall  x \neg B \right) \to \neg \forall  x \neg \left(A \vee B \right) \tag{逆否}\\
(5) \quad &\vdash \left(\neg \forall  x \neg A \vee \neg \forall  x \neg B \right) \to \neg \left(\forall  x \neg A \wedge \forall  x \neg B \right) \\
(6) \quad &\vdash \left(\neg \forall  x \neg A \vee \neg \forall x \neg B\right) \to \neg \forall  x \neg \left(A \vee B \right) \tag{(4)(5)三段论}\\
(7) \quad &\vdash \exists x A \vee \exists x B \to \exists x \left(A \vee B \right) 
\end{align*}
证明 $\vdash \exists x \left(A \vee B \right) \to \exists x A \vee \exists x B $ 也是完全类似的. 本题就相当于上一题的对偶的版本, 使用deMorgan律转化而来. 
\end{enumerate}  
\end{problem}
\end{document}
