\documentclass[a4paper, 10pt]{ctexart} %中文支持
\usepackage{float}              %防止浮动元素浮动
\usepackage{rotating}           %旋转图片
\usepackage{amsfonts}           %对某一些字体之支持
\usepackage{amsmath}          %数学公式
\usepackage{amsthm}             %定义, 定理, 证明, 例子环境的支持
%使用方法:
%\newtheorem{environment name}{caption}
%比如 \newtheorem{example}{这是例子}
%效果 \begin{example} xxx \end{example} -> 这是例子 1 xxx
%proof就不需要了
\usepackage{graphicx}           %插入图片
\usepackage[left=1.5in,right=1.5in,top=1in,bottom=1in]{geometry}   %用来排版的
\usepackage[]{color}            %给部分文本上色的
\usepackage{algorithm}          %写伪代码的
\usepackage{algorithmic}        %同上
\usepackage{minted}             %书写代码
\usepackage{amssymb}            %用来加入一些数学符号, 比如说 $\varnothing$
\usepackage{fontspec}           %不知道用来干嘛的
\usepackage{titlesec}           %用来调整section等的大小和字体
\usepackage{hyperref}           %生成可跳转的书签
\usepackage{blindtext}          %for test
\pagestyle{plain}               %这是调整页面的风格, plain的情况下只有页码
%页眉太几把烦了, 不想管
\titleformat*{\section}{\huge\bfseries}             %管理title的字体和大小
\titleformat*{\subsection}{\Large\bfseries}         %bfseries就是默认的字体.
\titleformat*{\subsubsection}{\large\bfseries}      % 日, content里的不还是没变? 难堪的一笔

\setmonofont{Ubuntu Mono}       %?
\usemintedstyle{custommanni}    %设置minted插入代码的风格

\newtheorem{theorem}{定理}
\newtheorem{example}{Example}
\newtheorem{definition}{定义}
\newtheorem{lemma}{引理}
\newtheorem{proposition}{命题}

% 以上是导言区

\title{test}
\begin{document}
\tableofcontents
\maketitle

$$ c_f \left(u ,v \right) = \begin{cases} c (u,v)  -f (u,v) & \text{if } \left(u ,v\right) \in E \\ f \left( v, u\right) & \text{otherwise} \end{cases} $$

$$ f\uparrow f' \left( u ,v\right)= \begin{cases} f( u , v) + f' \left(u ,v\right) - f\left( v , u\right), &\text{ if } \left(u ,v\right) \in E \\ 0 & \text{ otherwise} \end{cases} $$

我们熟知周期函数 $f$ 可以使用傅里叶级数表示, 设其角速度为 $\omega$, 周期为 $T$, 有
$$ \begin{aligned} f \left(t\right) & = a_0 + \sum_{n=1} ^{\infty} \left[ a_{n} \cos  \left(n\omega t\right) + b_{n} \sin  \left(n \omega t\right)\right] \end{aligned} $$
\section{the outline of logic}
\[
    \text{数理逻辑}
    \begin{cases}
        \text{命题逻辑}
        \begin{cases}
            \text{}\\
            \text{}
        \end{cases}\\
        test\\
        test
    \end{cases}
\]
\section{the development of logic}
\subsection{start from leibniz}
\subsection{三次数学危机}
\section{Propositional Connectives}
\subsection{negation}
\subsection{conjuction, disjuction and conditional}

conditional : 比如说 $a\Rightarrow b$ 的意思是 $\neg a \land b$

\O
\section{Notation}

\end{document}