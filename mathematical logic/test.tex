\documentclass[12pt, a4paper]{ctexart} %中文支持
\usepackage{float}              %防止浮动元素浮动
\usepackage{rotating}           %旋转图片
\usepackage{amsfonts}           %对某一些字体之支持
\usepackage{mathrsfs}           %mathscr e.g.
\usepackage[]{amsmath}          %数学公式
\usepackage{amsthm}             %定义, 定理, 证明, 例子环境的支持
%使用方法:
%\newtheorem{environment name}{caption}
%比如 \newtheorem{example}{这是例子}
%效果 \begin{example} xxx \end{example} -> 这是例子 1 xxx
%proof就不需要了
\usepackage{graphicx}           %插入图片
\usepackage[left=1.30in,right=1.30in,top=1in,bottom=1in]{geometry}   %用来排版的
\usepackage{color}            %给部分文本上色的
\usepackage{algorithm}          %写伪代码的
%\usepackage{algorithmic}       %同上
\usepackage{algorithmicx}
\usepackage{algpseudocode}
%\usepackage{minted}
\usepackage{amssymb}            %用来加入一些数学符号, 比如说 $\varnothing$
\usepackage{titlesec}
\usepackage{fontspec}           %不知道用来干嘛的
\usepackage{hyperref}           %生成可跳转的书签

%\usemintedstyle{vs}    %设置minted插入代码的风格

\titleformat*{\section}{\huge\bfseries}             %管理title的字体和大小
\titleformat*{\subsection}{\Large\bfseries}         %bfseries就是默认的字体.
\titleformat*{\subsubsection}{\large\bfseries}

\newcommand{\N}{\mathbb{N}}
\newcommand{\R}{\mathbb{R}}

\theoremstyle{plain}
\newtheorem{theorem}{Theorem}[section]
\newtheorem{lemma}{Lemma}[section]
\newtheorem{proposition}{Proposition}[section]

\theoremstyle{definition}
\newtheorem{example}{Example}[section]
\newtheorem{definition}{Definition}[section]
\newtheorem*{remark}{Remark}
\newtheorem*{comment}{Comment}
\newtheorem{problem}{Problem}
\newtheorem*{solution}{Solution}
\newtheorem*{sketchoftheproof}{Sketch of the Proof}

\pagestyle{plain}
\title{}
\begin{document}

\section{演绎定理}
We have the deduction theorem. That is a theorem which can 
simplify the proof of some theorem in the Propositional 
Calculus. 

\begin{theorem}
    \begin{equation}
        \Gamma \vdash \alpha \to \beta \iff \Gamma \cup \left\{\alpha   \right\} \vdash \beta
    \end{equation}
\end{theorem}
\begin{proof}[proof by induction]
    {\bfseries 必要性:}

     这个是显然的, 因为我们说, $\alpha \to \beta$ 是后面的一个定理, 
     我们再次使用rmp分离规则就能够证明

     \begin{align*}
        \Gamma \cup \left\{\alpha   \right\} \vdash \alpha\to\beta
     \end{align*}

     {\bfseries 充分性:}
     我们需要使用归纳法. 我们说, 
     emmm, 对长度进行归纳是怎么回事..

     我们已知 $\Gamma \cup \left\{\alpha    \right\} \vdash \beta $ 
     记这个推理过程为 $\left( \beta _{1} , \cdots  , \beta _{n}\right)$ 这个东西的长度为 $n$.
     我们要证明对于里面的所有 $\beta _{i}$ 都有 $\alpha \to \beta_{i}$

     使用归纳法进行证明, 对于 $\beta _{1}$ 只有两种可能: 1. $\beta_{1}$ 是 $\Gamma$ 中的一个 2. $\beta_{1}$ 是其中一个公理.
     这个时候当然有 $\alpha \to \beta_{1}$ , 毕竟后件为真. 

     对于 $i \le n$ 我们说, 假设成立, 有 $\Gamma \vdash \alpha \to \beta _{i}$ 

     对于 $i  = n$ 的时候, 我们要证明 $\Gamma \vdash  \alpha \to \beta_{i}$
     对于 $\beta _{i}$ 是一个推理结果, 就是说, $\beta_{i}$ 是有一个rmp分离规则
     分离出来的 (当然 $\beta_{i}$ 也可以是公理或者是属于 $\Gamma$), 这就是说他是前面两个 $\beta_{j} , \beta_{k}$ 分离出来的, 设 $\beta_{k} = \beta_{j} \to \beta_{i}$, 并且我们还有 
     $\Gamma \vdash  \alpha \to \beta_{j}$, $\Gamma \vdash  \alpha \to \beta_{k}$

     卧槽, 这个时候就简单了: 
     \begin{align*}
        \Gamma \vdash \alpha \to \beta_{k} 
     \end{align*}
     viz.
     \begin{align*}
        \Gamma \vdash \alpha \to \left( \beta_{j} \to \beta_{i}\right) 
     \end{align*}
     viz. 
     \begin{align*}
        \Gamma \vdash  \left( \alpha \to \beta_{j} \right) \to \left( \alpha \to \beta_{i}\right)
     \end{align*}
     然后捏, 使用rmp就行了, 毕竟我们有 $\Gamma \vdash \alpha \to \beta_{j}$. 我们就有 $\Gamma \vdash \alpha \to \beta_{i}$, 
     然后 $\beta_{i} = \beta$, 因为 $\beta_{i}$ 是最后一个. 
\end{proof}
\begin{comment}
这个定理好用的一批, 只不过不让用. 
\end{comment}
\begin{example}
    Prove
    \begin{equation}
        \vdash \left(A \to \left(B \to C\right) \right) \to \left(\left(C \to D\right) \to \left(A \to \left(B\to D\right)\right)\right)
    \end{equation}
    \begin{proof}[使用演绎定理证明]
        Trivial !
    \end{proof}
\end{example}
\begin{example}
    证明
    \begin{equation}
    \vdash  \left( \left(A \to B \right) \to \left(A \to C\right) \right) \to \left(A \to \left( B \to C\right)\right)
    \end{equation}
\end{example}
\begin{comment}
    这个例子实际上是 A2 的逆命题. Wow, impressive.
\end{comment}
\section{Completeness and Soundness of Propositional Calculus}
\begin{theorem}[合理性]
    如果说 $\Gamma \vdash A$, then $\Gamma \vDash A$, 或者说 $\Gamma \Rightarrow A$
\end{theorem}
这里的这个 $\vDash$ 是另一个方面的东西, 就是另一本书 logic for application 之中介绍的 tableau proof. 
但是那里讲的比较诡异, 毕竟没有介绍如何证明 $\Gamma \vDash \alpha$ 只介绍了 $\vDash \alpha$. 即, 如何证明 
tautology. 

艹能不能讲点动机. 你妈的. 

\begin{theorem}[一致性]
在 Propositional Calculus 之中, 不存在 $\Gamma $ 使得 $\Gamma \vdash A , \Gamma \vdash  \neg A$.
\end{theorem}
\begin{remark}
    如果说是不一致的, 那么因为反证法,  $\Gamma$ 能够推出任意的公式.
\end{remark}
\begin{proof}[proof of 一致性]
    反证法, 加上这个合理性就能够证明. 简单来说, 我们不能确定 $A^{v}$ 的值, 它既是1又是0.    
\end{proof}
\begin{definition}[完全性]\label{def:complete}
    已知 Propositions 的一个 collection $\Gamma$, 如果说对于任意的公式 $A $, 都有 
    \begin{equation}
        \Gamma \vdash  A \text{ or } \Gamma \vdash \neg A
    \end{equation}
\end{definition}
但是, 我们说, 在 Propositional Calculus 之中, 并不是所有的公式集合都是这样的. 
\subsection{part 2}
\begin{definition}[theory]
    Theory 是这样定义的. 
    \begin{equation}
    \text{Th}\left(PC\right)\equiv \left\{ A \mid \vdash A\right\}
    \end{equation}
\end{definition}

完备性定理: 艹, 能不能多一点介绍.
\begin{theorem}[Completeness]
    \begin{equation}
        \Gamma \Rightarrow \alpha \implies \Gamma \vdash \alpha
    \end{equation}
\end{theorem}
分为很多命题来进行证明 

\begin{enumerate}
\item $\Gamma$ 是一致的, 如果说 $\Gamma \vdash A$ 不成立, 那么说 $\Gamma \cup  \left\{\neg A\right\}$ 是一致的. 一致的定义见前面.
\item $\Gamma$ 是一致的, 如果说 $\Gamma \vdash  A$, 那么说 $\Gamma \cup \left\{A\right\}$ 也是一致的. 这两个定理是用来扩大的, 就是将前面这个东西. $\Gamma$ 进行扩张, 可以说是数学家喜欢干的事情. 
\item $\Gamma $ 是一致的, 存在公式集合 $\Delta $ s.t. $\Gamma \subseteq \Delta$, $\Delta$ 是一致的, 并且 $\Delta$ 是完全的. 简单来说, 一直扩大, 像一个闭包一样. 使用上面两个命题进行构造, 对于每一个命题公式 $A$ 都进行了扩大. 于是, 对于任意的 $A$, 都有 $\Delta \vdash A$ 或 $\Delta \vdash \neg A$, 且 $\Delta $ 一致. 就是我们给定了 $\Gamma $, 构造了定义良好的 collection $\Delta$. 

完全性的定义见 Definition~\ref{def:complete}. 
\begin{sketchoftheproof}我们说, 这个命题的证明, 是非常数学的证明, 其他科里好像也有类似的东西, 就是整一个 bigcup. 
    我们找到一个 $A$, 有 $\Gamma \vdash  A$ 或者说 $\Gamma \vdash A$ 不成立. 进行对其进行扩张 $\Delta _{1} = \Gamma \cup  \left\{ A\right\} $ 或者说 $\Delta _{1}  = \Gamma \cup  \left\{ \neg A\right\} $. 一直扩.
\end{sketchoftheproof}
\begin{remark}
这里需要注意极限. 说实话非常棘手.
\end{remark}
\item 1. 如果说 $\Delta \vdash  A$ , 那么 存在 $n$ s.t. $\Delta _{n} \vdash A$
只需要注意到, 任意一个 Propositional Calculus 之中的`证明'都是有限序列.

2. 我们还要证明 $\Delta $ 是一致的. 为什么要证明? 这是因为 $\Delta = \bigcup^{\infty}_{n=0} \Delta _{0}$, 这是无限并, 所以需要证明. 我们使用反证法证明. 
\end{enumerate}
\begin{comment}
    这里就不讲了, 哈哈, 考试不考, 这样一说谁都不看了. 当然啦, 这个东西是整个 Propositional Calculus 之中最重要的东西. 乐. 
\end{comment}
\end{document}