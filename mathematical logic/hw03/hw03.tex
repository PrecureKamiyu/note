\documentclass[a4paper, 12pt]{ctexart} %中文支持
\usepackage{float}              %防止浮动元素浮动
\usepackage{rotating}           %旋转图片
\usepackage{amsfonts}           %对某一些字体之支持
\usepackage{amsmath}            %数学公式
\usepackage{amssymb}            %用来加入一些数学符号, 比如说 $\varnothing$
\usepackage{amsthm}             %定义, 定理, 证明, 例子环境的支持
%使用方法见下面的例子
\usepackage{graphicx}           %插入图片
\usepackage[left=1.5in,right=1.5in,top=1in,bottom=1in]{geometry}   %用来排版的
\usepackage{color}              %给部分文本上色的
\usepackage{algorithm}          %写伪代码的
\usepackage{algorithmic}        %同上
%\usepackage{minted}             %书写代码
\usepackage{fontspec}           %不知道用来干嘛的
\usepackage{titlesec}           %用来调整section等的大小和字体
\usepackage{hyperref}           %生成可跳转的书签
\usepackage{blindtext}          %for test

\pagestyle{plain}               %这是调整页面的风格, plain的情况下只有页码

%titlesec
\titleformat*{\section}{\huge\bfseries}             %管理title的字体和大小
\titleformat*{\subsection}{\Large\bfseries}         %bfseries就是默认的字体.
\titleformat*{\subsubsection}{\large\bfseries}      

%\setmonofont{vs}                %调整这个minted代码的风格
\theoremstyle{plain}
\newtheorem*{comment}{Comment}

\theoremstyle{definition}
\newtheorem{theorem}{\llap{Theorem}\space}
\newtheorem{example}{\llap{Example}\space}
\newtheorem{definition}{\llap{Definition}\space}
\newtheorem{lemma}{\llap{Lemma}\space}
\newtheorem{proposition}{\llap{Proposition}\space}
\newtheorem{num}{}
\newtheorem*{solution}{Solution}

\theoremstyle{remark}
\newtheorem{remark}{Remark}

% 以上是导言区
\title{zuoye3}
\author{Your \and mother}
\begin{document}
\maketitle
\begin{num}
使用演绎定理在Propositional Calculus中证明
\begin{enumerate}
    \item[(1)] $\vdash \left(B \to A \right) \to \left( \neg A \to \neg B\right)$
    \item[(2)] $\vdash \left(A \to B\right) \to \left( \left( B \to C\right) \to \left(A \to C\right)\right)$ 
    \item[(3)] $\vdash \left(\left(A \to B\right) \to A      \right) \to A  $ 
    \item[(4)] $\vdash \neg \left(A \to B   \right) \to \left(B \to A\right)$
\end{enumerate}
\end{num}
\begin{solution}
    \begin{enumerate}
        \item[(1)]
        \begin{align*}
            & \vdash \left(B \to A  \right) \to \left( \neg A  \to \neg  B\right)\\
            \Leftrightarrow & \left(B \to A \right)  \vdash \left( \neg  A \to \neg B\right) \\ 
            \Leftrightarrow & \left( B \to A \right) , \neg A  \vdash \neg B
        \end{align*}
        \begin{align*}
            (1)\quad &\neg A \to \left(A \to \neg B \right)  \tag{thm 6}\\
            (2)\quad &\neg A \tag{已知} \\
            (3)\quad &A \to \neg B \tag{rmp} \\ 
            (4)\quad &B \to A  \tag{已知} \\
            (5)\quad &B \to \neg B \tag{(4),(3),thm 8} \\
            (6)\quad &\left(B \to \neg B \right) \to \neg B \tag{thm 10}\\
            (7)\quad &\neg B \tag{rmp} 
        \end{align*}
        \item[(2)]
        \begin{align*}
            & \vdash  \left(A \to B\right) \to \left(B \to C\right) \to \left(A \to C\right)\\
            \Leftrightarrow & \left(A \to B \right) \vdash \left(B \to C\right) \to \left(A \to C\right) \\
            \Leftrightarrow & \left(A \to B \right) , \left(B \to C\right) \vdash \left(A \to C\right) \\
            \Leftrightarrow & \left(A \to B \right) , \left( B \to C\right) , A \vdash C
        \end{align*}
        \begin{align*}
            (1)\quad &A \tag{已知} \quad \quad \quad \quad \\
            (2)\quad &A \to B \tag{已知} \quad \quad \quad \quad \\
            (3)\quad &B \tag{rmp} \quad \quad \quad \quad \\
            (4)\quad &B \to C \tag{已知} \quad \quad \quad \quad \\
            (5)\quad &C \tag{rmp}\quad \quad \quad \quad \\
        \end{align*}
        \item[(3)]
        \begin{align*}
            &\vdash \left( \left(A \to B \right) \to A  \right) \to  A \\
            \Leftrightarrow & \left(A \to B \right) \to A  \vdash A 
        \end{align*}
        \begin{align*}
            (1)\quad&\neg A \to \left(A \to B \right) \tag{thm 6} \\
            (2)\quad&\left(A \to B \right) \to A \tag{已知} \\
            (3)\quad&\neg A \to A \tag{(1),(2),thm 8}\\
            (4)\quad&\left(\neg A \to A \right) \to A \tag{反证法}\\
            (5)\quad&A \tag{(3),(4),rmp}
        \end{align*}
        \item[(4)]
        \begin{align*}
            & \vdash \neg \left(A\to B \right) \to \left(B \to A \right) \\
            \Leftrightarrow & \neg \left(A \to B \right) \vdash \left(B \to A \right)
        \end{align*}
        \begin{align*}
            (1) \quad&B \to \left(A \to B \right) \tag{A1}\\
            (2) \quad&\neg \left(A \to B \right) \to \neg  B\tag{逆否}\\
            (3) \quad&\neg B \to \left(B \to A\right) \tag{thm 6}\\
            (4) \quad&\neg \left(A \to B \right) \tag{已知} \\
            (5) \quad&\neg B \tag{(2),rmp}\\
            (6) \quad&B\to A \tag{(3),rmp}\\
        \end{align*}
    \end{enumerate}
\end{solution}
\begin{num}
    将A3换为 $\left(\neg A \to B\right) \to \left(\left(\neg A \to \neg B \right) \to A\right)$, 设这个得到的系统为 $\text{PC}_{1}$, 证明 
    \begin{enumerate}
        \item[(1)] $\vdash _{\text{PC}} \left(\neg A \to B \right) \to \left(\left( \neg A \to \neg B\right) \to A\right)$
        \item[(2)] $\vdash _{\text{PC}_{1}} \left(\neg A \to \neg B\right) \to \left( B \to A \right)$ 
    \end{enumerate}
\end{num}
\begin{solution}
\begin{enumerate}
    \item    \begin{align*}
        (1)\quad & \left(\neg A  \to \neg B \right) \to \left(\neg A  \to \neg B\right) \tag{thm 1} \\
        (2)\quad & \neg A  \to \left( \left( \neg A  \to \neg B     \right) \to \neg B     \right) \tag{rmp, thm 3} \\
        (3)\quad & \neg A  \to \left(B \to \neg  \left(\neg A \to \neg B \right)\right) \tag{rmp, thm 15} \\
        (4)\quad & \left(\neg A  \to B \right) \to \left( \neg A \to \neg \left( \neg A \to \neg  B\right)\right)\tag{rmp, A2}\\
        (5)\quad & \left(\neg A \to B \right) \to \left( \left(\neg A \to \neg B \right) \to  A    \right) \tag{thm 8, A3}
    \end{align*}
    \item 
    因为PC之中的thm 3以及thm 8的证明并没有使用到 A3, 那么在 $\text{PC}_{1}$ 之中也显然成立. 
    \begin{align*}
        (1)\quad& \left(\left(\neg A \to B \right) \to A   \right) \to \left( \left( \neg A \to B \right)  \to A \right) \tag{thm 1} \\
        (2)\quad& \left(\neg A \to B \right) \to \left( \left( \neg  A \to B \right) \to  A    \right) \to A \tag{rmp, thm 3}\\
        (3)\quad& B \to \left( \neg  A\to B \right) \tag{A1} \\
        (4)\quad& B \to \left(\left( \neg A \to B \right) \to A \right) \to  A \tag{(2),(3),thm 8}\\
        (5)\quad& \left(\left(\neg  A \to B \right) \to A \right) \to \left(B \to A \right) \tag{rmp, thm 3} \\
        (6)\quad& \left(\neg A \to B\right) \to \left(\neg A \to \neg B \right) \to A \tag{A3$'$}\\
        (7)\quad& \left(\neg A \to \neg B \right) \to \big(\left(\neg A  \to B \right) \to A \big)\tag{rmp, thm 3}\\
        (8)\quad& \left(\neg A \to \neg B \right) \to \left( B \to  A\right) \tag{(5),(7),thm 8} \\
    \end{align*}
\end{enumerate}
\end{solution}
\end{document}