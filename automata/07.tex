\documentclass[../main.tex]{subfiles}
\begin{document}
\section{Shin chapter seven}
\label{sec:Shin chapter seven}
\begin{itemize}
\item What is the outline of the chapter? 
	\begin{itemize}
		\item [1] We learn about how to get Chomsky Normal Form
		which is very important for us to
		prove the pump lemma of the 
		context free language. 
		\item [2] We talk about the pump lemma of the 
		context free language, and we prove it. 
		\item [3] We talk about the close property of 
		the context free language. 
	\end{itemize}

\end{itemize}


\subsection{Simplification and Chomsky Noraml Form}
\begin{itemize}
\item 	What is CNF? 

The grammar with only the productions with the form 
\[
	A \to a \quad \text{ or } \quad A \to BC
\]
\item Why simplification? 

Because we need it to convert the grammar into CNF. 

\item What to simplify? 

We need to eliminate \textbf{useless} symbols and 
\(\boldsymbol{\epsilon}\)\textbf{-productions} and 
\textbf{unit productions}. 

\item What is \textbf{useful} symbols? 

Symbols that are used in the derivation of a string of the 
language are \textbf{useful}, that is to say if a symbol is 
useless, it shall not appear during the derivation and thus 
is indeed useless. 

\item What is a \textbf{generating} symbol? 

Symbols that suits that \(A \to w\), where \(w\) is in \(T^{*}\)
are generating symbols. 

\item What is a \textbf{reachable} symbol?

Symbols that can be derived from the start variable \(S\). 

\item What is the procedure to get rid of the 
useless symbols? 

\textbf{First}, get rid of non-generating symbols. \\
\textbf{Next}, get rid of non-reachable symbols.

\item Do we have to first get rid of non-generating symbols and then next? 

Yes. You may get wrong answer otherwise. 

\item Can you prove that? 

Yes. Let us we have a grammar \(G\), and we employ 
the procedure one, we have \(G _{2}\), and after 
next procedure, we have \(G_{1}\). We need to 
prove that \(V _{1} \cup T_{1} \) are all useful. 

First given a symbol in \(V _{1} \cup T _{1}\), \(X\). It 
should be a symbol after the second procedure, so it should 
be a symbol that is reachable in \(G _{2}\). And we can 
easily notice that \(X\) is also generating in \(G\). 
		That is to say \(X \mathop{\Rightarrow}\limits_{G} w\). And we have that \(X \mathop{\Rightarrow} \limits_{G_{2}} w\). (Since we know that \(X\) is generating in \(G\)).

		From the fact that \(X\) is reachable in \(G_{2}\), that is to say \(S \mathop{\Rightarrow} \limits_{G_{2}}\alpha X\beta \), we know that all the symbols that is used in 
		\(S \mathop{\Rightarrow}\limits _{G_{2}}\alpha X\beta\) are reachable and thus \(S \Rightarrow _{G_{1}}\alpha X\beta  \) is true in \(G _{1}\)  since the second procedure 
		does not eliminate the symbols that used in the 
		derivation (that is because they are reachable in \(G_{2}\)). Thus we have proven that if \(X \in V_{1} \cup T_{1}\), then we have that \(S \Rightarrow\alpha X\beta \Rightarrow\alpha w\beta\)

		Lastly we have to prove that \(\alpha\) and \(\beta\) consist of only generating symbols. And that is trival since it is true that \(S \mathop{\Rightarrow} \limits_{G_{2}} \alpha X\beta\), which is already implies that the symbols are 
		generating since all the symbols in \(G_{2}\) are 
		generating. 

		\textbf{Most} importantly, 
		we want to prove that 
		\[
		S \Rightarrow\alpha X\beta \Rightarrow xwy
		\]
		in \(G_{1}\) the first part use that fact that 
		\(X\) is reachable in \(G_2\) and thus \(X\) is reachable in \(G_{1}\). 
		For the second part, we know that \(X\) is of course generating in \(G\) because \(X\) is also symbol in \(G_{2}\). From the fact that \(X \Rightarrow _{G} w\), we have that 
		\(X \Rightarrow _{G _{2}} w\) since the fact implies that \(X \) is generating and all the symbols are generating. 
		And since \(X\) is reachable, then we have that 
		\(X \Rightarrow _{G_{1}} w\), since the symbols in the derivation are not eliminated.

		Similarly, the symbols in \(\alpha\) and \(\beta\) are just like \(X\). They are all reachable and generating in\(G_{1}\). 
		During the proof you shall see that indeed we have to eliminate non-generating symbols first. 
\end{itemize}
\hrule
\begin{itemize}
\item What is next? 

		We have to eliminate the \(\boldsymbol{\epsilon}\)-production. 
\item How? 

		First we have to find all the nullable symbols.
\item What is a nullable symbol? 

		A nullable symbol \(A\) is a symbol suit that \(A \Rightarrow\epsilon\). 
\item Does \(A\) have to be a variable? 

		Yes. 
\item How do we find all the nullable variable? 

		We shall use an algorithm. Basic: If \( A \to\epsilon\), then \(A\) is nullable. Induction: If \(A \to B_1 \dots B_{k}\), and \(B_{i}\)'s are nullable then \(A\) is nullable.

\item Does the algorithm find all the nullable symbols? 

	Yes, you can prove that. 

\item Can you prove it?

	No, I forgot how, but I remember that 
	one should prove that 
	if \( B \Rightarrow\epsilon\) then \(B\) can be found by the algorithm and use 
	induction on the length of the derivation of a nullable symbol that produce \(\epsilon\).
\end{itemize}

\hrule 

\begin{itemize}
\item What is next? 

We have to eliminate \(\epsilon\)-productions.

\item How? 

If \(A \to\alpha X\beta\) and \(X\) is nullable, we have that 
\(A \to\alpha X\beta \mid\alpha\beta\), treating \(X\) as null. 

\item What is more? 

Of course we have to eliminate the \( A \to\epsilon\) productions.

\item So the result grammar can not produce \(\epsilon\). 

Yes, \(L' = L - \{ \,\epsilon \, \}\) 

\item Can you prove it? 

Maybe not, but surely you can. 
\end{itemize}

\hrule 

\begin{itemize}
\item What is unit productions? 

\(A \to B\) where \(B\) is a variable, is a unit production.

\item How to determine whether two variables are in unit production? 

If \(A \to B\), we use \( (A, B)\) to denote that \(A\) can 
produce \(B\). 

\item How can I find all the unit production? 

	If \( (A, B)\) is given and \(B \to C\) is 
also given, then we have \((A, C)\).

\item Can \dots

I don't want to prove it.

\item How to eliminate the unit production

Given \(( A, B)\), \(A\) gets all the productions (except for unit productions) of \(B\). 

If we have that \(A \to\alpha  \mid\beta\mid B\) and \(B \to\gamma \mid \tau\), then after we have that 
\[
A \to\alpha \mid \beta \mid\gamma \mid \tau
\]
notice that the unit production is gone. 
\item \dots
\end{itemize} 
\hrule 
\begin{itemize}
\item What is Chomsky Normal Form? 

The grammar whose productions are in form \(A \to a\) or \(A \to BC\). 

\item How to convert a gammar (which is simplified) to CNF? 

First, for the productions whose bodies contain both 
terminates and variables, let us say the terminate are \(a_{i}\), we introduce productions like \(A _{i} \to a_{i}\), making the original productions bodies contains only the variables. 

Second, for productions whose bodies have three or more variables like \(A \to X_1 X_2 \dots X_{n}\). We introduce the 
variables and corresponding productions like 
\(A \to X_1 B_1\) and 
the rest is trivial. 

\item What do you mean \emph{trivial}? 

It means that I don't wanna say it.
\end{itemize} 
% subsection:Simplification and Chomsky Noraml Form 

\subsection{The pump lemma}
\begin{itemize}
\item What is pump lemma? 

It says that if \(w\) is in \(L\) and suits pump lemma, then 
part of \(w\) can be pumped.

\item What do you mean \emph{pumped}? 

For example, if \(w = x y z \), and \(y\) is pumped, then 
we will get \(w' = x y ^{i} z\), for any \(i \in \mathbb{N}\).

\item What is pump lemma in context free grammar? 

If \(L\) is context free grammar, then \(\exists n \in N\), such
that for all \(w \in L\) suits that \(\vert w \vert \ge n\), 
suits that there exists a partition of \(w =  uv w xy\) such that
\begin{itemize}
\item \(\vert   vw x \vert \le n\) 
\item \(vx \ne\epsilon\)
\item \(w ' =  u v^{i} w x ^{i} y \in L\), for all \(i \in \mathbb{N}\)
\end{itemize}

\item Can you prove it? 

Yes it requires the knowledge of Chomsky Normal Form and 
that of the parse tree. 

\item What is parse tree? 

It is the tree that tells the procedure of derivation. 

\item Why Chomsky Normal Form 

	Because the parse tree of CNF is a binary tree (almost a binary tree).

\item Why binary tree? 

	If binary tree is with height of \(n\) (which is to say that the longest path in the tree have \(n\) edges), then
	the length of the yield is less than \(2 ^{n-1}\)
\item Why is that important? 

We can use the fact to make that the longest path
have at least two identical variables, which help us to
complete the prove.

\item It is true that when the yield is no shorter than \( 2 ^{n}\),
the height of the tree is no less than \(n+1\)? 

Yes, you can prove that. Note that the tree is a little bit longger. 

\item How to make that there are two identical variables on the 
	longest path? 

If you choose \(n = 2 ^{m}\) and the \(m\) is equal to the number
of the variables, then the longest path may have more than \(m+1\) edges. 
Note that the path is ended with a terminate, so on the path, there should be \(m + 1\) variables (since that there are \(m\) edges).
And then there are at least two identical variables on the path. 

\item Why are they useful? 

Let us say that \(A _{i} = A_{j}\), \(i < j\). We assume that 
\(A _{j}\) derive \(w\), and \( A _{i}\) derive \(v w x\). 
We can assert that \(A_{i}\) derive \( v ^{i} w x ^{i}\). 

\item Why? 

Let us say that \(A_{i} = A_{j}= A\). We have that 
\[
A \Rightarrow\alpha A \beta 
\]
where \(\alpha \Rightarrow v \) and \(\beta \Rightarrow x\). 
And it is also true that 
\[
A \Rightarrow\alpha\alpha A \beta\beta
\]
So it is also true that if we substitute \(A\) with \(\alpha A\beta\) we have
\[
A \Rightarrow\alpha ^{i} A\beta ^{i} , \quad i \ge 1
\]
And because \( A \Rightarrow w\), so \(i\) could be \(0\).

\item Cool, man. I can fill with details.
\end{itemize}

\hrule

\begin{itemize}
\item How to use pump lemma to prove that a language is not a 
context free language? 

Note the pump lemma can restate as 
\[
	L \text{ is context free } \to L \text{ can be pumped}
\]
Then we have 
\[
	L \text{ can't be pumped } \to L \text{ is not context free}
\]
We need to prove that \(L\) can't be pumped. 

\item Hey, I already know that ``can be pumped'' can be 
restate as 
\[
\exists n \forall \omega, \vert \omega\vert \ge n, \exists uvwxy, uvwxy  = \omega (\forall i \in \mathbb{N} , u v ^{i} w  x ^{i} y  \in L)
\]
then ``can't be pumped'' is that 
\[
	\forall n \exists \omega ,\vert \omega \vert \ge n , \forall uvwxy , uvwxy  = \omega (\exists i \in \mathbb{N}, uv ^{i} w x ^{i} y \notin L)
\]
Is that true?

Yes. 
\item So we shall prove that statement to show that 
	\(L\) is not context free.

Yes.
\end{itemize}
% subsection:The pump lemma 

\subsection{The closure property}
\begin{itemize}
\item What \emph{close}? 

We want to know context free language is closed under what kind of operations.

\item What kind of the operation we are going to talk about? 

\begin{itemize}
\item \emph{Substitution}
\item \emph{intersection with regular language}
\item \emph{inverse homomorphism}
\end{itemize}

\item What is a \(s\)? 

\(s\) is a function called \emph{substitution}. 

\item What is a \emph{substition}? 

A substitution is function that 
map a letter to a language. 
\[
s \colon a \mapsto L_{a}
\]

\item It seem weird. 

Indeed, but when comes to a string, let us say \(w\), things get
clear. 
\[
s \colon w \mapsto L_{a_1}L_{a_2}\dots L_{a_{n}}
\]
where \(w = a_1 a_2 \dots a_{n}\).

\item How about when it comes to a language? What is \(L\) mapped to? 
\[
	s \colon L \to \bigcup_{w \in L} s (w)
\]

\item So is it close? 

Yes, if \( \forall a \in \Sigma\), \(L _{a}\) is context free,
		and \(L\) is context free, then \(s (L)\) is 
		context free.

\item Can you prove it? 

The proof is trivial.

\item What do you mean \emph{trivial}? 

	\dots
\end{itemize}
% subsection:The closure property 
% section Shin chapter seven 
\section{Pump lemma and its applications}
\label{sec:The Pump Lemma of grammar}

\subsection{The content of Pump Lemma}
\begin{thm}[Pump Lemma]
\label{Pump Lemma}
Let \(L \) be the language of a grammar. Then there exist \(n\) that if \(|z| \ge n\), then \(z = uvwxy\), suit the following 
conditions: 

	\bigskip
	\setlength{\hangindent}{33pt}
	1. \(| v w x | \le n\). That is to say \( |v  w x |  \) can't be too long.

	\smallskip
	2. \(v x \ne\epsilon\). Since \(v,  x \) are the pieces to be pumped, \(v\) or \(x\) should not be zero, that is to 
	say, \(v\) and \(x\) can be both empty, \emph{and} that is to say \( v x \ne\epsilon\).

	\smallskip 
	3. For all \(i \ge 0\), \(u v ^{i} w x ^{i} y\) is in \(L\).
\end{thm}
% theorem Pump Lemma 
% subsection:The content of Pump Lemma  

\subsection{The application of Pump Lemma}
Similarly, the Pump Lemma is used to prove a language \(L\) is not a context free language. Let us restate the lemma using the 
mathematical logic. If \(L\) is a context free language then 
\[
\exists n \in N \ \forall\alpha \in L (\forall uvwxy =\alpha (|v w x | \le n , vx \ne\epsilon \to  u v ^{i} w x ^{i} y \in L))
\]
Let the proposition above be \(A\), then we have \(L \text{ is CFL} \to A\). Thus, we have \(\neg A \to L \text{ is not CFL}\). 
And \(\neg A\) equates 
\[
\forall n \in N ,\exists\alpha \in L ,\exists u v w x y = \alpha (\neg  (|vwx| \le n , vx \ne \epsilon \to u v ^{i} w x ^{i} y \in L))
\]
So we have the procedure here, similar to the one in previous chapter about the formal expressions, that we check for every \(n\) in \(\mathbb{N}\), considering as a random variable\footnote{not that variable}, and find a \(\alpha \in L\), and prove that for all kinds of \( uvw xy\), there exists \(i\) s.t. \( u v ^{i} w x ^{i}y\) is not in \(L\), where we often discuss about the different conditions. 

\begin{exam}
	Use pump lemma to show that \(L = \{ 0 ^{n} 1 ^{n} 2 ^{n} \mid n \ge 1 \}\) is not context free language.
\end{exam}
There is another example to show that the grammar can't describe the string that have two pairs of equal numbers of symbols. 
\begin{exam}
	Let \(L = \{ 0 ^{i} 1 ^{j} 2 ^{i} 3 ^{j} \}\). Use pump lemma to prove that \(L\) is not context free language. 
\end{exam}
Mover, since pushdown automata are equivalent to context free grammar, you can easily see that (not prove that) 
\(L =  \{ ww\}\) is not context  free language. 
\begin{exam}
	Let \(L = \{ ww \}\). Use pump lemma to prove that \(L\) is not context free language.

	Given a \(n\), we shall prove that if \(z \in L\), and \(z  =  uv w xy\), we have that \( u w y\) does not in \(L\) which leads to
	contradiction.  We shall let \(z =  0 ^{n} 1 ^{n} 0 ^{n} 1 ^{n}\). 

	\setlength{\hangindent}{33pt} 
	\medskip
	1. Let us talk about \( v w x \) first. Since \( |v w x | \le n\), we assume that \(v w x \) is all in the first 
	block of \(0\)'s. Let \( |v x |\) be \(k\). Then, \( |  u w y | = 4n - k\) and moreover, since \(v w x\) is all 
	in the first block, \( u w y \) starts with \( 0 ^{n - k} 1 ^{n}\) for sure. If \( u w  y = tt\) for some \(t\),
	then \( |t| = 2n - k/2 \ge 2n - k\), viz., the length of \(t\) is longer than that of \( 0 ^{n - k} 1 ^{n}\), and 
	thus \(t\) should end with \(0\). However, \( uwy \) ends with \(1\). If \( u wy\) is \(tt\) then it should have ended 
	with \(0\) since \(t\) end with \(0\), which, leads to a contradiction.
	\medskip

	2. Suppose that \( v w x\) straddles the first block of \(0\)'s and the first block of \(1\)'s.	
	The same, we assume that \( u w y\) can written as \(tt\). Since \(k\), which is the length of 
	\(v x\), is no bigger than \(n\)\footnote{That is because \(|v w x| \le n\)}, we have \( |uwy| \ge 3n\).
	Thus \( |t | \le 3n / 2\). There are two possiblily: 1. \( v x \) contains no \(1\); 2, \(v x\) contains 
	at least one \(1\). For situation 1., the discussion in (1) is also applied. For situation 2., We assert 
	that \(t\) does not end with \(1 ^{n}\), for there is at least one \(1\) in \(v x\) and it is deleted from 
	\(z\), while \( uw y\) ends with \(1^{n}\), which is for sure.

	\medskip 
	3. The further discussion is omitted here.  You can check page 285 of the textbook for help.
\end{exam}
% subsection:The application of Pump Lemma 
% section The Pump Lemma of grammar 

\end{document}
