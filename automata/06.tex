\documentclass[../main.tex]{subfiles}
\begin{document}
A pushdown automata is a type of automation that use a stack structure, with the remaining part the same as the NFA (non-definite finte state automata). With the repect to the stack the pushdown automata has gained a more greater ability to describe the language. We will in the further section prove that pushdown automata has just the same ability describing as Context free grammar.

\section{The definition of PDA}
\label{sec:The definition of PDA}

What is new in the pushdown automata is that there is a stack that we can operate when we receive a character.
When it receive a character of \( \Sigma\), the automata is going to next state based on three object: 1. the current state; 2. the received character of \( \Sigma\); 2. the character on the top of the stack.

If you wrote out the defintion of state-transition function, it would be something like: \(\delta \colon Q \times \Sigma \times Z \to Q\), where \(Z\) denote the character in the stack (and of course that \(Z\) can equate \(Q\)).

Let's take the pushdown automata that describe the language \(L = \{ w w ^{R} \}\) as an example.

\begin{exam}
	{\sl
It is clear that we should push \(a\) into the stack if we receive \(a\), before we finish going through the string \(w\). And it is the same clear that we should get the top of the stack and check if it is the same as the received character after we go through the string \(w\).

The tricky one is that we have to check these two kinds of situation at the same time, for that every time we get a character, it could be that the \(w\) is done or not.

We can use NFA with two state to construct the pushdown automaton we need.

We say that the automata is at \(q_{0}\) saying that we are at \(w\), and the state \(q_1\) is saying that we are at \(w ^{R}\). And before we receive a character we use a \(\epsilon\) transition from \(q_{0}\) to \(q_1\),
after figuring out which, all is clear.
}
\begin{figure}
	\centering
\begin{tikzpicture}[>=latex,node distance=3cm,on grid,auto]
   \node[state,initial] (q_0)   {$q_0$};
   \node[state] (q_1) [right=of q_0] {$q_1$};
    \path[->]
    (q_0) edge  node {\(\epsilon\)} (q_1)
          edge [loop above] node {push} ()
	(q_1) edge [loop above] node {pop if equates} ();
\end{tikzpicture}
\end{figure}
\end{exam}

\subsection{The precise definition of the pushdown automata}
A pushdown automaton is a seven tuple: \(P = (Q , \Sigma , \Gamma ,\delta , q_{0}, Z _{0} , F)\).
\begin{quotation}
	\( \Gamma\) is the set of the character that appears in the stack

	\smallskip
	\( Z _{0}\) is the initial character that located in the stack.

	\smallskip
	Moreover, the state-transition function should be like:
	\[
		\delta _{0} \colon Q \times \Sigma \times \Gamma \to \mathfrak P (Q \times (V\cup T)^{*})
	\]
	for that the the pushdown automata is built on a NFA.
\end{quotation}
Also, the element on the top of the stack is always popped. If you want the stack to remain un-changed, you can define something like \(\delta (q_{i}, a,\alpha) = \{ \ip {q_{j}}\alpha \}\)


\begin{exam}
Let us just skip the blahblah and draw a diagram that shows how a pushdown automaton work.
\begin{figure}
	\centering
\begin{tikzpicture}[>=latex,node distance=4cm,on grid,auto]
   \node[state,initial] (q_0)   {$q_0$};
   \node[state] (q_1) [right=of q_0] {$q_1$};
   \node[state,accepting] (q_2) [right=of q_1] {\(q_{2}\)};
    \path[->]
	(q_0) edge  node { \(%
	\begin{aligned}%
		&\epsilon, 0 / 0 \\[-5pt]
		&\epsilon , 1 / 1 \\[-5pt]
		&\epsilon , Z_{0} / Z _{0}
	\end{aligned}\) } (q_1)
          edge [loop above] node {push} ()
	(q_1) edge [loop above] node {pop if equates} ()
	      edge node { \(\epsilon, Z _{0} / Z_{0}\)} (q_2);
\end{tikzpicture}
\end{figure}
\end{exam}
% subsection:The precise definition of the pushdown automata

\subsection{Instantaneous Descriptions of a Pushdown automata}
We introduce the instantaneous descriptions of PDA to show how
a PDA accept a string.

A string is given and the string is the
input of the PDA. What will happen in the PDA is that part of
the string is received and state changes and stack is pushed into
character or the other way. Then the information in the middle of
the procedure contains three parts: 1. the state; 2. the string in
the stack; 3. the remaining string.
\begin{enumerate}
\item the state
\item the string in stack
\item the remaining string
\end{enumerate}
So we can describe the middle state of the whole PDA with a three
tuple---\((q, w ,\gamma)\), where \(q \in Q\), \( w\in \Sigma ^{*} \), \(\gamma \in \Gamma ^{*}\). We use \(\vdash\) to indicate that
an ID can transit to the other. Then we know that if \(\delta (q , a , X )\) contains \( ( p ,\alpha)\) we will know that
\[
	(q , a w , X\beta ) \vdash ( p , w ,\alpha \beta )
\]
It is very similar to the function \( \hat\delta\) in the finite state automata. And it is very similar to the symbol \( \Rightarrow\)
in the context free grammar.   Similarly we use \( \overset{*}{\vdash}\) to indicate that one ID can transit to the other after zero or one or more
than one character are received.

If \((q_{1} , w_{1} ,\gamma _{1}) \vdash ^{*} (q_{2}, w_2,\gamma_2)\) then
\((q_{1} , w_{1} ,\gamma _{1}) \sim (q_{2}, w_2,\gamma_2)\)


\begin{thm}[Deduction principle]
\label{Deduction principle}
	\(P  = ( Q , \Sigma , \Gamma ,\delta , q _{0}, Z _{0}, F)\)
	is a PDA. If \((q , x ,\alpha )\vdash ^{*} ( p, y  ,\beta)\) , then
	\[
		(q,  x w ,\alpha \gamma ) \vdash ^{*} (p , y w, \beta \gamma)
	\]
\end{thm}
% theorem Deduction principle

\begin{proof}
The proof is trivial.
\end{proof}
However, it is worth noting that the converse of the thm is not
true.
% subsection:Instantaneous Descriptions of a Pushdown automata

% section The definition of PDA

\section{Pushdown automata and Context Free Grammar}
\label{sec:Pushdown automata and Context Free Grammar}

Pushdown automata and context free grammar have the same ability to describe a 
language. In order to prove that, we shall prove that given an automaton we can construct 
context free grammar such that \(w \in N(P) \implies w \in L (G)\), and that given a
context free grammar we can construct an automaton such that \( w \in L (G) \implies w \in N (P)\).

\subsection{From grammar to automaton}
The  
%
\marginpar{\(x A\alpha\) \\ \bigskip empty \(x\), then \(\epsilon\)-transit \(A\)}
%
idea to prove is that since a deduction in a grammar is \(x A\alpha \underset{l m}{\Rightarrow} x\beta\alpha\), with \( A \to\beta\) being given, where 
\(x\in T^{*}\), \(\alpha \in (V \cup T) ^{*}\), we use \(\epsilon\)-transition to deal with \(A \to\beta\). Before dealing with \(A\), we pop all the terminates in the stack, that is if \(x A\alpha\) is in the 
stack, what we receive is the corresponding terminate, and we pop the terminate to get \( A\alpha\) in the stack, that is to make \(A\) at the top of the stack. We have something like: 
\[
	\delta (q , a , a) = (q ,\epsilon)
\] 
so that we pop the terminate. Moreover, we use a \(\epsilon\)-transition to achieve \(A \to\beta\), that is \[\delta (q , A,\epsilon) = (q,\beta)\]which we complete that \( x A\alpha \Rightarrow x\beta\alpha\). You may have noticed that there can be only one state. 

Now \(\beta\alpha\) is in the stack. If \(\beta\alpha  = y B\gamma\), then we do the same procedure to make \(B\gamma\) in the stack and make another generation to produce \( \varphi\gamma\) if \(B \to \varphi\) is given, until there is no variable and therefore no terminate in the stack (cause we keep inputing terminates to pop them out). 

Consequently, we have that \(\alpha \in T ^{*} (\alpha \in L(G) \implies\alpha \in N (P))\).
And therefore we construct an automaton from a grammar. Note that the automaton has only one state and that it is empty-stack accepting. 

\begin{exam}
	Given a grammar whose production is \(S \to a  A  A \), \(A \to a S \mid b S \mid a\), construct an automaton.
\end{exam}
While it is a theorem to prove that \(G\) and \(P\) here are equivalent, we have no room nor time for it. 
% subsection:From grammar to automaton 


\subsection{From automaton to grammar}
The %
\marginpar{\( S \to [q_{0} Z_{0}p]\) then }%
%
procedure of transforming an automaton to grammar is concerning with \textbf{Greibach Normal Form}. For example, given an automaton \(P\),
let \( (r_{n}, Y_1 Y_2 \dots Y_n)\) be contained in \(\delta ( q, a , X)\)\footnote{We use \( [ p  X q]\) to denote that we are at \(p\) and after \(X\) is gone from the 
stack, the state goes to \(q\). View \( [p X q]\) as variable.}, that is to say if we are at \(q\), we receive \(a\), \(X\) is at the top
of the stack, then we go to state \(r_{n}\) and \(X\) is popped, \(Y_1 Y_2 \dots\) is pushed. 

What will happen if we receive \(a\) at state \(q\), viewing the automaton as a grammar? We actually have 
\[ [q Xr_{n} ] \to a [q Y_1 r_{1}] [r_1 Y_2 r_2] \dots [r_{n-1} Y_{n} r_{n}]\] 
which is indeed a recursive procedure. If the latter variables are all terminative\footnote{It could be this word}, then we will produce a 
string that is accepted by the automaton. 

Note that \(r_{i}\) is arbitrarily chosen and it may produce many \textbf{useless} variables consequently, and that the grammar is in \textbf{Greibach Normal Form}, where every production is like 
\[
A \to a B_1 B_2 \dots B_n
\]
Indeed, the procedure looks like some kind of algorithm in computer, while it actually is. Anyway, this is the main idea 
of the transformation. 

Let the original automaton be \(P\), and let the grammar constructed here be \(G\). It is true that we should check \marginpar{From bottom to top to check the variables}
\(L (P) = N (G)\). Since that  we have poor time here, let us just skip it.

\begin{exam}
	Transfer the automaton to check (if) and (else) into grammar. \(P_{N} = ( \{ q \} , \{ i, e\}, \{Z\},\delta _{N}, q ,Z)\). 
	\(\delta _{N}  ( q , i , Z) = (q , ZZ)\), \(\delta _{N} (q , e , Z)= ( q ,\epsilon)\). 

	The production of the grammar \(G\) are as follows: 
	
	\medskip
	\setlength{\hangindent}{33pt}
	1. The only production for \(S\) is \(S \to [qZq]\), for there is only one state. If there are \(n\) states, then there will be \(n\)
	productions like this one. 

	\smallskip
	\setlength{\hangindent}{33pt}
	2. From \(\delta _{N}(q , i ,Z) = (q, ZZ)\), we know that \([q Zq] \to i [qZq][qZq]\). However, if there are \(n\) states, there will be \(n^{2}\) productions in this type. It should looks like \( [q Z q] \to i [q Z q_1][q_2 Zq]\). The middle two states can be arbitrary. So there should be \(n ^{2}\) in this type. Similiarly, if \((q, ZZZ)\) is contained, there will be \(n ^{4}\) productions out of \( (q, ZZZ)\)

	\smallskip
	\setlength{\hangindent}{33pt}
	3. \(\delta _{N} (q ,e , Z) = (q,\epsilon)\), so 
	\[
		[qZ q] \to e
	\]

	\setlength{\hangindent}{0pt}
	We can use \(A\) instead of \( [ qZ q]\). So the production are
	\[
	\begin{aligned}
		& S \to A \\ 
		&A \to i  A  A \mid e
	\end{aligned}
	\]
	And it is easy to see that \(A \) is equivalent to \(S\). Then we can substitude \(A\) with \(S\). Then we got: 
	\[
	\begin{aligned}
		S \to i SS \mid e
	\end{aligned}
	\]
\end{exam}
% subsection:From automaton to grammar 
% section Pushdown automata and Context Free Grammar 
\end{document}
