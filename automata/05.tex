\documentclass[12pt]{ctexart}
\usepackage{amsmath}
\usepackage{amsthm}
\usepackage{amssymb}
\usepackage{amsfonts}
% \usepackage{geometry}
\usepackage{graphicx}
\usepackage{bookmark}
\usepackage{tikz-cd}
% \usepackage{hyperref}

\theoremstyle{definition}
\newtheorem{definition}{定义}[section]
\newtheorem{lemma}[definition]{Lemma}
\newtheorem{thm}[definition]{定理}
\newtheorem{proposition}[definition]{性质}

\theoremstyle{plain} 
\newtheorem{exam}[definition]{Example}
\theoremstyle{remark}
\newtheorem{remark}[definition]{Remark}
\pagestyle{plain}

\begin{document}

\section{两种范式}
\label{sec:两种范式}
范式有两种范式, 一种是乔姆斯基范式 (CNF), 另一种是格雷巴赫范式 (GNF). 

\begin{definition}[CNF]
\label{def:CNF}
每一个不包含 \(\epsilon\) 的 \(G\) 的产生式都能够写为下面的形式: 
\[
	\begin{aligned}
	A  \to BC  \quad A \to a 
	\end{aligned}
\]
其中 \(A,  B , C \) 是变量, 然后 \(a\) 是终结符. 
\end{definition}
% theorem CNF 

\begin{proof}
能够知道这个定理实际上是显然的. 对于 \(G\) 的一个产生式, 其写为 
\[
A \to X_1 X_2 X_3 \dots X_m 
\]
	其中 \(X _{i} \in V \cup\). 这个时候, 我们将其中的终结符进行这样的处理, \(X_{i}\) 换为 \(V_{i}\), \(V_{i}\) 是新引入的变量. 
	并且引入产生式 \(V_{i} \to X _{i}\). 这就是将一个表达式 \(A \to X_1 \dots X_{m}\) 之中的终结符处理掉了. 
	
	我们将表达式 \(A \to X _{i}\dots\) 写为 \(A \to V_1 \dots V_{m}\), \(V_{i}\) 是变量. 我们再次引入变量, 将 \(A \to V_1\dots V_{m}\) , 拆解为多个产生式, 就是 \(A \to V_1 B_1\), \(B_1 \to V_2 B_2\)\dots 这样我们就能够将这个产生式拆分为多个满足乔姆斯基范式的形式的产生式
\end{proof}

\begin{exam}
给定 CFG\footnote{CFG stands for context free grammar. 只是有点忘了} \(G\), \(P\) 为 
\[
S \to b A \mid a B , \quad A \to b A  A \mid a S \mid a, \quad B \to a BB \mid bS \mid b 
\]
流程为
\begin{enumerate}
\item 对于产生式之中有终结符的, 进行变量的替换, 比如说 \(A \to b A\) 我们引入新的变量 \(C_{b}\), 将这个式子换为 
	\[
	A \to C_{b} A , \quad  C_{b} \to b
	\]
\item 对于形如 \(A\to  B_1 B_2 \dots B _{m}\) 的式子, 引入中间变量. 
\end{enumerate}
\end{exam}

\begin{exam}
	将 \(G\) 的产生式进行处理, 使得其为乔姆斯基范式. 
	\[
	\begin{aligned}
	S \to A S \mid B A B C \quad A \to A 1 \mid 0 A 1 \mid 01 \quad B \to 0 B \mid 0 \quad C \to 1 C \mid 1
	\end{aligned}
	\]
\end{exam}


\begin{definition}[格雷巴赫范式]\label{def:格雷巴赫范式}
	每个产生式的形式为 \(A \to a\alpha\) 其中 \(a\) 是一个终结符, 然后 \(\alpha\) 是变元的串, 也就是 \(\alpha \in  V ^{*}\). 简写为GNF. 
\end{definition}
% definition: 格雷巴赫范式 
\begin{exam}
将 \(S \to A B  , A \to Aa A \mid bB \mid b , B \to b\) 转化为 GNF.

虽然说, CNF 的转化是不要求掌握的. 
\end{exam}
就尼玛讲完了! 
% section 两种范式 

\section{下推自动机}
\label{sec:下推自动机}

\noindent Contents:
\begin{enumerate}
	\item 接受的语言
	\item 其和文法的等价性
	\item 确定型下推自动机. 
\end{enumerate}

\subsection{下推自动机的定义}
\begin{definition}[下推自动机]\label{def:下推自动机}
	下推自动机 (PDA)\footnote{PDA for push down automata}. 一个 PDA \(G\) 是一个七元组. 
	PDA 相当于一个 NFA 加上一个栈. 进行每一次状态转移的时候, 会将栈顶元素弹出, 然后压入某些字符串. 
	初始阶段, 栈里面有一个初始元素 \(Z_{0}\). 比如说我们接受了 \(0\), 而后弹出 \(Z_{0}\), 压入 \(0Z_{0}\). 对于一个字符串, 压入顺序是\textbf{从后往前的}. 
\end{definition}
% definition: 下推自动机 

\begin{exam}
	设计识别 \(L _{01} = \{ 0 ^{n} 1 ^{n} \mid n \ge 1 \} \) 的PDA
\end{exam}

\begin{exam}
	设计识别 \(L _{w w ^{R}}= \{ w w ^{R} \mid w \in ( 0 + 1) ^{*} \}\)
\end{exam}
% subsection:下推自动机的定义 

% section 下推自动机 
\end{document}
