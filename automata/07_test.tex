\section{The Normal Form of grammar}
\label{sec:The Normal Form of grammar}

There are two kinds of normal forms that we should know, which are \textbf{Chomsky Normal Form} and \textbf{Greibach Normal Form}. 
The definitions have been introduced in previous chapter. You should check them out.
% section The Normal Form of grammar 

\section{The simplification of grammar}
\label{sec:The simplification of grammar}

This section is on its place. However, it has been learned in previous chapter. You may 
check previous chapter for more info.
% section The simplification of grammar 
\section{The Pump Lemma of grammar}
\label{sec:The Pump Lemma of grammar}

\subsection{The content of Pump Lemma}
\begin{thm}[Pump Lemma]
\label{Pump Lemma}
Let \(L \) be the language of a grammar. Then there exist \(n\) that if \(|z| \ge n\), then \(z = uvwxy\), suit the following 
conditions: 

	\bigskip
	\setlength{\hangindent}{33pt}
	1. \(| v w x | \le n\). That is to say \( |v  w x |  \) can't be too long.

	\smallskip
	2. \(v x \ne\epsilon\). Since \(v,  x \) are the pieces to be pumped, \(v\) or \(x\) should not be zero, that is to 
	say, \(v\) and \(x\) can be both empty, \emph{and} that is to say \( v x \ne\epsilon\).

	\smallskip 
	3. For all \(i \ge 0\), \(u v ^{i} w x ^{i} y\) is in \(L\).
\end{thm}
% theorem Pump Lemma 
% subsection:The content of Pump Lemma  

\subsection{The application of Pump Lemma}
Similarly, the Pump Lemma is used to prove a language \(L\) is not a context free language. Let us restate the lemma using the 
mathematical logic. If \(L\) is a context free language then 
\[
\exists n \in N \ \forall\alpha \in L (\forall uvwxy =\alpha (|v w x | \le n , vx \ne\epsilon \to  u v ^{i} w x ^{i} y \in L))
\]
Let the proposition above be \(A\), then we have \(L \text{ is CFL} \to A\). Thus, we have \(\neg A \to L \text{ is not CFL}\). 
And \(\neg A\) equates 
\[
\forall n \in N ,\exists\alpha \in L ,\exists u v w x y = \alpha (\neg  (|vwx| \le n , vx \ne \epsilon \to u v ^{i} w x ^{i} y \in L))
\]
So we have the procedure here, similar to the one in previous chapter about the formal expressions, that we check for every \(n\) in \(\mathbb{N}\), considering as a random variable\footnote{not that variable}, and find a \(\alpha \in L\), and prove that for all kinds of \( uvw xy\), there exists \(i\) s.t. \( u v ^{i} w x ^{i}y\) is not in \(L\), where we often discuss about the different conditions. 

\begin{exam}
	Use pump lemma to show that \(L = \{ 0 ^{n} 1 ^{n} 2 ^{n} \mid n \ge 1 \}\) is not context free language.
\end{exam}
There is another example to show that the grammar can't describe the string that have two pairs of equal numbers of symbols. 
\begin{exam}
	Let \(L = \{ 0 ^{i} 1 ^{j} 2 ^{i} 3 ^{j} \}\). Use pump lemma to prove that \(L\) is not context free language. 
\end{exam}
Mover, since pushdown automata are equivalent to context free grammar, you can easily see that (not prove that) 
\(L =  \{ ww\}\) is not context  free language. 
\begin{exam}
	Let \(L = \{ ww \}\). Use pump lemma to prove that \(L\) is not context free language.

	Given a \(n\), we shall prove that if \(z \in L\), and \(z  =  uv w xy\), we have that \( u w y\) does not in \(L\) which leads to
	contradiction.  We shall let \(z =  0 ^{n} 1 ^{n} 0 ^{n} 1 ^{n}\). 

	\setlength{\hangindent}{33pt} 
	\medskip
	1. Let us talk about \( v w x \) first. Since \( |v w x | \le n\), we assume that \(v w x \) is all in the first 
	block of \(0\)'s. Let \( |v x |\) be \(k\). Then, \( |  u w y | = 4n - k\) and moreover, since \(v w x\) is all 
	in the first block, \( u w y \) starts with \( 0 ^{n - k} 1 ^{n}\) for sure. If \( u w  y = tt\) for some \(t\),
	then \( |t| = 2n - k/2 \ge 2n - k\), viz., the length of \(t\) is longer than that of \( 0 ^{n - k} 1 ^{n}\), and 
	thus \(t\) should end with \(0\). However, \( uwy \) ends with \(1\). If \( u wy\) is \(tt\) then it should have ended 
	with \(0\) since \(t\) end with \(0\), which, leads to a contradiction.
	\medskip

	2. Suppose that \( v w x\) straddles the first block of \(0\)'s and the first block of \(1\)'s.	
	The same, we assume that \( u w y\) can written as \(tt\). Since \(k\), which is the length of 
	\(v x\), is no bigger than \(n\)\footnote{That is because \(|v w x| \le n\)}, we have \( |uwy| \ge 3n\).
	Thus \( |t | \le 3n / 2\). There are two possiblily: 1. \( v x \) contains no \(1\); 2, \(v x\) contains 
	at least one \(1\). For situation 1., the discussion in (1) is also applied. For situation 2., We assert 
	that \(t\) does not end with \(1 ^{n}\), for there is at least one \(1\) in \(v x\) and it is deleted from 
	\(z\), while \( uw y\) ends with \(1^{n}\), which is for sure.

	\medskip 
	3. The further discussion is omitted here.  You can check page 285 of the textbook for help.
\end{exam}
% subsection:The application of Pump Lemma 
% section The Pump Lemma of grammar 
