\documentclass[../main]{subfiles}
\begin{document}
We 
\marginpar{Make an abstract\\ about intersection and homomorphism}
first make an abstract about the property of the 
context free language. There is some difference between 
context free language and formal language. It has been 
proved that context free language is not closed under intersection 
and difference. However, the intersection between a context free 
language and a regular language is always a context free 
language. Indeed a very intriguing property. 

\section{Substitution}
\label{sec:Substitution}
A 
\marginpar{Who be substituted symbols in a CFL to strings in CFL is still CFL}
subsitution is a function. It is a function that map the symbol 
within a string of a language to another \textbf{language}.
That is to say, \(s \colon a \mapsto  s (a) \), where \(s (a) \) is a language. 
So it is to say, a \(s\) is to make a string in \(L\) bigger, by subsititute the 
symbols with a language.
\begin{definition}[substitutions]\label{def:substitutions}
A substitution \(s\) is a function: 
\[
	s \colon a \mapsto s (a) 
\]
	where \(s(a)\) is a language.
Moreover, the scope of \(s\) can be expanded to \(L\). Let us say \(w = a_1 a_2 \dots a_{n}\)
\[	
	s (w ) =  s (a_1) s(a_2) \dots s (a_n)
\] 
\end{definition}
% definition: substitutions 

\begin{thm}[Closure property under substitution]
\label{Closure property under substitution}
	Given a context free language \(L\), and given an \(s\), if \(s (a)\) is context free language for all \(a\) in \( \Sigma\),
	then \(s (L)\) is also a context free language. 
\end{thm}
% theorem Closure property under substitution 
\begin{proof}
	We proof by construction. Given an \(s\) and \(G\) we can construct a \(G'\) such that 
	\(L (G') = s (L)\). First let say that the grammar of relative symbol has no common variables viz., 
	\(G _{a}\) of \( s (a)\) has independent variables. 
	
	Now, since \(a\) is a symbol in \(G\), and meanwhile, it is a terminate in \(G\), we subsititute the \(a\) with the 
	initial variable \(S_{a}\) in all the production in \(G\). Think about the production that produce \(w = a_1 a_2 \dots a_{n}\). Now it produce 
	\(w = S_{a_1} S_{a_{2}} \dots S_{a _{n}}\). Further more, the initial variable \(S_{a _{i}}\) can produce a 
	string in \(L _{a _{i}} =  s ( a _{i})\). That is to say after production of \(G _{i}\), we have 
	\( w ' =   w_1 w_2 \dots w_{n}\), where \(w _{i} \in s (a _{i})\). \(w ' \in s (L)\) while the production of 
	\(w '\) is indeed a production in context free grammar. 

	So the production of \(G '\) is the union of \(G _{a_{i}}\) and the original production of \(G\) with terminate being 
	subsituted by the corresponding initial variable. And the variable is the union of all the variables. 

	Let us look at parse tree. Consider the parse tree that produce \(w\), where the leaf nodes are symbols in \( \Sigma\). We treat 
	the leaf nodes as the roots of other parse trees. We glue the subtree that produce \textbf{a} string \(G_{a_{i}}\) to the leaf node 
	which is exactly \(a _{i}\). After doing such gluing, we make a parse tree that produce a \(w \in s (L)\). 

	It is indeed true that we should check that \( s (L)= L (G)\). But emm\dots
\end{proof}

\subsection{The application of substitution theorem}

\begin{thm}[The closure properties of context free language]
\label{The closure properties of context free language}
The context free languages are close under following operations: 

\bigskip
1. Union

\smallskip
2. Concatenation

\smallskip
3. Closure and positive closure

\smallskip 
4. Homomorphism
\bigskip 

	where concatenation between two languages is that \(L_1 L 2 = \{ \, w_1 w_2 \mid w_1 \in L_1 , w_2 \in L_2 \,\}\) and a 
	homomorphism is function from \( L\) to \(L_2\), which suit that \( h ( w_1 w_2 ) = h (w_1) h(w_2)\).  
\end{thm}
% theorem The closure properties of context free language 
% subsection:The application of substitution theorem 

\subsection{The reversal}
Context free grammar is closed under reversal, that is to say, if \(L\) is context free grammar, then
\( L ^{R} = \{\, w^{R} \mid w \in L \,\}\) is context free grammar.

\medskip 
the proof of this property is easy. We shall construct a reverse version of the given grammar \(G\).
% subsection:The reversal 

\subsection{Intersection with a regular language}
Not closed under intersection but close use intersection with 
regular language.

exam of \(\{ \, 0 ^{n} 1 ^{n} 2 ^{n} \,\}\) shows not close.

\begin{exam}
	We already know that \( L  = \{ \,  0  ^{n} 1 ^{n} 2 ^{n} \,\}\) is not a context free language, for that it does not suit the 
	pump theorem. However, \( L_1 = \{ \, 0 ^{i}  1 ^{n} 2 ^{n} \,\}\)  , \( L_2 = \{ \, 0 ^{n} 1 ^{n} 2^{i} \, \}\) are context
	free languages, \textbf{and} \(L  = L_1 \cap L_2\). Consequently, we show that context free language is not closed under intersection.
\end{exam}
%
It is true that for two context free languages \(L_1 , L_2\), \(L_1 \cap L_2\) could be a language that is not context free language. 
But we have a theorem to prove that if \(L\) is context free, and \(R\) is regular, then \(L \cap R\) is context free. The proof of the 
theorem provide a insight into what the nature of intersection is. 
\begin{thm}[Closure property with regular language]
\label{Closure property with regular language}
If \(L\) is a context free language, and \(R\) is a regular language, then \(L \cap R\) is a context free language.
\end{thm}
% theorem Closure property with regular language 
\begin{proof}
The idea is to construct a pushdown automata to accept the language \( L \cap R\). The construction is 
to parrellel a finite state automaton and a pushdown automaton. 

Let us say that the finite automaton accept language \(R\) is 
\[
	A  = ( Q _{A}, \Sigma ,\delta _{A} , q _{A},  F_{A})
\]
Let us remind what a finite automaton is. \(Q _{A}\) is the states. \( \Sigma\) is the alphabet, \(\delta _{A}\) is the transition function, 
\(q _{A}\) is the initial state, and \(F _{A}\) is the accepting state. And let us say that the pushdown automaton is 
\[
	P = ( Q _{P} , \Sigma , \Gamma ,\delta_{P}, q _{P}, Z _{0}, F _{P})
\]
where \(Q _{P}\) is the family of the states of pushdown 
automaton, \( \Sigma\) is the alphabet, \( \Gamma\) is the alphabet
in the stack, \(\delta _{P}\) is the transiton function, \(q _{P}\)
is the initial state of \(P\), \(Z _{0}\) is the initial character 
in the stack, \( F_{P}\) is the accepting state.

We construct a new pushdown automaton. 
\[
	( \, Q _{A} \times Q _{P}, \Sigma, \Gamma,\delta ' , ( q _{A},  q _{P}) Z_{0} , F' \, )
\]
	if \(\delta _{A} ( q_1 , a) =  p_1\), \(\delta _{P} (q_2 , a , X\kern 1pt) = (p_2 ,\gamma)\), then 
	\[
		\delta ' (  \{ \, q_1 , q_2 \, \} , a ,X ) = 
		( \{ \, p_1 , p_2  \, \} ,\gamma)
	\] 
	And \( \{ \,  p , q \, \} \in F'\) if and only if 
	\(p\), \(q\) are seperately the accepting state in \(A, P\).
	We can then prove that \(w \in L '\)  if and only if 
	\( w \in L\) and \(w \in R\).
\end{proof}

\begin{exam}
	We already know that \(L  =  \{   i ^{n}  e ^{n}  \}\) is context free language, and that \( R  = \{  i ^{*}  e ^{*}\}\) is 
	regular language. 

	Consequently, \( L \cap R  \) is context free language. We can construct the pushdown automaton that accepting \(L \cap R\). 
\end{exam}

\begin{thm}[Other closure properties]
\label{Other closure properties}
Let \(L \) be a context free language, \(R\) be a regular language. 

\medskip
	\setlength{\hangindent}{33pt}
	1. \( L - R\) is context free language. 
	\smallskip

	2. \( \bar L \) could be a language other than context free language. 
	\smallskip

	3. \(L_1 - L_2\) could be a language other than context free 
language. 
\end{thm}
% theorem Other closure properties 

\begin{proof}
1. \(L - R = L \cap \bar R\)
\smallskip

	2. Use contradiction. If \(\bar L \) is always a context free language. 
	Consider \( L_1 \cap  L_2 = \overline{\bar L_1 \cup \bar L_2}\). Since \(L_1 \cap L_2\) may be not a context free language, 
	then either \(\bar L_1\) is not context free language, or 
	\(\bar L_1 \cup \bar L_2\) is context free language while 
	\( \overline {\bar L_1 \cap \bar L_2}\) is not context free 
	language. 
	Consequently, the statement is not true.
\end{proof}
% subsection:Intersection with a regular language 

% section Subsitution 
\end{document}
