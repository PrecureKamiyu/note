\documentclass[12pt]{ctexart}
\usepackage{amsmath}
\usepackage{amsthm}
\usepackage{amssymb}
\usepackage{amsfonts}
\usepackage{graphicx}
\usepackage{bookmark}
% \usepackage{hyperref}
\usepackage{tikz-cd}

\theoremstyle{definition}
\newtheorem{theorem}{Def}[section]
\newtheorem{lemma}[theorem]{Lemma}
\theoremstyle{definition}
\newtheorem{definition}{定义}[section]
\newtheorem{thm}[definition]{定理}
\newtheorem{proposition}[definition]{性质}

\theoremstyle{plain} 
\newtheorem{exam}[definition]{Example}
\theoremstyle{remark}
\newtheorem{remark}[definition]{Remark}

\begin{document}
\section*{P5}\label{sec:p5}
\paragraph{1.} Prove \(A \cup (B \cap C) = (A \cap B) \cup (A \cap C)\)
\begin{align}
	& \forall x \in A \cup( B\cap C) \\
	\iff & \forall x (x \in A \wedge (x \in B \vee x \in C) \\
	\iff & \forall x ( (x \in A \wedge x \in B ) \vee (x \in A \wedge x \in C) ) \\ 
	\iff & \forall x (x \in (A \cap B ) \cup (A \cap C) ) 
\end{align}

\paragraph{3.} Prove \(| A \cup B | + | A \cap B| = |A| + |B|\)
\begin{proof}
由基数的定义出发: \(X \cap Y = \varnothing\) 的时候, 有 
\begin{equation}
|X \cup Y | = |X | + |Y | 
\end{equation}
成立. 
对于 \(A , B\), 有 
\begin{equation}
	|A \cup B | = | A - B | + | B  -A | + | A \cap B |
\end{equation}
成立. 并且 \( |A| = |A - B | + |A \cap B|\) 也成立. 就有 
\begin{align}
	|A \cup B  | + |A \cap B | & =  |A - B | + |A \cap B | + | B  - A| + | A \cap B | \\
	& = |A| + |B| 
\end{align}
成立.
\end{proof}

\section*{P11}\label{sec:p11}
\paragraph{5. } 
\begin{proof}
\begin{align}
	f \colon \mathbb{Z} \to \mathbb{Z} , x \mapsto x + 1 \\
	g \colon \mathbb{Z} \to \mathbb{Z}, x \mapsto x -1
\end{align}
\(f,g\) 是两个双射.
\end{proof}

\section*{P15}\label{sec:p15}
\paragraph{2. } 
\begin{proof}
	设 \(F^{n\times n}\) 表示数域 \(F\) 上的 \(n\) 阶方阵的全体. \(A , B \in F ^{n \times n}\)
	\begin{align}
		& f\colon F^{n \times n} \times F ^{n \times n} , (A, B) \mapsto f(A,B) = A + B + I  \\
		& g \colon F^{n \times n} \times F ^{n \times n} , (A, B) \mapsto I
	\end{align}
\end{proof}
\paragraph{3. } 
\begin{proof}
	\begin{align}
		|T (M) | = |A| ^{|A|} = 27 \\
		|S (M) | = |A|! = 6
	\end{align}
	图~\ref{tab:m}是置换的乘法表. 
\end{proof}

\begin{figure}
	\centering
\begin{tabular}{|l|l|l|l|l|l|l|} \hline
$\circ$ & 	    $\varphi_1$ & \(\varphi _{2}\) & \(\varphi _{3}\) & \(\varphi _{4}\) & \(\varphi _{5}\) & \(\varphi _{6}\) \\ \hline 
 \(\varphi _{1}\)   & \(\varphi _{1}\) & \(\varphi _{2}\) & \(\varphi _{3}\) & \(\varphi _{4}\) & \(\varphi _{5}\) & \(\varphi _{6}\) \\ \hline
 \(\varphi _{2}\)   &\(\varphi _{2}\)  &\(\varphi _{1}\)  &\(\varphi _{5}\)  &\(\varphi _{6}\)  &\(\varphi _{3}\)  &\(\varphi _{4}\)  \\ \hline 
 \(\varphi _{3}\)   &\(\varphi _{3}\)  &\(\varphi _{4}\)  &\(\varphi _{1}\)  &\(\varphi _{2}\)  &\(\varphi _{6}\)  &\(\varphi _{5}\)  \\\hline
 \(\varphi _{4}\)   &\(\varphi _{4}\)  &\(\varphi _{3}\)  &\(\varphi _{6}\)  &\(\varphi _{5}\)  &\(\varphi _{1}\)  &\(\varphi _{2}\)  \\\hline
 \(\varphi _{5}\)   &\(\varphi _{5}\)  &\(\varphi _{6}\)  &\(\varphi _{2}\)  &\(\varphi _{1}\)  &\(\varphi _{4}\)  &\(\varphi _{3}\)  \\\hline
 \(\varphi _{6}\)   &\(\varphi _{6}\)  &\(\varphi _{5}\)  &\(\varphi _{4}\)  &\(\varphi _{3}\)  &\(\varphi _{2}\)  &\(\varphi _{1}\) \\
\hline
\end{tabular}
	\caption{ \(M\) 的置换乘法表}\label{tab:m}
\end{figure}

\section*{P19}\label{sec:p19}
\paragraph{2.1.}  
\begin{proof}
不满足结合律. 
\begin{equation}
	1 \circ (2 \circ 3)  = 1 ^{2} + (2 ^{2} + 3^{2}) ^{2} = 170 \ne 34 = 3 ^{2} + (1 ^{2} + 2^{2}) = (1 \circ 2 ) \circ 3
\end{equation}
满足交换律
\begin{equation}
a \circ b = a ^{2} + b ^{2} = b ^{2} + a ^{2} = b \circ a
\end{equation}
\(a^{2} + b^{2} = b^{2} + a^{2}\) 成立是因为加法满足交换律.
\end{proof}

\paragraph{2.2} 
\begin{proof}
满足结合律: 
\[
\begin{aligned}
	(a \circ b ) \circ c & = (a + b - ab ) +c - (a+b -ab ) c \\
	& = a + b + c - ab - ac - bc + abc
\end{aligned}
\]
同时, 
\[
\begin{aligned}
	a \circ (b \circ c) & = a + (b + c - bc ) - a (b +c - bc) \\ 
	& = a + b  +c - bc - ab - ac + abc 
\end{aligned}
\]
所以
\begin{equation}
(a \circ b ) \circ c = a \circ (b \circ c)
\end{equation}
成立.

满足交换律, 因为 
\begin{equation}
a \circ b = a + b - ab = b + a - ba = b \circ a
\end{equation} 
中间的等号成立是因为加法和乘法满足交换律. 
\end{proof}

\section*{P23}\label{sec:p23}
\paragraph{1.1.}
是自同态, 因为
\begin{equation}
	f (a b ) = |ab | = |a| |b | = f (a ) f (b) 
\end{equation}
\(f	\) 不是满的, 因为实数域 \(\mathbb{R}\) 上, 不存在数使得其绝对值为负数. 

\paragraph{1.2} 
不是自同态. 
\begin{equation}
	f (ab ) = 2 a b \ne 2 a \times 2b  = f(a) f(b) 
\end{equation}

\paragraph{1.3} 
是自同态. 
\begin{equation}
	f (a b) = (a b ) ^{2} = a ^{2} b ^{2} = f (a) f (b) 
\end{equation}
不是满的, 因为 \(\mathbb{R}\) 上不存在平方为负数的数. 

\paragraph{1.4} 
不是自同态. 
\begin{equation}
	f (ab ) = - ab \ne (-a) (-b) = f(a) f(b) 
\end{equation}
\end{document}
