\documentclass[../main.tex]{subfiles}
\begin{document}
\chapter{TODO}
\section{cyclic group}
\label{sec:cyclic group}


What is cyclic group? 

\(S \subseteq G\) if \( \langle S \rangle = G\) then \(G\) can be generated by \(S\).  Maybe we can use \(M\) instead of \(S\).

If \(S\) is a singleton \(\{ g\}\)then \(G\) is a cyclic group and can be denoted as \( \langle g \rangle\).

If \(G\) is infinite and \(G = \langle a \rangle \), then \(G\) is isomorphic to \(\mathbb{Z}\). 

If \(G\) is finite and \(G = \langle a \rangle\), then \(G\) is isomorphic to \(\mathbb{Z} _{n}\)

A group \(G\) is cyclic then \(G\) is abelian. 

If a finite group \(G\) has a group member whose order is \(|G|\), then \(G\) is cyclic.

\vskip .5cm
\hrule
\vskip .5cm
A Euler function \( \varphi (n)\) is defined as the number of the positive integers that are less than \(n\) and are relative prime to \(n\)

\(G\) is finite, and \(|G| = n\), then \(G\) has \( \varphi (n)\) generator. 
\medskip

the proof is easy, given a  number \( q\) , \(q \) is relative prime to \(n\), then \(\exists  u , v \in \mathbb{Z}, uq + v |G| = 1\)

\(G\) is a finite cyclic group, given a positive factor \(k\), there is one and only one subgroup that is \( \langle a ^{n / k} \rangle\).
% section cyclic group 

\section{Transformation groups}
\label{sec:Transformation groups}

\(M\) is a set. The transformation over a \(M\) is a monoid.

\(S (M)\) is the set of the bijective transformation on \(M\). \(S (M)\) is a group and is called symmetric group.

If \(|M| = n\), then \(S (M)\) is sometimes denoted as \( S_{n}\). 

And \(|S _{n}|  = n !\).  

Transformation group \(G\) on \(M\), 

\begin{thm}
	Transformation group \(G\) on \(M\), if there is onto function or one-one function in \(G\) then \(G = S(M)\)
\end{thm}
% theorem  

\begin{proof}
	If \(\tau \) is onto, consider \( \epsilon \tau(a) =  \tau (a)\) then \(\epsilon\) is \( \text{id}\). 

	If \( \tau\) is one-one, consider \(\tau\epsilon (a) =\tau (a)\), then \(\epsilon\) is \( \text{id}\), because \(\epsilon (a) = a\) forall \(a\) in \(G\). 

	\(\forall a \in G\), \( \exists b \in G , b = a ^{-1}\), viz., \( a b =\epsilon\). So \(a\), \(b\) are inversible and thus are
	bijectives. Then every \(a\) in \(G\) is bijective then 
	\(G  = S (M)\). 
\end{proof} 

If a bijective is in a group on \(M\) then the group is 
bijective group, that is a group whose elements are 
bijectives. 

If a group is not a bijective group then the group 
has no bijective or subjective or injective. 

\begin{exam}
	\(M = \{ \, \ip xy \mid x, y \in \mathbb{R} \, \}\), \(\forall a \in \mathbb{R}\):  
	\[
		\tau_{a} \colon M \to M , \ip xy \mapsto (x + a, 0)
	\]
	Prove that \(G = \{ \,\tau _{a} \mid a \in \mathbb{R} \, \}\) is a group. 
\end{exam}

\begin{thm}[Cayley]
\label{Cayley}
Given a \(G\), there exists a bijective group that is isomorphic to \(G\). 
\end{thm}
% theorem Cayley 
\begin{proof}
Treat the elements in \(G\) as functions which is absolutely a  
bijective.
\end{proof}
% section Transformation 
\end{document}
