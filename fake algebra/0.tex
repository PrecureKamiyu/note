\documentclass[../main.tex]{subfiles}
\begin{document}
\chapter{Some classic groups}
\section{cyclic group}
\label{sec:cyclic group}

\begin{itemize}
	\item
		What is cyclic group? 

		If \(\exists q \) s.t. \( \langle q \rangle = G\), then \(G\) 
		is a cyclic group. 
	\item What is \( \langle q \rangle\)? 

		It is the smallest group that contains the \(q\). 
		And we can prove that \( \langle q \rangle \) is equal to 
		\[
			\{ \, \dots q ^{-1} , 1 , q  , q^{2} , \dots \, \}
		\]
	\item
		Can the idea of it be generalized? 

		There is also \( \langle S \rangle\) (where 
		\(S\) is a subset of \(G\))
		which is 
		the smallest subgroup that contains the 
		\(S\).
		We can prove that 
		\[
			\langle S \rangle = 
			\{ \,  q _{1} \dots q_{k}\mid k \in \mathbb{Z} , q_{i} \in S \cup S ^{-1}\, \}
		\]
		This thing can be more generalized when we reach the section of free groups.
	\item If \(G\) is equal to \(\langle a \rangle\), then \(a\) is
		called generator of \(G\).
	\item Clearly enough, if \(G\) is cyclic then \(G\) is
		abelian.

	\item If \(G\) is infinite and is equal to \(\langle a \rangle\),
		then \(G\) is isomorphic to \(\mathbb{Z}\).
	\item If \(G\) is finite and is equal to \(\langle a \rangle\),
		then \(G\) is isomorphic to \(\mathbb{Z} _{n}\), where \(n\) is \(\vert G \vert\).
\end{itemize}

\vskip .5cm
\hrule
\vskip .5cm
\vskip .5cm

\begin{itemize}
	\item  What is Euler (totient) function?

		\( \varphi (n)\) is defined as the number of the positive intergets that are less than \(n\) and are relative primes to \(n\).
	\item 
		What is the number of the generators? 

		If \(G\) is finite, and \(\vert G \vert = n\), then \(G\) 
		have \( \varphi (n)\) generators. 

	\item 
		Can you prove it? 

		This trivial that \(k\) is relative prime to \(n\) then \(\exists a , b  \in \mathbb{Z}\) such that 
		\[
			a k + b n  = 1
		\]
		which is to say that if \(q\) is generator, then \( q ^{ ak}\) is equals to \(q\), which is to say that \(\langle q ^{ak} \rangle = G\). 
	\item  
		\(G\) is a finite cyclic group, given a positive factor \(k\), there is one and only one subgroup that is \( \langle a ^{n / k} \rangle\).
	\item Can you tell the number of the subgroups in a cyclic group?

		It is clear that there are subgroups as many as the factors of 
		\(\vert G \vert\)..

	\item
		We can study the Automorphisms of the cyclic groups. We have the conclusion that \( \text{Aut} (C_{n}) \simeq (\mathbb{Z}/ n \mathbb{Z} ) ^{*}\). The latter one is defined as the inversable members under the multiplication in the \(\mathbb{Z}/ n\mathbb{Z}\).


\end{itemize}
% section cyclic group 

\section{Transformation groups}
\label{sec:Transformation groups}

\(M\) is a set. The transformation over a \(M\) is a monoid. \(S (M)\) is the set of the bijective transformation on \(M\). \(S (M)\) is a group and is called \textbf{symmetric} group. \(S (M)\) is sometimes denoted as \( S_{n}\) if \(\vert M \vert = n\). 

\begin{thm}
	For a transformation group \(G\) on \(M\), if there is onto function or one--one function in \(G\) then \(G\) is a permutation group.
\end{thm}
% theorem  

\begin{proof}
	(1) If \(\tau \) is onto, consider \( \epsilon\tau(a) = \tau (a)\), then \(\epsilon\) is \( \text{id}\).

	\smallskip
	(2) If \( \tau\) is one-one, consider \(\tau\epsilon (a) =\tau (a)\), then \(\epsilon\) is \( \text{id}\), because \(\epsilon (a) = a\) forall \(a\) in \(G\).

	\(\forall a \in G\), \( \exists b \in G , b = a ^{-1}\), viz., \( a b =\epsilon\). So \(a\), \(b\) are inversible and thus are
	bijectives. Then every \(a\) in \(G\) is bijective then \(G\) is a permutuation group.
\end{proof}

If \(G\) is the group whose elements are in \(T(M)\). then the fact that \(G\) has a bijection suggests that \(G\) is a permutuation group, that is, all of the elements of \(G\) are bijection; the fact that \(G\) has a non--bijective function suggests that all of the elements in \(G\) are not bijection.

\begin{exam}
	\(M = \{ \, \ip xy \mid x, y \in \mathbb{R} \, \}\), \(\forall a \in \mathbb{R}\):
	\[
		\tau_{a} \colon M \to M , \ip xy \mapsto (x + a, 0)
		\]
		Prove that \(G = \{ \,\tau _{a} \mid a \in \mathbb{R} \, \}\) is a group.
\end{exam}

\begin{thm}[Cayley]\label{Cayley} 
	Given a \(G\), there exists a permutuation group that is isomorphic to \(G\).
\end{thm}
% theorem Cayley 
% section Transformation 

\section{Symmetric group}
\label{sec:Symmetric group}

\begin{itemize}
	\item Something writte as
		\[
			\begin{pmatrix}
				1 & 2 & 3 \\ 
				3 & 1 & 2 
			\end{pmatrix}
		\]
		is called as a \textbf{permutuation}. And also called \textbf{cycles},
	\item The above transformation can be written as \((132)\).
	\item For such thing as \( ( i_{1}  i_{2} i _{3} \dots i _{k})\), the order of it is \(k\).
	\item For the compostion of the transformation, we read from left to right. 
		And if we have \(\sigma\), and \(\tau\), and they are: 
		\[
			\begin{pmatrix}
				1 & 2 & 3\\
				\sigma (1) & \sigma (2) & \sigma (3)
			\end{pmatrix}
			\qquad 
			\begin{pmatrix}
				1 & 2 & 3 \\ 
				\tau (1)  & \tau (2 ) &\tau (3)
			\end{pmatrix}
		\]
		then we have that 
		\(\sigma\tau\) is equals to 
		\[
			\begin{pmatrix}
				1 & 2 & 3 \\ 
				\sigma (\tau 1) &\sigma (\tau 2) &\sigma (\tau 3)
			\end{pmatrix}
		\]
		You may easily know it from the fact that 
		\[
			\begin{pmatrix}
				1 & 2 & 3 \\
				\tau (1) &\tau (2) &\tau (3)
			\end{pmatrix}
			= 
			\begin{pmatrix}
				\sigma (1) &\sigma (2) &\sigma (3) \\ 
				\sigma\tau (1) &\sigma\tau (2) &\sigma\tau (3)
			\end{pmatrix}
		\]
	\item 	
		Moreover we have that 
		\[
			\sigma ^{-1} = 
			\begin{pmatrix}
				\sigma (1) &\sigma (2) &\sigma (3) \\ 
				1 & 2 & 3 
			\end{pmatrix} 
		\]
		because we have that 
		\[
			\begin{pmatrix}
				i_{1} & i_{2} & i_{3} \\ 
				j_1 & j_2 & j_3 
			\end{pmatrix}
			{\large \times}
			\begin{pmatrix}
				j_1 & j_2 & j_3 \\
				k_1 & k_2 & k_3 
			\end{pmatrix} 
			= 
			\begin{pmatrix}
				i_{1} & i_{2} & i_{3} \\ 
				k_1 & k_2 & k_3 
			\end{pmatrix} 
		\]
	\item There is a theorem that easily compute the \(\sigma\tau\sigma ^{-1}\). 
		If \(\tau\) is written as \( (i_{1} i_2 \dots i_{k})\), then we have that
		\[
			\sigma\tau\sigma ^{-1} =  (\sigma (i_{1}) \sigma (i_{2})\dots\sigma (i_{k}))
			\]
			It is given by the fact that 
			\[
				\sigma\tau\sigma ^{-1} =  
				\begin{pmatrix}
					\sigma (1) &\sigma (2) &\sigma (3) \\
					\sigma (i_{1}) &\sigma (i_{2}) & \sigma (i_{3 }) % TODO there is a problem here: what is i_k? 
				\end{pmatrix}
				\]
				which exactly is saying that \(\sigma\tau\sigma ^{-1} = (\sigma (i_{1})\sigma (i_{2})\sigma(i_{3}))\). One can test the propostion. \[\sigma\tau\sigma^{-1} (\sigma(1)) = \sigma\tau (\sigma^{-1}(\sigma (1))) = \sigma\tau (1) = \sigma (i_{1})\]
				\(1\) is randomly chosen. One can test all other numbers.
			\item The compostion of transformation is quite tricky. 
			\item A cycle can be decomposed as the compostion of the \textbf{duihuan}. In general, a \(k\)--cycle: \((i_1\dots i_{k})\) can be decomposed as \((i_{1}i_{k})(i_{1}i_{k-1})\dots (i_{1}i_{2})\).
			\item If there is no identical number in transformation \(\sigma\), \(\tau\), then \(\sigma\) and \(\tau\) are commutable. that is \(\sigma\tau =\tau\sigma\). For a random transformation you can easily find the composition of such form like \(\tau\sigma \varphi \dots \), where the transformations are independent (let us call it temporarily).
			\item The number of the duihuan in the decomposed form of a \(k\)--cycle can be odd or even. If the number is odd then the permutuation is called \textbf{odd}, and if the number is even, then the permutuation is called \textbf{even}. It can be proved that all the \textbf{even} permutuation forms a subgroup of \(S_{n}\), often denoted as \(A_{n}\). And moreover, the zhishu of \(A_{n}\) is \(2\).
			\item One can prove that if there is a odd permutuation in the permutuation group, then the number of the odd permutuations and the number of even permutuations are equal. Such that \textbf{either} the permutuations are all even, \textbf{or} half the permutuations are odd and half of the permutuations are even.
\end{itemize}

\paragraph{Summary} Symmetric groups are very important. Why? When it comes to the very section of `group actions' we will use the properties of symmetric groups. However, most of the properties above is very useless. Why? Because they are.
% section Symmetric group
\end{document}
