\documentclass[../main.tex]{subfiles}
\begin{document}
\chapter{TODO}
\section{cyclic group}
\label{sec:cyclic group}


What is cyclic group? 

\(S \subseteq G\) if \( \langle S \rangle = G\) then \(G\) can be generated by \(S\).  Maybe we can use \(M\) instead of \(S\).

If \(S\) is a singleton \(\{ g\}\)then \(G\) is a cyclic group and can be denoted as \( \langle g \rangle\).

If \(G\) is infinite and \(G = \langle a \rangle \), then \(G\) is isomorphic to \(\mathbb{Z}\). 

If \(G\) is finite and \(G = \langle a \rangle\), then \(G\) is isomorphic to \(\mathbb{Z} _{n}\)

A group \(G\) is cyclic then \(G\) is abelian. 

If a finite group \(G\) has a group member whose order is \(|G|\), then \(G\) is cyclic.

\vskip .5cm
\hrule
\vskip .5cm
A Euler function \( \varphi (n)\) is defined as the number of the positive integers that are less than \(n\) and are relative prime to \(n\)

\(G\) is finite, and \(|G| = n\), then \(G\) has \( \varphi (n)\) generator. 
\medskip

the proof is easy, given a  number \( q\) , \(q \) is relative prime to \(n\), then \(\exists  u , v \in \mathbb{Z}, uq + v |G| = 1\)

\(G\) is a finite cyclic group, given a positive factor \(k\), there is one and only one subgroup that is \( \langle a ^{n / k} \rangle\).
% section cyclic group 

\section{Transformation}
\label{sec:Transformation}

\(M\) is a set. The transformation over a \(M\) is a monoid.

\(S (M)\) is the set of the bijective transformation on \(M\). \(S (M)\) is a group and is called symmetric group.
% section Transformation 
\end{document}
