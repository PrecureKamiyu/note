\documentclass[../main.tex]{subfiles}
\begin{document}
\chapter{TODO}
\section{cyclic group}
\label{sec:cyclic group}



\begin{itemize}
\item 
What is cyclic group? 

If \(\exists q \) s.t. \( \langle q \rangle = G\), then \(G\) 
is a cyclic group. 
\item What is \( \langle q \rangle\)? 

It is the smallest group that contains the \(q\). 
And we can prove that \( \langle q \rangle \) is equal to 
\[
	\{ \, \dots q ^{-1} , 1 , q  , q^{2} , \dots \, \}
\]
\item 
	Can the idea of it be generalized? 

There is also \( \langle S \rangle\) (where 
		\(S\) is a subset of \(G\))
which is 
the smallest subgroup that contains the 
\(S\).
We can prove that 
\[
\langle S \rangle = 
		\{ \, q ^{k} \mid k \in \mathbb{Z} , q \in S\, \}
\]
\item If \(G\) is equal to \(\langle a \rangle\), then \(a\) is 
	called generator of \(G\). 
\item Clearly enough, if \(G\) is cyclic then \(G\) is 
	abelian.

\item If \(G\) is infinite and is equal to \(\langle a \rangle\),
then \(G\) is isomorphic to \(\mathbb{Z}\). 
\item If \(G\) is finite and is equal to \(\langle a \rangle\),
then \(G\) is isomorphic to \(\mathbb{Z} _{n}\), where \(n\) is \(\vert G \vert\) 
\end{itemize}

\vskip .5cm
\hrule
\vskip .5cm
\vskip .5cm

\begin{itemize}
\item  What is Euler function? 

\( \varphi (n)\) is defined as the number of the positive intergets that are less than \(n\) and are relative primes to \(n\).
\item 
What is the number of the generators? 

If \(G\) is finite, and \(\vert G \vert = n\), then \(G\) 
have \( \varphi (n)\) generators. 

\item 
Can you prove it? 

This trivial that \(k\) is relative prime to \(n\) then \(\exists a , b  \in \mathbb{Z}\) such that 
\[
a k + b n  = 1
\]
which is to say that if \(q\) is generator, then \( q ^{ ak}\) is equals to \(q\), which is to say that \(\langle q ^{ak} \rangle = G\). 
\item  
	\(G\) is a finite cyclic group, given a positive factor \(k\), there is one and only one subgroup that is \( \langle a ^{n / k} \rangle\).
\item Can you tell the number of the subgroups in a cyclic group?

It is clear that there are subgroups as many as the factors of 
\(\vert G \vert\)
\end{itemize}
% section cyclic group 

\section{Transformation groups}
\label{sec:Transformation groups}

\(M\) is a set. The transformation over a \(M\) is a monoid.

\(S (M)\) is the set of the bijective transformation on \(M\). \(S (M)\) is a group and is called symmetric group.

If \(|M| = n\), then \(S (M)\) is sometimes denoted as \( S_{n}\). 

And \(|S _{n}|  = n !\).  

Transformation group \(G\) on \(M\), 

\begin{thm}
	Transformation group \(G\) on \(M\), if there is onto function or one-one function in \(G\) then \(G = S(M)\)
\end{thm}
% theorem  

\begin{proof}
	If \(\tau \) is onto, consider \( \epsilon \tau(a) =  \tau (a)\) then \(\epsilon\) is \( \text{id}\). 

	If \( \tau\) is one-one, consider \(\tau\epsilon (a) =\tau (a)\), then \(\epsilon\) is \( \text{id}\), because \(\epsilon (a) = a\) forall \(a\) in \(G\). 

	\(\forall a \in G\), \( \exists b \in G , b = a ^{-1}\), viz., \( a b =\epsilon\). So \(a\), \(b\) are inversible and thus are
	bijectives. Then every \(a\) in \(G\) is bijective then 
	\(G  = S (M)\). 
\end{proof} 

If a bijective is in a group on \(M\) then the group is 
bijective group, that is a group whose elements are 
bijectives. 

If a group is not a bijective group then the group 
has no bijective or subjective or injective. 

\begin{exam}
	\(M = \{ \, \ip xy \mid x, y \in \mathbb{R} \, \}\), \(\forall a \in \mathbb{R}\):  
	\[
		\tau_{a} \colon M \to M , \ip xy \mapsto (x + a, 0)
	\]
	Prove that \(G = \{ \,\tau _{a} \mid a \in \mathbb{R} \, \}\) is a group. 
\end{exam}

\begin{thm}[Cayley]
\label{Cayley}
Given a \(G\), there exists a bijective group that is isomorphic to \(G\). 
\end{thm}
% theorem Cayley 
\begin{proof}
Treat the elements in \(G\) as functions which is absolutely a  
bijective.
\end{proof}
% section Transformation 

\section{Symmetric group}
\label{sec:Symmetric group}

\begin{itemize}
\item Something writte as
\[
	\begin{pmatrix}
		1 & 2 & 3 \\ 
		3 & 1 & 2 
	\end{pmatrix}
\]
is called as transformation. And also called \textbf{zhihuan}, 
\item The above transformation can be written as \( (123)\), which is to say that the number on position \(1\) is given 
	to postion \(2\), and number on position \(2\) is given to \(3\), and number on position 3 is given to \(1\).
	
	Note that there can be more than one expression of the transformation.
\item For such thing as \( ( i_{1}  i_{2} i _{3} \dots i _{k})\), the order of it is \(k\).
\item For the compostion of the transformation, we read from left to right. 
	And if we have \(\sigma\), and \(\tau\), and they are: 
\[
\begin{pmatrix}
 1 & 2 & 3\\
	\sigma (1) & \sigma (2) & \sigma (3)
\end{pmatrix}
\qquad 
\begin{pmatrix}
	1 & 2 & 3 \\ 
	\tau (1)  & \tau (2 ) &\tau (3)
\end{pmatrix}
\]
then we have that 
\(\sigma\tau\) is equals to 
\[
\begin{pmatrix}
	1 & 2 & 3 \\ 
	\sigma (\tau 1) &\sigma (\tau 2) &\sigma (\tau 3)
\end{pmatrix}
\]
You may easily know it from the fact that 
\[
\begin{pmatrix}
1 & 2 & 3 \\
\tau (1) &\tau (2) &\tau (3)
\end{pmatrix}
= 
\begin{pmatrix}
\sigma (1) &\sigma (2) &\sigma (3) \\ 
\sigma\tau (1) &\sigma\tau (2) &\sigma\tau (3)
\end{pmatrix}
\]
\item 	
Moreover we have that 
\[
\sigma ^{-1} = 
\begin{pmatrix}
	\sigma (1) &\sigma (2) &\sigma (3) \\ 
	1 & 2 & 3 
\end{pmatrix} 
\]
because we have that 
\[
\begin{pmatrix}
	i_{1} & i_{2} & i_{3} \\ 
	j_1 & j_2 & j_3 
\end{pmatrix}
		{\large \times}
\begin{pmatrix}
	j_1 & j_2 & j_3 \\
	k_1 & k_2 & k_3 
\end{pmatrix} 
= 
\begin{pmatrix}
	i_{1} & i_{2} & i_{3} \\ 
	k_1 & k_2 & k_3 
\end{pmatrix} 
\]
\item There is a theorem that easily compute the \(\sigma\tau\sigma ^{-1}\). 
	If \(\tau\) is written as \( (i_{1} i_2 \dots i_{k})\), then we have that
		\[
			\sigma\tau\sigma ^{-1} =  (\sigma (i_{1}) \sigma (i_{2})\dots\sigma (i_{k}))
		\]
	It is given by the fact that 
	\[
	\sigma\tau\sigma ^{-1} =  
	\begin{pmatrix}
		\sigma (1) &\sigma (2) &\sigma (3) \\
	\sigma (i_{1}) &\sigma (i_{2}) & \sigma (i_{3 }) % TODO there is a problem here: what is i_k? 
	\end{pmatrix}
	\]
	which exactly is saying that \(\sigma\tau\sigma ^{-1} = (\sigma (i_{1})\sigma (i_{2})\sigma(i_{3}))\)
	\item The compostion of transformation is quite tricky. 
	\item If there is no identical number in transformation \(\sigma\), \(\tau\), then \(\sigma\) and \(\tau\) are commutable. 
		that is \(\sigma\tau =\tau\sigma\).
		For a random transformation you can easily find the composition of such form like \(\tau\sigma \varphi \dots \), where the transformations are independent (let us call it temporarily).
\end{itemize}
% section Symmetric group 
\end{document}
