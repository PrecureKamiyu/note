\documentclass[12pt]{ctexart}
\usepackage{amsmath}
\usepackage{amsthm}
\usepackage{amssymb}
\usepackage{amsfonts}
\usepackage[width=12cm,left=2cm,top=3cm]{geometry}
\usepackage{graphicx}
\usepackage{bookmark}
\usepackage{tikz-cd}
\usepackage{hyperref}

\usepackage{tikz}
\usetikzlibrary{automata,positioning}

\setlength{\marginparwidth}{2in}

\theoremstyle{definition}
\newtheorem{definition}{Definition}[section]
\newtheorem{thm}[definition]{Theorem}
\newtheorem{proposition}[definition]{Proposition}
\newtheorem{corollary}[definition]{Lemma}
\newtheorem{lemma}[definition]{Corollary}

\newtheorem{exam}[definition]{Example}
\theoremstyle{remark}
\newtheorem{remark}[definition]{Remark}

\newcommand{\ip}[2]{(#1,#2)}
\font\ninerm=cmr9


\begin{document}
\noindent 第一题
\begin{proof}
\(G/ K\) 交换, 于是
\[
\forall a , b \in G , \quad K a b = K ba 
\]
也就是说, \( a b ( b a ) ^{-1} \in K\). 因为 \(K \le H\), 那么
\( a b (ba) ^{-1} \in H\), 于是
\[
\forall a , b \in G, \quad H ab = H ba
\]
于是 \(G / H\) 交换.
\end{proof}

\newpage
\noindent 第二题
\begin{proof}
分三步走

\medskip%
	\noindent \textbf{1}. 证明 \(\vert\, \text{Aut}(G) \,\vert = \varphi (n)\)

\smallskip%
	\noindent \textbf{2}. 证明 \(\forall f \in \text{Aut} (G)\), 都有
	\[
		\forall g\in G, f (g) = g ^{i}
	\]

\smallskip%
	\noindent \textbf{3}. 验证 \(\text{Aut}(G)\) 是循环的.
\medskip

\noindent \textbf{证明} (1): 
	设 \(g\) 是 \(G\) 的生成元. 设 \(f\) 是一个自同态, 且有 \(f (g) = g^{i}\). 那么我们有: \(f \in \text{Aut}(G)\) 当且仅当 \( \gcd (i, n ) = 1\), 于是说 \( \vert \, \text{Aut} (G) \,\vert = \varphi (n)\). 

	\bigskip
\noindent \textbf{证明} (2): 
	设
	\[
	\forall a\in G, \quad a = g ^{\alpha}
	\]
	那么
	\[
	\forall a \in G, \quad f(a) = f (g ^{\alpha} ) = g^{\alpha i} = a ^{i}
	\]

	\bigskip
\noindent \textbf{证明} (3):
	设 \(f _{i} \colon G \to G, g \mapsto g ^{i}\)

	{
	\setlength{\parindent}{40pt}
	1 和 2. 当 \(\vert G \vert = 1\) 或者 2 的时候, \( \varphi (n) = 1\), 
	则 \(\text{Aut}(G)\) 是循环的

	\smallskip
	3. \(\vert G \vert = 3\), \( \varphi (n) = 2\). \(\text{Aut}(G) = \{ \, f_1, f_2 \, \}\), 因为 \(f_2 ^{2} = f_{1}\)
	则 \(\text{Aut}(G)\) 是循环的

	\smallskip
	4. \(\vert G \vert = 4\), \( \varphi (n ) = 2\). 和 (3) 完全类似. 

	\smallskip
	5. \(\vert G \vert = 5\), \( \varphi (n) = 4\). 
	\[
		\text{Aut}(G) = \{ \, f_1,f_2,f_3,f_4\, \}
	\]
	有
	\[
	\begin{cases}
	f_2^{2} = f_4\\
	f_2^{3} = f_3\\
	f_2^{4} = f_1
	\end{cases}
	\]
	则 \(\text{Aut}(G)\) 是循环的

	\smallskip
	6. \(\vert G \vert = 6\), \( \varphi (n) = 2\). 和 (3) 完全类似. 

	\smallskip
	7. \(\vert G \vert = 7\), \( \varphi (n) = 6\), 且
	\[
		\text{Aut}(G) = \{ \, f_1,f_2,f_3,f_4,f_5,f_6 \, \}
	\]
	有
	\[
	\begin{cases}
		f_{3} ^{2} = f_2 & 9 \bmod{7} = 2\\
		f_3^{3} = f _{6} & 27 \bmod{7} = 6\\
		f_3^{4} = f _{4} & 81 \bmod{7} = 4\\
		f_{3}^{5} = f_{5} & 243  \bmod{7} = 5 \\
		f_{3}^{6} = f_1 & 729 \bmod{7} = 1
	\end{cases}
	\]
	则 \(\text{Aut}(G)\) 是循环的, 证毕.
	}
\end{proof}


\end{document}
