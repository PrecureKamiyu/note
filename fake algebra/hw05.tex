\documentclass[12pt]{ctexart}
\usepackage{amsmath}
\usepackage{amsthm}
\usepackage{amssymb}
\usepackage{amsfonts}
\usepackage[width=12cm,left=2cm,top=3cm]{geometry}
\usepackage{graphicx}
\usepackage{bookmark}
\usepackage{tikz-cd}
\usepackage{hyperref}

\usepackage{tikz}
\usetikzlibrary{automata,positioning}

\setlength{\marginparwidth}{2in}

\theoremstyle{definition}
\newtheorem{definition}{Definition}[section]
\newtheorem{thm}[definition]{Theorem}
\newtheorem{proposition}[definition]{Proposition}
\newtheorem{corollary}[definition]{Lemma}
\newtheorem{lemma}[definition]{Corollary}

\newtheorem{exam}[definition]{Example}
\theoremstyle{remark}
\newtheorem{remark}[definition]{Remark}

\newcommand{\ip}[2]{(#1,#2)}
\font\ninerm=cmr9
\begin{document}
\noindent
1. \(G\) 是有限的, 设 \(\vert G\vert  = n\), 证明对于任意的 \(G\) 的群元素 \(x\), 都有
\[
x ^{n} = 1
\]
\begin{proof}
	因为群元 \(g\) 的生成子群的阶数为 \(g\) 的阶数, \(g\) 的阶数, 因此为 \(n\) 的因子. 这是说 \(n = k * \hbox{ord}\, g\), 
	\(k \in \mathbb{N}\). 则 \(g ^{n} = (g ^{\text{ord}\, g})^{k} = 1 ^{k}  = 1\) 
\end{proof}

\noindent
2. \(H, K \) 是 \(G\) 的两个子群, \(\vert H \vert = n\), \(\vert K \vert = m\). 证明, 如果说 \( (m , n) = 1\), 则 \(H \cap K = \{1 \}\) 
\begin{proof}
\(H \cap K\) 也是子群, 因此, 其阶数为 \(\vert H \vert\), \(\vert K\vert\) 的因子. 因为 \( (m , n) = 1\), 也就是说, 其公共的因子只有 \(1\). 那么 \(\vert H \cap K \vert = 1\), 那么 \(H \cap K = \{e \}\)
\end{proof}



\noindent 
3. (1) \(\tau _{1}\) 的阶数为 \(4\), 因为当 \(\alpha ,\beta \) 是交换的时候 \(\alpha\beta \) 的阶数为 \(  \text{lcm}\, (\text{ord}(\alpha ) , \text{ord}(\beta))\), 而 \( ( i _{1} i_2 \dots  i _{k})\) 的阶数为 \(k\).

(2) \(\tau_{2}\) 的阶数为 \(12\). 

(3) \(\tau _{3} = (163) (245)\), 所以阶数为 \(3\). 

(4)  \(\tau _{4} = (15) ( 27) (346)\), 所以阶数为 \(6\).


\noindent 4. 我们有: 
\[
\sigma = 
\begin{pmatrix}
 1 & 2 & 3& 4 & 5 & 6 & 7 \\ 
 4 & 2 & 1 & 3 & 7 & 6 & 5 
\end{pmatrix}
\]

所以说 \(\sigma \tau\sigma ^{-1} =  ( 125) (26) ( 43)\). 并且我们有: 
\[
\sigma ^{-1} = 
\left( 
	{ 1 \atop 3}
	{ 2 \atop 2}
	{ 3 \atop 4}
	{ 4 \atop 1}
	{ 5 \atop 7}
	{ 6 \atop 6}
	{ 7 \atop 5}
\right)
\]
那么 \(\sigma ^{-1}\tau\sigma =  ( 42 5) (26) ( 31)\)

\centerline{\large introduction} 



\begin{itemize}
\item 什么是置换? 置换群之中的元素就是置换. 
\item 什么是对称群? 集合 \(M\) 上的\textbf{所有}双射构成的群
\item 什么是置换群? 集合 \(M\) 上的\textbf{部分}双射构成的群
\item 置换长什么样? 使用一种不太高明的方法, 可能导致歧义的方法: 
	\[
		\tau = \left( { 1 \atop 1}{ 2 \atop 2}{ 3 \atop 3}{ 4 \atop 4}{ 5 \atop 5}  \right)
	\]
	\(\tau \) 是 \(1\).
\item 还有什么表示方法? \((12)\) 表示 \(1\), \(2\) 位置上的元素互换. \((123)\) 表示 \(1\) 的值赋给 \(2\), \(2\) 的值赋给 \(3\), \(3\) 的值赋给 \(1\). 
\item 为什么要这样表示? 因为简洁一点. 
\item 置换有什么性质? 如果说两个置换 \((ijk), (abc)\), 之中没有相互相等的, 那么 \( (ijk), (abc)\) 是可交换的. 
\item 还有什么性质? \((ijk)\) 的阶为 \(3\), 和其内部的数字个数相等. 
\item 还还有什么性质? \( (i j k)\) 可以进行分解, 分解为两个对换. \((ij)(ik)\), 阅读顺序为左向右.
\item 分解有什么性质? 设标准分解为, 将一个置换分解为对换的复合, 那么一个内部有奇数个数字的置换, 将会生成偶数个对换. 
\item 分解还还有什么性质? 关于 \( \sigma  \tau\sigma ^{-1}\) 的值, 我们有简单计算. 
\end{itemize}
\end{document}
