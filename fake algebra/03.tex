\documentclass[12pt]{ctexart}
\usepackage{amsmath}
\usepackage{amsthm}
\usepackage{amssymb}
\usepackage{amsfonts}
% \usepackage{geometry}
\usepackage{graphicx}
\usepackage{bookmark}
\usepackage{tikz-cd}
% \usepackage{hyperref}
\pagestyle{plain}

\theoremstyle{definition}
\newtheorem{theorem}{Def}[section]
\newtheorem{lemma}[theorem]{Lemma}
\theoremstyle{definition}
\newtheorem{definition}{定义}[section]
\newtheorem{thm}[definition]{定理}
\newtheorem{proposition}[definition]{性质}

\theoremstyle{plain} 
\newtheorem{exam}[definition]{Example}
\theoremstyle{remark}
\newtheorem{remark}[definition]{Remark}

\begin{document}
\section{子群}
\label{sec:子群}

\begin{definition}[subgroup]\label{def:subgroup}
\(H \subset G\) 是 \(G\) 的子群, 如果说 \(H\) 满足下面两个条件. 
\begin{enumerate}
\item  \(1\in H\)
\item  \(\forall h \in H\), \(h ^{-1} \in H\)
\end{enumerate}
	即, \(H\) 也是群, 但其为 \(G\) 的子集. \(G\) 当然是自身的子群, \(\{1\}\) 也是 \(G\) 的子群, 这两种称为平凡子群. \(H\) 是 \(G\) 的子群记为 \(H \le G\), 如果说 \(H \ne G\) 则可以记为 \(H < G\).
\end{definition}
% definition: subgroup 

\begin{exam}[\(\mathbb{Z}\) 的子群]
	\(f_{m} \colon \mathbb{Z} \to \mathbb{Z}, n \mapsto mn, \) 是 \(\mathbb{Z}\) 上的自同态, 能够看出其值域 \(\{ mn \mid n \in \mathbb{Z} \}\) 是 \(\mathbb{Z}\) 的子群. 
\end{exam}
\begin{remark}
	可以证明 \(\mathbb{Z}\) 的子群均形为 \(\{ mn \mid n \in \mathbb{Z} \}\)  
\end{remark}

回想 Monoid 的一个衍生定义, 设 \(U (M_{1})\) 是 Monoid \(M_{1}\) 的子集, 内部元素全部有逆, 可以验证 \(U (M_{1})\) 是群. 对于群有类似的东西. 
\begin{definition}[中心]\label{def:中心}
	\(G\) 的中心 \(C (G)\) 是 \(G\) 的子集, 内部元素与所有群元关于群乘法交换. 
	这是说对于 \(\forall c \in C (G) , g \in G, cg = g c\). 好吧其实不是很类似, 但是有点类似. 
\end{definition}
% definition: 中心 

我们注意中心的定义, 和所有群元关于群乘法交换, 这到底能不能说明其是最大的交换子群? 
假设 \(C (G) \) 不是极大的, 是用反证法, 设存在 \(g\in G - C (G) \), 其满足: 对于任意的 \(g' \in C (G)\), \(g ' g = g g' \), 但是存在 \(g'' \in G\) 使得, \(g '' g \ne g g''\).

\begin{quote}
``群中心作为共轭作用的核所以其重要. ''
\end{quote}

% section 子群 
\end{document}
