\documentclass[../main.tex]{subfiles}
\begin{document}
\chapter{Normal subgroups and quotient groups}
\section{What is quotient groups}
\begin{definition}[Formal subgroups]\label{def:Normal subgroups}
\(N\) is subgroup of \(G\), and if \(N\) suit that \(\forall g \in G, g N = N g\), viz., \( N = g ^{-1} N g\),
then \(N\) is a normal group. 
\end{definition}
% definition: Normal subgroups 
Normal groups have a property that 
\[
 N g_1 N g_2 = g_1 N N g_2  = g_1 N g_2 = N g_1 g_2
\]
Consequently, if we have \( \{ \, [g] \mid g \in G \,\} \), where \(g_1 \sim g_2\) iff \( N g_1 = N g_2\). And then, 
the above property says that \( \{ [g]\}\) is closed under group operation. And furthermore, \(\{ [g]\}\) is a group if we define the operation between \( [ g_{1}] , [g _{2}]\) as \( [g _{1} ] [g_{2}] = [g _{1} g _{2}]\) 
\begin{definition}[Quotient groups]\label{def:Quotient groups}
	Given a normal subgroup \(N\) we have that \(\{ \, [g] \mid g \in G \,\}\)  is a group, denoted as 
	\(G / N\), and we called \(G / N\) a quotient group.
\end{definition}
% definition: Quotient groups 
\begin{definition}[Conjugate]\label{def:Conjugate}
Given a group member \(g\) the conjugation of \(g\) is defined as \( h \mapsto g ^{-1} h g\). If \(\exists g \in G\)
s.t. \(h_1 = g ^{-1} h_2 g \), then we say that \(h_1\) and \(h_2\) are conjugate. The relation of conjugate is 
a equivalence relation.
\end{definition}
% definition: Conjugate 

\begin{exam}
While it remain a little bit ambiguous that we choose normal subgroups to construct 
quotient group, we can have a look at quotient in linear space and topological space 
to further understand what quotient is.

In linear algebra, every subspace of a linear space is a normal group if we treat it as group. 
Given a linear space \(V\) and a subspace \(W\), we have that 
\[
\dim V / W = \dim V - \dim W
\]
It seem that the subspace \(W\) is eliminated and that the space which is orthogonal to \(W\) is isomorphic to 
\(V / W\). 

Consider the topological space \(X\), and given a equivalence relation of \(X\), we can construct a quotient space 
	\(Y\), where every member is the equivalence class of the relation. \textbf{And} the family of open sets \(\mathscr F'\) suit that for the function \(\pi \colon O \mapsto \bigcup_{ x \in O} [x]\), 
	we have that \(\pi (O)\) is an open set in \(Y\) iff \(O\) is an open set in \(X\). 

	The quotient space of topological space \(X\) is also called identical space. That is to say, we glue the 
	members in a class into a piece. And that is what we called quotient.  \qed
\end{exam}
% section: What is quotient groups

\section{A basic homomorphism theorem of quotient}
\label{sec:A basic homomorphism theorem of quotient}
Next we talk about an important theorem about quotient groups, before which, we first introduce some 
definitions.
\begin{definition}[kernal]\label{def:kernal}
	Given a homomorphism \(f \colon G \to G '\) , the kernal of \(f\) (denoted as \(\ker f\)), is defined as 
	\[
		\ker f = f ^{-1} (1) \subseteq G
	\]
	and it is easy to show that \(\ker f\) is a normal subgroup of \(G\).
\end{definition}
% definition: kernal 
\noindent \textbf{Exercise:} Prove that \(\ker f\) is a normal subgroup of \(G\).
\medskip

\begin{definition}[Image]\label{def:Image}
The image of \(f\) is defined as 
\[
	\text{Im}\ f = \{ \, f (g) \mid g \in G\,\}
\]
The image of \(f\) is less important than kernal, since \verb|\ker| is a macro provided by \LaTeX\ while that \verb|\Im| or \verb|\im| does not exist.
\end{definition}
% definition: Image 
Let us state the theorem
\begin{thm}[a basic theorem of quotients]
\label{a basic theorem of quotients}
	Given a homomophism \(f \colon G \to G'\), we define \(\bar f\) as that \( \bar f ([g]) = f(g)\), and we have that 
	\[
		\bar f \colon G/ \ker f \to \text{Im}\ f
	\]
	is an isomorphic.
\end{thm}
% theorem a basic theorem of quotients 
\begin{proof}
	We shall prove that \(\bar f\) is a homomorphism and also a bijective. Then we prove that \(\bar f\) is isomorphism.
\end{proof}
% section A basic homomorphism theorem of quotient 
\end{document}
