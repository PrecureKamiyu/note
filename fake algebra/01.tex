\documentclass[10pt]{ctexart}
\usepackage{amsmath}
\usepackage{amsthm}
\usepackage{amssymb}
\usepackage{amsfonts}
\usepackage{graphicx}
\usepackage{bookmark}
\usepackage{hyperref}

\theoremstyle{definition}
\newtheorem{theorem}{Def}[section]
\newtheorem{lemma}[theorem]{Lemma}
\theoremstyle{definition}
\newtheorem{definition}{定义}[section]
\newtheorem{thm}[definition]{定理}

\theoremstyle{plain} 
\newtheorem{exam}[definition]{Example}
\theoremstyle{remark}
\newtheorem{remark}[definition]{Remark}

\pagestyle{plain}

\begin{document}
\title{集合论}
\author{你野爹}
\maketitle
\tableofcontents
\newpage

\section{一点介绍}\label{sec:intro}
What is algebra? 我们专业的人现在可能会说 ``不知道'', 并且以后也可能维持这个不知道的状态; 而有的人会说, 这是数学的两个基础课程之一, 另外一个是数学分析; 但是有的人会说, 给定一个集合 \(A\), \(A\) 上的一个 algebra 是 \(A\) 的幂集的子集 \(\mathcal A\), 满足对于任意的 \(\alpha_{1}, \alpha_{2} \in \mathcal A\) 都有 \(\alpha_{1} \cup \alpha_{2} \in \mathcal A, \alpha_{1} \cap \alpha_{2} \in \mathcal A\), 且 \(A , \varnothing \in \mathcal A\). 

Algebra 是一种代称, 大部分的基础代数课程可以称为``群环模域''课, 这是在说, 这门课的主要内容为这个四个单字, 他们是四种代数结构. 那么什么是代数结构, sa\~ , 谁知道呢? 人连集合的定义都说得不明不白, 怎能说清别的呢? 当我说出 ``集合以及集合上的代数运算'' 的时候, 是否有意识到, 仅为都合之意, 才说其为 ``集合上的'' ? 或许我们可以说, 代数结构是, 满足某些运算性质的 collection. 这么说是否严谨一些呢? 我们后面将会意识到, 代数结构的性质, 出自于代数结构的定义, 也即, 其满足的运算性质. 

可能会有人想到, ``终究是错付了人'', 也许是对的, 代数这样重要的课程, 落得这般境地, 可称可悲, 着实引人唏嘘. 错付了人呐. 

\paragraph{伽罗瓦} 可否有人听闻过伽罗瓦? 伽罗瓦开创了群论, 证明了五次方程五根式解. 当我们认为数学均是与数字打交道的时候, 面对这个问题, 指定是摸不着头脑. 一般的五次方程长这个样子: \(a x^{5} + b x ^{4} + c x ^{3} + d x ^{2} + e x + f\), 没有一个数字. 更难的是: 如何证明不存在? 存在的话简单, 我们找到那个根式就行了. 可是不存在呢? 


\section{集合之间的运算和性质}\label{sec:set}
虽然我们还是无法准确地说出集合定义, 但是我们有两种朴素的描述方法. 
\begin{definition}
我们说一些元素放在一起便是集合; 或者是满足一些性质的元素的全体
\begin{equation}
A = \{ a \mid P (a) \} 
\end{equation}
或者是说, 我们将某些东西用 \(\{ \}\) 框起来便是集合.  我们常用大写字母来表示集合. 
\end{definition}
\begin{remark}
人们常用 \(\mathbb Z\) 来表示 \(\{\dots ,  -1 ,0 , 1 , 2 , \dots \} \) --- 整数集, 这是二十世纪的布尔巴基的著作之中使用的符号, 之后得到了流传, 类似的符号还有 \(\mathbb N, \mathbb Q , \mathbb R, \mathbb C\) 分别表示自然数集, 有理数集, 实数集, 复数集. 
\end{remark}

\begin{definition}[属于]
 \(a \in A \) 是说, \(a\) 是 \(A\) 的一个元素, 也可以说 \(a\) 是 \(A\) 的一个成员. 
\end{definition}
尽管说我不能将 \(\in\) 的定义说清楚, 但是, 差不多就行. 
\begin{definition}[subset] 
\(A \subseteq B \) 是说 \(A \) 是 \(B\) 的子集, \(A \subsetneq B\) 意指 \(A\) 是 \(B\) 的真子集. 对于前者来说: 
\begin{equation}
A \subseteq B \iff \forall a \in A (a \in B)
\end{equation}
对于后者来说: 
\begin{equation}
A \subsetneq B \iff (\forall a \in A (a \in B)) \wedge (\exists a' \in A  (a' \notin B))
\end{equation}
\(A \subsetneq B\) 就是说 \(A\) 是 \(B\) 的子集, 但是 \(A \ne B \). 
\end{definition}
\begin{remark}
我们现在有三种符号 \(\subseteq , \subset , \subsetneq\), 都可以说是包含于的二元运算符\footnote{运算符并不是真的指 `运算`}, \(\subsetneq\) 并没有歧义, 表示的是 ``真包含'' 的意思, \(\subset\) 既可以指 ``包含'' 也可以指 ``真包含'', 得看他怎么说的.
\end{remark}
\begin{definition}[equal]
\(A = B \) 的定义如下: 
\begin{equation}
A = B \iff A \subseteq B \wedge B \subseteq A 
\end{equation}
\(A \ne B \) 定义为 \(\neg (A = B ) \) , 于是我们知道: 
\begin{equation}
A \ne B \iff  \exists a \in A (a \notin B) \vee \exists b \in B (b \notin A) 
\end{equation}
\end{definition}
\begin{definition}[交和并]
我们学过的, 交的定义为: 
\begin{equation}
A \cap B = \{ x \mid x \in A \wedge x \in B \} 
\end{equation}
并的定义就省略不说了. 
\end{definition}
\begin{definition}[Power] 
我们断言, 对于任意一个集合, 存在其幂集, 记为 \(P (A) \) 或者是 \(\mathfrak P (A) \) 或者是 \(2 ^{A}\), 其定义为
\begin{equation}
\mathfrak P (A)  = \{ B \mid B \subseteq A \} 
\end{equation}
\end{definition}
\begin{definition}[Product]
乘积的定义并不好说. 我们当然可以引入有序对的说法, 可是, 有序对真的是 product 的本质吗? 关注两个集合的product的势, 其为两个集合的势的乘积. 好吧, 我有点搞不懂. 
\end{definition}
\section{映射}
人们常说, 映射是一个对应法则. 我说, 这是什么几把. 别tm复读别人瞎几把说的定义. 我说, 一个映射, 或者说函数, 先要给定函数的值域和定义域, 函数将定义域之中的元素射到值域中的元素上. 我们这样表示: 设 \(f\) 是函数
\begin{equation}
f \colon X \to Y , x \mapsto f (x) 
\end{equation}
这就是说, \(f (x) \in Y\) \(f (x) \) 是 \(Y\) 之中的一个元素. 这便是一个函数. 你尽可以认为什么 ``允许多对一, 不允许一对多'' 完全是扯淡. 
\begin{definition}[inverse]
函数 \(f\) 的逆记为 \(f ^{-1}\), 我可以将其定义为函数, 你看就行了. 
\begin{equation}
f^{-1} \colon \mathfrak P (B) \to \mathfrak P(A) , \mathcal B \mapsto f ^{-1} (\mathcal B) 
\end{equation}
为了方便, 对于单元集, 我们将 \(\{ \}\) 省略, 也就是将 \(f ^{-1} (\{b \} ) \) 简写为 \(f ^{-1} (b)\). 逆的定义也好写: 
\begin{equation}
f ^{-1} (\mathcal B ) = \{ x \mid f(x) \in \mathcal B \} 
\end{equation}
\end{definition}


\end{document}

