\documentclass[12pt]{ctexart}
\usepackage{amsmath}
\usepackage{amsthm}
\usepackage{amssymb}
\usepackage{amsfonts}
\usepackage[width=18cm,top= 3cm]{geometry}
\usepackage{graphicx}
\usepackage{bookmark}
\usepackage{tikz-cd}
\usepackage{hyperref}

\theoremstyle{definition}
\newtheorem{definition}{Definition}[section]
\newtheorem{thm}[definition]{Theorem}
\newtheorem{proposition}[definition]{Proposition}
\newtheorem{corollary}[definition]{Lemma}
\newtheorem{lemma}[definition]{Corollary}

\theoremstyle{plain} 
\newtheorem{exam}[definition]{Example}
\theoremstyle{remark}
\newtheorem{remark}[definition]{Remark}

\begin{document}
\title{作业3}

\paragraph{1. } 证明 \(G\) 的任意个子群的交仍是子群. 
\begin{proof}
	设 \(\{ H _{i} \}_{i\in \Lambda \subseteq \mathbb{N}}\) 是这些子群. 需要证明
	\[
	\bigcap _{i \in \Lambda} H_{i} 
	\]
	仍是子群. 

	对于任意的 \(a , b \in \bigcap _{i \in \Lambda } H _{i}\), 也即满足 \(\forall i \in \Lambda , a ,b \in H_{i}\), 因为 \(H_{i}\) 对群乘法封闭, 于是 \(a , b \in H _{i} \implies a b \in H_{i}\). 
	于是能够知道 \(\forall i \in \Lambda (a, b \in H_{i} \implies ab \in H_{i})\), 就有 
	\[
	\forall i \in \Lambda ( a , b \in H _{i} ) \implies \forall i \in \Lambda (ab \in H_{i})
	\]
	于是就有 \(a , b \in \bigcap _{i \in \Lambda} H_{i} \implies ab \in \bigcap _{i \in \Lambda } H_{i}\), 所以 \(\bigcap _{i \in \Lambda } H_{i}\) 对群乘法封闭. 

	因为 \(\forall i \in \Lambda ( e \in H_{i})\) 成立, 所以说 \(e \in \bigcap_{i\in\Lambda}H_{i}\) 成立. 

	同理
	\[
		\forall i \in \Lambda ( a \in H_{i} ) \implies \forall i \in \Lambda (a ^{-1} \in H_{i})
	\]
	成立, 于是有 \(a \in \bigcap _{i \in \Lambda} \implies a ^{-1} \in \bigcap _{i \in \Lambda} H_{i}\). 所以, \(\bigcap _{i \in \Lambda} H_{i}\) 是子群. 
\end{proof}

\paragraph{2.} 证明 \(\text{GL}_{n}(F)\) 的群中心是 \(n\) 阶纯量矩阵.

\begin{proof}
因为 \(T \in C(\text{GL} _{n} (F) \iff \forall S \in \text{GL} (F) , TS = ST\), 我们要证明的即是后者.  将群元看作是线性变换. 于是
\[x \in \text{Ker}\ T \implies Tx = \vec{0} \implies ST x = \vec 0 \implies TS x = \vec 0 \implies Sx \in \text{Ker}\ T\]

设 \(S\, \text{Ker}\ T = \{ Sx \mid x \in \text{Ker} \ T \}\), 这说明 \(S\, \text{Ker}\ T \subset \text{Ker}\ T\). 设 \(\text{Ker}\ T\) 的基底 \(\alpha_{1} , \dots ,\alpha_{t}\), 该基底可以扩展为 \(F^{n}\) 上的基底, 记为 \(\alpha _{1} , \dots ,\alpha_{t},\alpha_{t + 1} , \dots ,\alpha _{n}\). 因为 \(S\) 是任意的, 设 \(S\) 对基底的作用为: 
\[
S(\alpha _{i}) = 
\begin{cases}
	\alpha_{i} & i \ne 1 , i \ne t + 1\\
	\alpha_{1} & i = t + 1 \\
	\alpha_{t + 1} & i = 1
\end{cases}
\]
这就是说, 其将 \(\alpha_{1},\alpha_{t+1}\) 两者的位置进行了互换. 那么说, 可以知道, \( \text{dim}\, \text{Ker}\,T\)  要么为 \(n\) , 要么为 \(0\). 否则 \(S\, \ker T \subset \ker T\) 不成立. 

随后, 令 \(S\) 为一个投影到\(F^{n}\)的一维子空间上的线性变换. 设这个子空间为 \(\langle v \rangle = \{ k v \mid  k\in F\}\).对于 \(v_1 \in \langle v \rangle\) 就有:
\[
Tv_1 = T(Sv_{1} ) = S (Tv_1) =\alpha v_1 ,\alpha \in F
\]
\(S (Tv_{1}) =\alpha v_1\) 成立是因为 \(S\) 是一个投影. 对于任意的 \(v  \in \langle v \rangle\) 有 \( T v  =\alpha v\). 也就是说,
\(T\) 至少有一个特征值. 又有: 
\[
T S  = ST \implies T S  -\alpha S = ST -\alpha S \implies (T -\alpha I ) S  = S ( T -\alpha I ) 
\]
由前面对于 \(\text{Ker}\,T\) 的维数的讨论, \(\text{Ker}\, (T -\alpha I)\) 的维数要么为 \(0\) 要么为 \(n\). 
但是 \(T\) 存在特征值 \(\alpha\), 这是说 \(T -\alpha I\) 的行列式为 \(0\), 等价于说 \(\text{dim}\, \text{Ker}\,(T-\alpha I)\) 的为维数大于等于 \(1\). 于是就有 \(\text{Ker}\, (T -\alpha I)\) 的维数为 \(n\) (因为其要么为 \(0\), 要么为 \(n\)). 

\noindent 由于:
\[
	\text{dim}\, \text{Ker}\, (T  -\alpha I) = n \iff T -\alpha I = \mathbf 0
\]
其中 \(\mathbf 0\) 代表零矩阵. 于是 \(T =\alpha I\). 即 \(T\) 是纯量矩阵. 
\end{proof}
\end{document}
