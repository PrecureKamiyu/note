 \documentclass[12pt]{ctexart}
 \usepackage{amsmath}
 \usepackage{amsthm}
 \usepackage{amssymb}
 \usepackage{amsfonts}
 \usepackage[width=16cm]{geometry}
 \usepackage{graphicx}
 \usepackage{bookmark}
 \usepackage{tikz-cd}
 \usepackage{hyperref}
 
 \usepackage{tikz}
 \usetikzlibrary{automata,positioning}
 
 \setlength{\marginparwidth}{2in}
 
 \theoremstyle{definition}
 \newtheorem{definition}{Definition}[section]
 \newtheorem{thm}[definition]{Theorem}
 \newtheorem{proposition}[definition]{Proposition}
 \newtheorem{corollary}[definition]{Lemma}
 \newtheorem{lemma}[definition]{Corollary}
 
 \newtheorem{exam}[definition]{Example}
 \theoremstyle{remark}
 \newtheorem{remark}[definition]{Remark}
 
 \newcommand{\ip}[2]{(#1,#2)}
 \font\ninerm=cmr9

\begin{document}
\noindent \textbf{1.} \(G = \langle a \rangle\) 是 \(6\) 阶循环群, 给出 \(G\) 的一切生成元和 \(G\) 的所有子群. 
\begin{proof}
	设 \(g = a ^{k}\), \(g\) 是生成元则 \(k\) 和 \(6\) 互素, 那么 \(k = 1 \text{ or } 5\), 故生成元为 \(a\) 或者 \(a ^{5}\)

	设子群为 \( \langle a^{m} \rangle\), 其为子群则 \(m\) 是 \(6\) 的因子, 故 \(m\) 的可能取值为 \(1 , 2, 3 , 6\). 
	子群共有四个. 
\end{proof}

\begin{thm}[生成元的个数]
\label{生成元的个数}
有限循环群的生成元的个数为 \( \varphi ( n)\) 其中 \(n\) 是 \(G\) 的阶, \( \varphi\) 是欧拉函数. 
\end{thm}
% theorem 生成元的个数 

\begin{thm}[循环群的子群]
\label{循环群的子群}
循环群的子群必定为循环群. 
\end{thm}
% theorem 循环群的子群 

\begin{thm}[子群的阶数]
\label{子群的阶数}
对于有限群 \(G\), 由 Lagerange 定理, 有, 子群的阶数必定为 \(|G|\) 的因子.
\end{thm}
% theorem 子群的阶数 

\noindent \textbf{2. } \(M = \{ 1, 2 ,3 ,4 \} , H = \{ \tau , \sigma\}\), where 
\[
r = \left(
\begin{matrix}
	1 & 2 & 3 & 4 \\
	1 & 1 & 3 & 4
\end{matrix}
\right)
\qquad 
\sigma = \left(
\begin{matrix}
	1 & 2 & 3 & 4 \\
	1 & 1 & 3 & 3
\end{matrix}
\right)
\]
\( H\) 关于变换的乘法是否做成有单位元的半群? 是否做成群? 

\begin{proof}
可以验证:
\[
\begin{cases}
\, \sigma \tau =\sigma \\ 
\, \tau \sigma =\sigma \\ 
\, \sigma\sigma =\sigma \\ 
\, \tau \tau = \tau 
\end{cases}
\]
于是 \(H\) 关于变换的乘法封闭, 且变换的合成自然满足结合律, 且 \(\tau\) 是单位元, 于是 \(H\) 是幺半群. 但不是群, 因为 \(\sigma\) 没有逆元, 这是说, \( \tau \) 或者 \( \sigma \) 都不满足其乘以 \(\sigma\) 为单位元 \(\tau\).
\end{proof}
\medskip

\noindent 纯水题.
参见课本变换群例2. 
\end{document}
