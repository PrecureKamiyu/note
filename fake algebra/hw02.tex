\documentclass[12pt]{ctexart}
\usepackage{amsmath}
\usepackage{amsthm}
\usepackage{amssymb}
\usepackage{amsfonts}
% \usepackage{geometry}
\usepackage{graphicx}
\usepackage{bookmark}
\usepackage{tikz-cd}
% \usepackage{hyperref}

\theoremstyle{definition}
\newtheorem{theorem}{Def}[section]
\newtheorem{lemma}[theorem]{Lemma}
\theoremstyle{definition}
\newtheorem{definition}{定义}[section]
\newtheorem{thm}[definition]{定理}
\newtheorem{proposition}[definition]{性质}

\theoremstyle{plain} 
\newtheorem{exam}[definition]{Example}
\theoremstyle{remark}
\newtheorem{remark}[definition]{Remark}

\begin{document}
\title{作业2}
\maketitle
\paragraph{1. }\(M\) 是一个非空集合, 设 \(I = \{ (a, b ) \mid a , b \in M\}\). 证明, \(M\) 的一个关系决定了 \(I\) 的一个子集, \(I\) 的一个子集决定了 \(M\) 的一个关系, 不同的关系决定了 \(I\) 的不同子集. 
\begin{proof}
	设 \(I ' \subseteq I\). 
	设 \(a R b\) \(\implies\) \( (a, b ) \in I'\), \( a \bar R b \implies ( a ,b ) \notin I'\). 那么对于任意 \(a, b \in M\), 要么 \(a R b\), 要么 \( a \bar R b \), 因为
	\((a, b ) \) 要么 \( ( a, b ) \in  I' \) , 要么 \( (a, b ) \notin I' \). 这就是说, \(I'\) 确实定义了一个关系 \(R\). 

	给定一个 \(I'\) 我们能够确定一个 \(R\). \((a, b ) \in I' \implies a R b \), \( ( a, b) \notin I' \implies a \bar R b \). 这就确定了一个关系 \(R\), 因为\((a, b ) \) 要么 \( ( a, b ) \in  I' \) , 要么 \( (a, b ) \notin I' \).  

	设有两个关系 \(R _{1}\ne R_2\). 那么存在 \((a, b ) \in  I\), 使得 \( a R _{1} b, a \bar R_2 b\) 成立, 或者 \(a \bar R_1 b, a R_2b\) 成立. 那么 \(R_1 , R_2\) 分别确定的 \(I_1' , I_2'\) 一定不同因为 \((a, b )\) 不会同时属于或者不属于 \(I_1', I_2'\)
\end{proof}


\paragraph{2. }设 \(G = \{ (a, b ) \mid a, b \in \mathbb{R} , a \ne 0 \}\), 且 \(\circ\) 的定义为 \( (a, b ) \circ (c ,d ) = ( a c , ad + b ) \) 证明 \( ( G , \circ ) \) 是一个群. 如果是其是群, 其是否为交换群? 
\begin{proof}
\[
	\begin{aligned}
		(a , b ) \circ (( c, d )  \circ (e ,f ) )
		& =  (a,  b ) \circ ( c e , cf + d )   \\
		& =  ( ace , a (cf + d) + b )  \\ 
		& = (ace , a cf + ad + b ) 
	\end{aligned}
\]
而 
\[
	\begin{aligned}
		(( a, b) \circ ( c , d )) \circ (e ,f )  
		& = (  ac , ad + b ) \circ (e ,f ) \\ 
		& = ( ace , acf + ad +b ) 
	\end{aligned}
\]
可以知道 
\[
	(( a, b ) \circ ( c ,d )) \circ ( e, f ) = ( a, b ) \circ( ( c ,d ) \circ ( e, f ))
\]
即, \(\circ\) 满足结合律, \(G\) 是半群, 只需验证其具有左幺元和左逆元. 
	明显, \( ( 1 , 0 ) \) 是左幺元. 
\[
	\big(\frac{1}{c} , - \frac{d}{c} \big) ( c ,d ) = \big( \frac{1}{c} \cdot c , \frac{1}{c} \cdot d + (- \frac{d}{c}) \big) = (1, 0) 
\]
	故 \(\Big(\dfrac{1}{c} , - \dfrac{d}{c}\Big)\) 是 \( (c, d ) \) 的左逆元, 故 \(G\) 是群. 
\end{proof}

\paragraph{3. }证明群 \(G\) 中 \(a , a ^{-1}, c a c ^{-1}\) 的阶相等. 
\begin{proof}
设 \(a \) 的阶为 \(n\), \(a ^{n} = 1\) 
因为 
\[
	\begin{aligned}
	(a^{-1} ) ^{n} a ^{n}  
	&  = (a ^{-1} ) ^{ n-1} a ^{-1} a a ^{n-1} \\ 
		& = ( a ^{-1} ) ^{n-1}  1 a ^{n-1} \\
	& = \dots \\
	& = 1 
	\end{aligned}
\]
故 \( (a ^{-1} ) ^{n} = ( a ^{n} ) ^{-1} = 1 ^{-1} = 1\), \(a ^{-1}\) 的阶为 \(n\). 
\[
\begin{aligned}
	(c a ^{-1} c ^{-1} ) ^{n}  
	& = c a ^{- 1} c  ^{-1} c a ^{-1} c ^{-1} ( c a ^{-1} c ^{-1} ) ^{n-2}  \\
	& = c (a ^{-1} )  ^{2} c ^{-1} (c a ^{-1} c ^{-1} ) ^{n-2} \\ 
	& = \dots \\
	& = c  ( a ^{-1} ) ^{n } c ^{-1} \\ 
	& = c  1 c ^{-1} \\ 
	& = 1 \\ 
\end{aligned}
\]
那么 \(c a^{-1} c ^{-1}\) 也是 \(n\) 阶的. 
\end{proof}

\paragraph{4. }设 \(a \) 是群 \(G\) 中的一个 \(n\) 阶群元, 证明
\[
	a ^{s} = a ^{t} \iff n \mid ( s -t ) 
\]
\begin{proof}
	``\( \Rightarrow \)'':  \(a ^{s} = a ^{t} \) 可以推出 \(a ^{ s -t}  =1 \), 其中 \(a ^{ s  - t}\) 定义为 \(a ^{ s} (a ^{t}) ^{-1}\). 
	因为 \(a ^ {-1} \) 也是 \(n\) 阶元, 因此 \(s - t\) 大于零或者小于零的时候, 下面结论均成立: 根据阶的定义, \(s -t\) 一定被 \(n\) 整除. 

	``\(\Leftarrow \)'':
	\(n \mid  ( s -t ) \) , 那么设 \( s = k _{1} n + c , t = k_{2} n + c\), 其中 \( k_{1} ,k_2 , c\in \mathbb{Z}\), 则 
	\[
	a ^{s} = a ^{k _{1}  n +c } = a ^{k_{1} n} a ^{c} = a ^{c} = a ^{k_{2} n } a ^{c} = a ^{k_{2}n + c } = a ^{t} \qedhere
	\]
\end{proof}
\end{document}

