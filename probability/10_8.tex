\documentclass[a4paper, 10pt]{ctexart} %中文支持
\usepackage{float}              %防止浮动元素浮动
\usepackage{rotating}           %旋转图片用的
\usepackage[]{amsmath}          %数学公式
\usepackage{amsfonts}           %加载数学字体, 比如说\mathbb, 没有这个宏包的话可能会报错
\usepackage{amsthm}             %定义, 定理, 证明, 例子这些环境的支持
%使用方法:
%\newtheorem{environment name}{caption}
%比如 \newtheorem{example}{这是例子}
%效果 \begin{example} xxx \end{example} -> 这是例子 1 xxx
%proof就不需要了
\usepackage{graphicx}           %用来插入图片
\usepackage[left=1.25in,right=1.25in,top=1in,bottom=1in]{geometry}   %用来排版的
\usepackage[]{color}            %用来给部分文本上色的
\usepackage{algorithm}          %用来写伪代码的
\usepackage{algorithmic}        %同上
%\usepackage{minted}
\usepackage{extarrows}
%以下是宏包 amsthm 的命令, 我们使用这些环境的时候必须先对其进行一个定义

\newtheorem{theorem}{定理}
\newtheorem{example}{Example}
\newtheorem{definition}{Definition}
\newtheorem{lemma}{Lemma}

\begin{document}
\section{求解形如 $g \left(X, Y\right)$ 的分布}
\subsection{求解r.v. 和的分布函数}
给定r.v. $X,  Y$, 求 $X + Y$ 的分布函数 $F$

面对这个问题, 理论上我们可以首先引入几个例子, 然后分为离散型r.v. 和绝对连续型的r.v. 进行讨论. 
但其实我觉得并没有这些必要, 我们直接引入即可. 
但是分类讨论在这种情况之下还是必须的, 这是因为求法不同. 一边是可以直接求和的方式求解, 和本节主要内容关系不大. 绝对连续型的, 就需要从密度函数出发.
\subsubsection{离散的情况}

对于离散型r.v. 设 $X, Y$ are discrete, 仅在自然数上有定义\footnote{我们遇到的大部分离散r.v. 都是这样的}, 即 $P\left(X = j\right) \ge  0 $ if $j \in \mathbb{N}$. 新的分布函数是 $F \left(z\right) = P\left(X + Y \le z\right)$ 

有 $$F \left(z\right) = \sum_{i = 0 } ^{ z} P \left(X = i\right) P\left(Y = z - i\right) $$, 这种就是书上介绍的方法, 因为我们平时接触的离散r.v. 并没有那么奇葩 (比如说 $\sum_{i =1} ^{\infty} a_i = 1$ , $\left\{b_i\right\}$ 是有理数的穷举, 我们有离散型r.v. $\sum_{i=1}  ^{\infty} a_i I _{b_i}$ \footnote{$I_{b_i} \left(x\right)= 1$ if $x > b_i$, else $I _{b_i} \left(x\right) = 0$})

\begin{example}
    X , Y 相互独立, 且是参数为 $\lambda_1 , \lambda_2$ 的泊松分布, 如果说泊松分布的记号是 $P$ 就是说 $X \sim P\left(\lambda_1\right), Y \sim P \left(\lambda_2\right)$
    问 $X + Y$ 是啥子分布?

    我们套用上面的公式 
    \[
    \begin{aligned}
        F (z) & = P \left(X  + Y \le z\right) \\
        & = \sum_{i=0} ^{z}  P \left(X = i\right)  P \left(Y = z - i\right)\\
        & = \sum_{i=0} ^{z} \frac{\lambda_1 ^{i}}{i !} e ^{ - \lambda_1}\frac{ \lambda _{2}^{ z-i} }{\left(z-i\right) !} e ^{- \lambda_2}\\
        & = \frac{1}{z !}e ^{- \left(\lambda_1 + \lambda_2\right)}\sum_{i=0} ^{z} \frac{z !}{ i! \left(z -i\right) ! } \lambda ^{i}_{1} \lambda_{2}^{z -i}\\
        & = \frac{1}{z !}e ^{- \left(\lambda_1 + \lambda_2\right)}\sum_{i=0} ^{z} C_{z}^{i} \lambda _{1 }^{i}  \lambda_{2}^{z-i}\\
        & = \frac{\left(\lambda_1 + \lambda_2\right)^{z}e ^{ - \left( \lambda_1 + \lambda_2\right)}}{z ! }  
    \end{aligned}
    \]
 $F (z) \sim P ( \lambda_1 + \lambda_2)$这就说明, 泊松分布具有一种线性可加性. 实际上对于伯努利分布也是如此.
\end{example}

\subsubsection{连续的情况}
首先还是从定义看:
\[
F \left(z\right) = P\left(X + Y \le z\right)
\]
我们可以联想到 $g \left(X\right)$ 的求法. 假设 $X$ 是绝对连续的, $g\left(X\right)$ 的分布函数 $\displaystyle F = \int _{G} f \left(x \right) \ \mathrm{d}x$, where $G = \left\{ x: g\left(x\right) \le z\right\}$

类似的, 对于随机变量之和的分布函数
\[ \displaystyle F \left(z\right) = \iint _{G} f_{X , Y} \left(x,  y\right) \ \mathrm{d}x \ \mathrm{d}y\] , 其中 $G = \left\{ \left(x,  y\right) : x + y \le z\right\}$
当然这里能够直接扩展到一般的情况, 即 $G=  \left\{ \left( x,  y\right) : g\left(x , y\right) \le z\right\}$, 这点我们之后在介绍.
\begin{lemma}
    给定了两个随机变量 $X, Y$ , 设 $f _{X,   Y   } \left(x,  y\right)$ 是其密度函数, 则 $Z = X +  Y$ 的密度函数 $f _{Z}$等于 
    \[
    f \left(z\right)   = \int  ^{ + \infty} _{ - \infty} f \left( x, z -x\right) \ \mathrm{d}x
    \]\[
    f \left(z\right) = \int  ^{ + \infty    } _{ - \infty } f \left(z -x , x \right) \ \mathrm{d}x
    \]
\end{lemma}
\begin{proof}
    \[
    \begin{aligned}
    F \left(z \right) & = \iint  _{G} f\left(x,  y\right)  \ \mathrm{d}x \ \mathrm{d}y \\ 
    & = \int ^{ + \infty  } _{ - \infty}  \left( \int  ^{ z -x} _{ - \infty } f \left(x, y\right) \ \mathrm{d}y\right)\ \mathrm{d}x\\
    \end{aligned}
    \]
    令 $y  = u - x$ 就有 
    \[
    F \left(z\right)  = \int  ^{+\infty} _{ - \infty} \left( \int  ^{z} _{-\infty } f \left(x, u -x \right) \ \mathrm{d}u \right) \ \mathrm{d}x
    \]
    交换积分次序: 
    \[
    F \left(z\right) = \int ^{z} _{- \infty } \left( \int  ^{ +\infty} _{ -\infty} f \left(x, u-x\right)\ \mathrm{d}x \right)\ \mathrm{d}u
    \]
    两边求导, 就能够得出答案了.
\end{proof}

\begin{theorem}
    $X , Y$ 是独立的, 那么 $X + Y $ 的密度函数是 $f\left(z\right) = \displaystyle \int ^{ + \infty } _{-\infty  } f_{X} \left(x \right) f_{Y} \left(z - x\right) \ \mathrm{d} x$
\end{theorem}
\begin{proof}
    使用上面的引理

    因为是独立的, 立刻能够得到结果.
\end{proof}

其中 $f _{X} , f _{Y}$ 涉及的计算称为卷积, 记为 $f _{X} * f _Y$

\begin{example}
    设 $X , Y \sim U \left[ -a  ,a \right]$, 求 $Z = X + Y$ 的分布函数
\end{example}
\begin{example}
    设 $X \sim N\left( \mu_1, \sigma_1^{2} \right), Y \sim N \left( \mu_2 , \sigma_{2} ^{2}\right)$, 求 $Z = X + Y$ 的分布. 
\end{example}

而后我们指出, 引理中使用的公式实际上可以直接用来求解问题, 当你涉及两个不独立的随机变量的时候可以这么做. 
\subsection{$\min  \left\{ X , Y \right\}, \max \left\{ X, Y\right\} $的求解}
$X \vee Y $ 就是两者取最大值的意思, $X \wedge Y$ 是取最小值的意思. 

我们可以将 $\vee , \wedge$ 看作是两个二元函数 (而实际上他们确实是二元运算符, 所以也没什么差别), 于是这就是求解 $g\left(X, Y\right)$ 
的范畴. 我们可以很轻松地解出来, 见下面的推导.


我们可以直接推导出公式: 
\begin{theorem}
    $F_{x \vee y} \left(z\right) = F_{x} \left(z\right) F_{y} \left(z\right)$  
\end{theorem}
\begin{proof}
\[
\begin{aligned}
F_{ X \vee Y} \left(z\right)  & = P\left( X \vee Y \le z\right)\\
& = P\left(X \le z \quad  \& \quad Y \le z\right)\\
& = P\left( X \le z\right) \cdot P\left(Y\le z\right)\\
& = F_{X } \left(z\right) \cdot  F_{Y} \left(z\right) \qedhere
\end{aligned}
\]
\end{proof}
对于 $X \wedge Y$ 也是类似的. 
\begin{theorem}
    $F_{ X\wedge Y} = 1 - \left(1 - F_{X} \left(z\right) \right) \left(1 -F_{Y} \left(z\right)\right)$
\end{theorem}
\begin{proof}
\[
\begin{aligned}
    F _{X \wedge Y} \left(z\right) & = 1- P\left( X \wedge Y > z\right)\\
    & = 1- P\left( X >z \quad \& \quad Y > z \right) \\
    & = 1 - \left(1 - F_{X} \left(z\right) \right) \left(1 - F_{Y} \left(z\right)\right)\qedhere
\end{aligned}
\]
\end{proof}

\section{条件概率}
\subsection{第一部分}
先前我们已经知道了条件概率的一种, 即 $P\left(A\ | \ B\right)$, conditioning an event on an event.

现在我们介绍 conditioning a r.v. on an event, 以及相关的概率密度之计算, 分布函数之计算. 离散型的先不介绍, 因为这个东西理应介绍过了. 
\begin{definition}
    $\displaystyle P\left(X \in B \ | \ A\right) = \int  _{B} f _{X  |  A } \left(x\right) \ \mathrm{d}x$, 其中 $A$ 是一个事件. 所定义的 $f _{X  |  A}$ 就是我们想要的条件概率密度.
\end{definition}
实际上我们可以将事件 $A$ 换成是 $\mathbb{R}$ 上的一个borel集, 这样的概率密度记为 $f _{X| \left\{X \in A\right\}}$, 我们为了避免麻烦, 还是确保 $P \left(A\right) > 0$, 但是这个条件并不是必要的, 我们之后将会处理等于 $0$ 的情况. 

\begin{definition}
    我们给出密度的值
    \[
    f _{X | X \in A} = 
    \begin{cases}
        \frac{f \left(x\right)}{P\left(X \in A\right)}& , x \in A\\ 
        0& , x \notin A
    \end{cases}
    \]
\end{definition}
我们不难验证, $f _{X | X \in A}$ 确实是一个密度函数. 这需要我们验证其在 $\mathbb{R}$ 上的积分是否等于 $1$.

对此我们当然能够推广到随机向量上
\begin{definition}
    \[
    f _{X , Y | (X, Y  ) \in A} = 
    \begin{cases}
        \frac{f \left(x, y\right) }{ P\left( \left(X , Y\right) \in A   \right)} & , \left(x, y\right) \in A \\ 
        0 & , \left(x, y\right) \notin A 
    \end{cases} 
    \] 
    这里的 $A$ 当然是 $\mathbb{R} ^{2}$ 上的一个borel集.
\end{definition}
另一方面, 我们能够给出全概率公式的另一个版本
\begin{theorem}
    $A _{i}$ 是 $\Omega$ 的划分, 那么
    \[
    f _{X} \left(x\right) = \sum_{  i} ^{n} P\left(A _{i}\right) f_{X | A_i} \left(x\right) 
    \]
\end{theorem}
因为这里的划分只是有限个, 当然可以对这式子两边的密度函数进行任意的积分, 可以得到原本的那个全概率公式. \footnote{划分 $A_i$ 可以是无穷个吗?}
\subsection{另一个部分}
接下来是另一种条件概率, i.e. conditioning a r.v. on another r.v.
我们研究的是 $f _{X | Y} (x | y)$, 意思是给定 $Y= y$ 的时候, $X$ 的条件概率密度. 在这里我们将会去处理前面所落下的, 
``条件的概率是 $0$''的情况. 

\begin{definition}
    \[
    f _{X | Y} \left(x | y\right) = \frac{f _{X , Y} \left(x, y\right) }{ f_{Y} \left(y\right)}
    \]
    其中 $f_{Y} \left(y\right)$ 不为 $0$
\end{definition}
\begin{example}
    对于圆盘上的一个均匀分布的随机向量 $\left(X, Y\right)$, 我们想要计算出 $f _{X | Y} (x | y   )$, 
    首先得是算出 $f _{Y} \left(y\right)$, 我们将会看到, $f _{Y}$ 并不是常数, 而 $f _{X | Y}$ 将会.
\end{example}
\end{document}