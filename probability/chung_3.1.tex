\documentclass[a4paper, 10pt]{ctexart} %中文支持
\usepackage{float}              %防止浮动元素浮动
\usepackage{rotating}           %旋转图片用的
\usepackage[]{amsmath}          %数学公式
\usepackage{amsfonts}           %加载数学字体, 比如说\mathbb, 没有这个宏包的话可能会报错
\usepackage{amsthm}             %定义, 定理, 证明, 例子这些环境的支持
%使用方法:
%\newtheorem{environment name}{caption}
%比如 \newtheorem{example}{这是例子}
%效果 \begin{example} xxx \end{example} -> 这是例子 1 xxx
%proof就不需要了
\usepackage{graphicx}           %用来插入图片
\usepackage[left=1.25in,right=1.25in,top=1in,bottom=1in]{geometry}   %用来排版的
\usepackage[]{color}            %用来给部分文本上色的
\usepackage{algorithm}          %用来写伪代码的
\usepackage{algorithmic}        %同上
%\usepackage{minted}
\usepackage{extarrows}
%以下是宏包 amsthm 的命令, 我们使用这些环境的时候必须先对其进行一个定义

\newtheorem{theorem}{定理}
\newtheorem{example}{Example}
\newtheorem{definition}{Definition}
\newtheorem{lemma}{Lemma}

\begin{document}
\section{General definition}
\begin{definition}[随机变量之定义]
    给定一个概率空间 $ \left(\Omega , \mathcal{F} , P  \right)$, $X$ 是一个函数 $\Omega \to \mathbb{R}$, 并且有
    \[
    \forall B \in \mathcal B , X ^{-1} \left(B\right) \in \mathcal F 
    \]
    那么, 我们称 $X$ 是随机变量.
\end{definition}

我们还有另一种更加广义一点的定义, 涉及到了 trace 的定义. 但是我们不必管那么多. 但我仍会加入这个定义. 

\begin{theorem}[逆函数之性质]
    $X$ 是 $\Omega \to \mathbb{R}$ 的任意一个函数, 则有
    \[
    X ^{-1} \left( A ^{c}\right) = \left(X ^{-1}  \left(A\right)\right) ^{c}
    \]
    \[
    X ^{-1}  \left( \bigcup_{a \in \alpha} A_{a}\right) = \bigcup _{a \in \alpha} X ^{-1}  \left(A_{a}\right)
    \]
    \[
    X ^{-1}  \left( \bigcap _{a \in \alpha} A_{a} \right) = \bigcap _{a \in \alpha} X ^{-1}  \left( A_{a}\right)
    \]
    where $a \in \mathcal B$, 并且 $\alpha$ 是一个指标集, 可以不可数.
\end{theorem} 
这是非常重要的 基础的知识, 我们在学习拓扑的时候就有提到过. \footnote{但是证明就完全不会了捏, 尤其是不可数的情况...这鬼知道啊}
\begin{theorem}[随机变量的一个等价条件]
    X is a random variable $\iff$ $\forall x \in \mathbb{R}, X ^{-1}  \left( (- \infty , x] \right) \in \mathcal F$
\end{theorem}
\begin{proof}

    只证明``$\Rightarrow$''\\
    证明方法非常有意思, 具有一定的借鉴价值.
    只不过说, 这个方法在前面的章节已经学过了捏. \\
    设 $\mathcal S = \{S: X ^{-1} \left(S\right) \in \mathcal F\}$,  不难验证, $\mathcal S$ 是一个sigma代数. 

    $X ^{-1} \left(S\right) \in \mathcal F$, $X ^{-1}  \left(S ^{c}\right) = \left(X ^{-1}  \left(S\right)\right)^{c} \in \mathcal F$
于是说明, 集合 $\mathcal S$ 对于补运算封闭. 

    而后, $X ^{-1}  \left(\bigcup S_{i}\right) = \bigcup X ^{-1}  \left(S_i\right) \in \mathcal F$
    说明 $\mathcal S$ 对于可数并封闭. 并且 $\emptyset , \Omega$ 都在其中. 
    $ \left(- \infty , x \right] \in \mathcal S$, 则
    $\mathcal B \subset \mathcal S$ \footnote{这是因为 $\sigma \left\{ \left( -\infty , x\right] : x \in \mathbb{R}\right\} = \mathcal B$}, 于是就
    \[
    \forall  B \in \mathcal B , X ^{-1} \left( B\right) \in \mathcal F
    \qedhere
    \]
\end{proof}
\begin{theorem}
    $\left(\Omega , \mathcal F , P\right)$ induces  $\left( \mathbb{R} , \mathcal B , \mu\right)$
\end{theorem}
Note: 这个过程中存在信息的丢失, i.e. 该过程不一定可逆. 存在两个不同的随机变量, 他们能够诱导出相同的 $\mu$

一种逆, $\left\{ X ^{-1}  \left(B\right) : B \in \mathcal B \right\}$ 这称为是 $X$ 生成的sigma代数. 

现在我们知道 $\left( \Omega , \mathcal F , P\right) \to \left( \mathbb{R} , \mathcal B , \mu\right) \to F$
但上面的逆过程都是不成立的. 与此同时, 我们应该注意 $\mu \iff  F$ 是可以的. 
\begin{theorem}
    X is a r.v. 那么, $f \left(X\right)$ 也是随机变量, 其中 $f$ 是一个 Borel 可测函数. 
\end{theorem}
\begin{definition}[随机向量]
    函数 $\Omega \to \mathbb{R} ^{2}$, 其每一个分量都是随机变量, 则这个函数是一个随机向量.
\end{definition}
\begin{definition}
    2 维Borel集.
\end{definition}

\begin{theorem}
    $\left\{ X_j , j \ge 1\right\}$, 都是随机变量, 那么
    \[
    \inf_{j} X_j, \liminf _{j} X_j , \sup_{j}  X_j ,  \limsup_{j} X_j 
    \]
    都是随机变量. 
\end{theorem}
\begin{proof}
    首先从 $\sup_{j}  X_{j}$ 入手, 
    $\sup_{j} X_j  \left( \left( -\infty ,x \right]\right) = $
\end{proof}

\begin{definition}[离散随机变量之定义]
    X 是离散的, if $\exists B \subset \mathbb{R} , P\left( X \in B\right) =1$, where
    $B$ 是可数的.
\end{definition}
\begin{definition}[indicators]
    对于 $\Delta \subset \Omega$, 
    $1_{\Delta}$ 称为是 indicator functions if
    \[
    \forall \omega\in \Omega ,1_{\Delta} =
    \begin{cases}
        1 & , \omega \in \Delta \\
        0 & , otherwise.
    \end{cases}
    \]
\end{definition}

好了, 这节就结束了, 是不是有点迷糊, 其实我也很迷糊. 
但是接下来就是习题啦! 

TODO
\end{document}