\section{其它方法}
\subsection{简化牛顿法}
\begin{frame}
为了避免计算 \( f'(x)\), 因为有点难算, 于是我们用常数 \(C\) 代替, 也就是
\[
x _{k+1} = x_{k} - \frac{f(x_{k})}{C}
\]
一般我们取 \(C\) 为 \(f ' (x _{0})\).

\end{frame}
\subsection{牛顿下山法}
\begin{frame}
我们引入标量 \( \lambda\) 来优化性能, 也即
\[
x _{k+1} = x _{k} - \lambda \frac{f(x_{k})}{f'(x_{k})}
\]
\(\lambda\)的取值是我们定的, 虽然不知道怎么定, 但是这个方法能够优化性能, 可以使原来不收敛的方法变得收敛.
\end{frame}
\subsection{多点迭代法 弦截法}
% 定义 指标 和其他的方法的性能比较

\begin{frame}
弦截法是多点迭代法, 其中的 \( \varphi\) 函数有两个参数, 也就是
\[
x_{k+1} = \varphi(x_{k}, x_{k-1})
\]
弦截法的几何意义是, 将 \( x _{k}, x _{k-1}\) 连线, 这个直线的零点就是 \(x _{k+1}\). 此方法的收敛阶为
\[
\frac{1 + \sqrt 5}{2}
\]
\end{frame}
