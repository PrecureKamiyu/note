\documentclass[12pt]{ctexart}
\usepackage{amsmath}
\usepackage{amsthm}
\usepackage{amssymb}
\usepackage{amsfonts}
\usepackage{graphicx}
\usepackage{hyperref}
\usepackage{multirow}

\pagestyle{plain} 
\theoremstyle{definition}
\newtheorem{definition}{定义}[section]
\newtheorem{thm}[definition]{定理}

\theoremstyle{plain} 
\newtheorem{exam}[definition]{Example}

\begin{document}
\title{浮点数表示}
\author{你野爹}
\date{\today}
\maketitle
\tableofcontents

我们所讲的浮点数大致分为三个部分讲解, 第一个部分为小数的定点表示, 第二部分为小数的浮点表示, 第三个部分为规格化表示, 第四个部分是 IEEE 754 标准. 

\section{算术位移和逻辑位移}
\subsection{两种位移的定义}
逻辑位移和算术位移面对两种编码的位移操作. 逻辑位移适用于无符号数, 算术位移针对有符号数的. 其中逻辑位移还是很好懂的, 会丢失高位或者低位. 补上那一位一直是 \(0\). 

但是对于算术位移, 需要我们分情况讨论. 在这之前, 我们需要知道, 算术位移会被称为 ``算术'' 位移. 我们考察 \texttt{00000001}, 设其是一个原码表示的正数 \(1\), 其向左移动一位, 得到了 \texttt{00000010}. 代表的数字变为了 \(2\). 这就是说, \(1\) 变为了原来的两倍. 是的, 在一定情况下, 我们将一位数字向做移动 \(n\) 位, 则结果为这个数字乘以 \(2 ^{n}\). 算术位移就是说, 我们要将这个特性延续到补码或者是反码上面. 

\begin{figure}
\centering
\begin{tabular}{|l|c|c|}
\hline
真值                  & 码制                  & 填补                        \\ \hline
正数                  & 原码, 反码, 补码          & 0                       \\ \hline
\multirow{4}{*}{负数} & 原码                  & 0						   	\\ \cline{2-3}
                    & \multirow{2}{*}{补码} & 左移 0                      	\\ \cline{3-3}
                    &                     & \multicolumn{1}{l|}{右移 1} 	\\ \cline{2-3}
                    & 反码                  & 1                         	\\ \hline
\end{tabular}
\caption{算术位移表}\label{tab:suan}
\end{figure}

图~\ref{tab:suan} 将算术位移的各种情况列了出来. 有人可能问, 这保留的乘法的功能是为什么呢? 这是因为, 计算机内部, 基本的乘法就是通过位移实现的. 对于一个整数 \(a\), 我们想要计算 \(7 \times a\), 那么我们实际上的操作为 \(7 \times a =  8 \times a - a \), 即为 \texttt{a<<3 - a}. 

\subsection{两种位移的硬件实现}
% TODO
乐, 其实挺简单的. 
算术位移, 对于正数补码, 高位不应该丢1, 对于负数补码, 高位不应该丢0. 否则会发生错误.
这个错误是指, 其结果并不是原本的数字乘以 \(2\) . 比如说, 对于 \texttt{10000000}, 进行了一个逻辑左移. 
那么结果\texttt{00000000}, 结果当然不是 \(2 ^{7} \times 2\). 

\section{小数的定点表示}\label{sec:ding}
我们之前已经学习了整数的表示方法, 这里我们介绍一种并不是很实用的, 表示小数的一个方法, 称为是小数的定点表示. 这里我们可以先复习一下原码, 补码和移码. 
\subsection{原码, 补码, 移码}
我们接下来表示的数字都是默认有符号的, 对于原码来说, 需要提供一位 bit 出来当作是符号位. 这个表示方法的特点便是, \(0\) 有两种表示方式, 因为 \(+0\) 和 \(-0\) 都是一样, 并且此等表示方式, 能够表出的正数和负数的个数是一致的. 这个是其显著特征. 
那么说, 对于一个 \(n\) 位的原码表示, 其表示范围为 \(2^{n-1} - 1\) 到 \(-(2 ^{n-1} - 1)\).

补码的定义为: 将所有的 bit 置反, 随后 \(+1\). 我们可能会觉得这个是什么几把, 为什么要\(+1\)? \(+1\) 的作用其实仅仅是对齐而已, 比如说 \texttt{00000000} 取反之后为 \texttt{11111111}, 我总不能用这个数字表示 \(0\) 吧, 于是就设置了一个偏移量, \(+1\) 使得这个数字变为了 \texttt{00000000}, 就是零了. \(-0\) 的原码表示被映射为了 \(-2 ^{n}\), 这样, 负数的表示范围就比正数要大 \(1\) 了. 此特点将补码和原码显著的分开来了. 
% TODO 为什么补码是自然的? 这点我还不知道. 但是我记得有人进行了解释. 只不过我忘记了.  

随后, 我们给出补码的表示公式, 记补码长这个样子: \(b_{n-1}b_{n-2}\dots b_{2}b_{1}b_{0}\), 记真值为 \(N\): 
\begin{equation}
N = 
\begin{cases}
b_{n-2}b_{n-3}\cdots b_{0} &, \text{ if } b_{n-1} = 0\\
2 ^{n-1} - b _{n-2} b_{n-3} \cdots b_{0} &, \text{ if } b_{n-1} = 1
\end{cases}
\end{equation}
当真值为正数的时候, 其表示和原码一样. 
% TODO 剩下将补码的定义讲完.和移码的定义讲完.
\subsection{定点表示}
定点表示的格式如下 
\begin{equation}
	b_{0}.b_{1}b_{2}b_{3}
\end{equation}
需要注意这其中的小数点, 并且我们从 \(0\) 开始计数, \(b_{0}\) 是符号位. 对于正数其表示公式为: 
\begin{equation}
	N = \sum_{i = 1 } ^{n-1} b_{i} \times 2^{-i}
\end{equation}
小数的定点表示可以表示出一个绝对值小于 \(1\) 的小数. 

对于定点表示, 也分为补码和原码两种. 对于原码, \texttt{0.111} 就是最大正数, 真值为 \(1 - 2^{-3}\), 一般化表示即为 \(1- 2^{-(n-1)}\); 最小负数为 \texttt{1.111}, 为 \(- (1- 2 ^{ -3})\). 剩下类似. 对于补码, 最大正数和原码一致, 但是最小负数为 \texttt{1.000}, 真值为 \(-1\). 

\section{小数的浮点表示}
浮点表示类似于科学计数法, 由两个部分组成: 1. 指数部分; 2. 尾数部分. 注意到, 尾数部分是定点表示的, 指数部分则不是. 如下:
\begin{equation}
j_{f}j_{1}j_{2}\dots j_{m}S_{f}S_{1}S_{2}\dots S_{n}
\end{equation}
其中 \(j_{f}, S_{f}\) 分别是阶码和尾数的符号位. 这个时候我们就可以开始分析, 在原码表示和补码表示之下, 表示范围分别是多少. 我们只考虑补码:

\begin{description}
\item [1] 指数部分---也称阶码---的最大值为: \(2^{m} -1 \), 最小值为 \(- 2 ^{m}\) 
\item [2] 尾数部分, 最大正数为 \(1-2^{n}\) , 最小正值为 \(2^{n}\), 对应表示分别为 \texttt{0.111}, \texttt{0.001} (以四位表示为例, \(n=3\))
\item [3] 尾数部分, 最大负数为 \(-2 ^{n}\), 最小负数为 \(-1\), 对应表示分别为 \texttt{1.111}, \texttt{1.000}.
\item [4] 于是表示范围为 \( [- 2 ^{2^{m}- 1} \times 1 , 2 ^{2^{m} -1} \times (1 - 2 ^{n} )]\)
\end{description}
\subsection{上溢和下溢}
准确地来说, 我们浮点数的表示区间为 \([\text{最小负数}, \text{最大负数}] \cup \{0\} \cup [\text{最小正数}, \text{最大正数} ]\). 如果说绝对值太大了, 超过了表示范围, 这样的情况称为是上溢出; 类似地, 绝对值太小, 浮点数的精度无法表示, 这样的情况称为下溢. 


\section{规格化表示}
对于不同的底数, 规格化有些许不同, 但是总体差不多, 我们只介绍底数为2的规格化.

我们首先介绍原码的, 原码和补码这两种情况还有区分. 
\subsection{原码的规格化}
对于原码, 尾数必须为 \texttt{0.1XX\dots XX} 的这种形式. 当尾数并不满足这种形式的时候, 需要进行移位, 移位分为左规和右规:
\begin{itemize}
\item [1] 左规: 尾数算数左移 (后面补上0), 阶码减一
\item [2] 右规: 当浮点数加法之中出现了溢出, 将尾数右移一位, 阶码加一
\end{itemize}

\begin{exam}\quad\\ 
左规:
\[
b = 2 ^{1} \times (+ 0.01001) \implies b = 2 ^{0} \times (+ 0.1 0010)
\]
右规: \(a = 2^{2} \times (00.1100) , b = 2^{2	} \times 00.1000\)
\[
\begin{aligned}
a + b & = 2 ^{2} \times ( 00. 1100 + 00.1000) \\ 
& =  2^{2} \times 01.0100 \\ 
&  = 2 ^{3} \times 00.1010 = 2 ^{3} \times 0.1010
\end{aligned}
\]
注意到, 使用两个bit表示符号是为了方便. 
\end{exam}

\subsection{补码的规格化}
\begin{itemize}
\item [1] 对于负数, 其补码形式应为 \texttt{1.0XXX\dots} 
\item [2] 对于正数, 其补码形式应为 \texttt{0.1XXX\dots}
\end{itemize}
左规右规所用的位移应为算术位移.

我们想要知道这种情况之下, 规格化数的表示范围. 我们分为正数和负数两个部分讨论. 只考察尾数部分. 
最大正数为 \(1 - 2 ^{-n}\) , 对应编码为 \texttt{0.1111}; 最小正数为 \(1/2\) , 对应编码为 \texttt{0.100}. 至于负数, 我们先考虑其原码, 再看转换的补码是否符合要求. 最小负数是比较好找的, \(-1\) , 对应编码的补码编码为 \texttt{1.000}; 最大负数则为 \(- (1/2 + 2 ^{-n})\), 我们来看看为什么. 

考虑原码为 \texttt{1.100} 的负数, 其值为 \(-1/2\), 绝对值刚好等于最小正数. 可是其补码却不是规格化数: \texttt{1.100}---补码是其本身. \texttt{1.100} 小数点后取反的话, 便是 \texttt{1.011} 如果将其看作是补码, 那么这确实是最大负数, 可是反码到补码还需要加一, 这就使得, 我们原码也要加一. 所以说, 最大负数为 \(1.011\) 对应的值, 也就是 \(-(1/2 + 2^{-n})\). 

上面的解释是说为什么最大负数不是 \(-1/2\), 我们是容易知道 \(-(1/2 + 2 ^{-n})\) 为最大负数的. 正是补码的 \(+1\) 导致了这一偏移, 偏移长度为一个最小单位: \(2 ^{-n}\). 
\subsection{规格化的目的}
当我们构想规格化的目的的时候, 我们应该能想到其中一个目的: 使得一个规格数的表示是唯一的. 这着实是理由之一. 除此之外, 规格化数的精度是相同的. 我们确保了规格数的精度一定是24位有效数字. 此二者或是规格化的目的吧. 
\subsection{浮点数溢出和机器0}
机器0 其实没什么好说的, 就是当且仅当尾数和阶码均为 \(0\) 的时候, 这个浮点数才表示 \(0\).

\section{IEEE 754 标准}
IEEE 754 浮点数表示标准是由 IEEE 电气电子工程师学会规定出的, 双精度或者单精度浮点数表示的规范. 
\subsection{构成}
对于单精度浮点, 32-bit, 构成如下
\begin{equation}
s\ \overbrace{e_{7}\dots e_{0}}^{\text{8-bit}}\ \underbrace{w_{22}\dots w_{0}} _{\text{23-bit}}
\end{equation}
\(s\) 为符号位, 而 \(e_{i}\) 为无符号指数, \(w_{i}\) 为无符号尾数. 说实话和我们上面学的几把东西没什么关联. 随后我们需要注意到, 虽然说 \(e_{i}\) 是无符号数, 但是, 我们之后会让其减去一个 bias 让其能够表示负数. bias 在不同情况下是不同的, 所以我们现在不能武断地说 \(e_{i}\) 是移码表示的, 我们后面将会看到其和移码表示差不多, 仅仅在一小点地方有差别. 

尾数的表示也分情况. IEEE 为了让其最大化表示精度, 设定了一些规定. 我们需要知道: 分为规格化数 (浮点数) 和非规格化数进行讨论.
\subsection{尾数}
IEEE 规定, 在规格化数表示的时候, 尾数默认省略了一个前置的 1, 就是说真实的尾数是 \(1+W\footnote{设大写字母代表其真值}\). 

在非规格化数表示的时候, 就默认没有前置的 \(1\), 真实的尾数就是 \(W\). 原因也很简单, 当我们表示的数字特别小的时候, 小于了 \(1\times 2 ^{\text{最小指数}}\) 的时候, 我们别无选择, 只能够消去前面这个默认的 \(1\). 
\subsection{指数}
指数是有 bias 的, 这使得, \(E\) 为 \(0\) 的时候, 其表示的数是最小的, 是最小负数. 在规格化数那里, 这个 bias
是 \(-127\)\footnote{但同时, \(E\) 的取值范围为 \([1, 254]\). \(0, 255\) 都用来表述特殊值, 其中 \(E = 0\) 的时候, 说明表示的是非规格化数.}\footnote{为什么 bias 是 \(-127\)捏? 移码是 \(-128\). 一般的解释是, \(-127\) 的表示范围更大. },  考虑规格化数, 真实指数的最小值---\(1- 127  = -126\). 若是再继续往下, 要让表示的数字更小, 就会走到非规格化数, 可是由于非规格化数的没有前置1, 所以说, 真实指数的值不应该发生变化, 否则表述的区间就不再连续, 故真实指数为 \(-126\), 所以这个 bias 变为了 \(-126\), 因为 \(0 - 126 = -126\). 

因此我们知道, 对于规格化数, 我们的转换公式如下: 
\begin{equation}
N = (-1) ^{s} \times 2 ^{E - 127} \times (1 + W \times (2 ^{-23} ))
\end{equation}
对于非规格化数, 则有
\begin{equation}
N = (-1) ^{s} \times 2 ^{E - 126} \times (0 + W \times (2 ^{-23}))
\end{equation}
其中, \(E - 126 = 0\)

\subsection{特殊值}
\(E = 255\) 的时候, 浮点数表示特殊值. 
\begin{description}
\item [当 \(E  = 255 ,W = 0\) 时候] 浮点数表示 \(\infty\).  
\item [当 \(E = 255, W\) 为非零值的时候] 浮点数表示 NaN (Not a Number).
\end{description}


\begin{exam}
试着将 \(-0.75\) 转化为 \verb|float| 型的编码
\end{exam}


\section{IEEE Rounding}
我们在进行运算的时候, 有的时候需要进行 rounding, 比如说我们将一个 int 转换为浮点数表示. 
\verb|float| 的精度为小数点后 23 位, 如果说我们要将一个 24 位以上的 \verb|int| 转换为 \verb|float|, 
那么精度就不满足了, 于是就要进行 rounding. 

IEEE 的 rounding 和我们生活中进行的四舍五入差不多, 但是有一个区别, 就是当要舍去的那位, 恰好为 5 的时候. 
为了方便讨论, 我们说 \(G\) 代表被舍入位的前一位, 称为 grounding number; \(R\) 为被舍去位, rounding number; \(S\) 为 \(R\) 之后的所有位的 \verb|or| 值. 当然, 这里考虑的是二进制. 

IEEE 规定的 rounding 只有一个特殊情况, 便是 ``向偶数进位''. 这是说, 当 \(G\) 后面所接的数字, 刚好是 \(1/2\) 的时候, \(G\) 应当向偶数进位, 就是说, \(G = 0\) 的时候, 不动; \(G = 1\) 的时候向上进位, 也就是 \(+1\).  

\begin{exam}
\verb|1.100| 进位, 结果为 \verb|10|; \verb|0.100| 不进位, 结果为 \verb|0|. 
\end{exam}
不难验证, 进位的条件判断为 \(R \land (G \lor S)\). 如果你很闲的话, 可以尝试自己写个程序, 将一个整数转化为浮点数. 
\end{document}
