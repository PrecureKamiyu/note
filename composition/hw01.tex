\documentclass[12pt]{ctexart}
\usepackage{amsmath}
\usepackage{amsthm}
\usepackage{amssymb}
\usepackage{amsfonts}
\usepackage{graphicx}
\usepackage{bookmark}
\usepackage{hyperref}

\theoremstyle{definition}
\newtheorem{theorem}{Def}[section]
\newtheorem{lemma}[theorem]{Lemma}
\theoremstyle{definition}
\newtheorem{definition}{定义}[section]
\newtheorem{thm}[definition]{定理}

\theoremstyle{plain} 
\newtheorem{exam}[definition]{Example}
\theoremstyle{remark}
\newtheorem{remark}[definition]{Remark}

\begin{document}
\setlength{\hangindent}{30pt}
\noindent
\textbf{1.1. } 寄存器里存有 \texttt{111111111}, 其真值为 \(-1\), 则该机器数是什么表示?

\setlength{\hangindent}{30pt}
\noindent
\textbf{1.2. } 寄存器内内容为 \texttt{80H}, 其真值为 \(-0\), 则该机器数是什么表示?

\setlength{\hangindent}{30pt}
\noindent
\textbf{1.3. } 浮点机之中, 若使机器零全为 \(0\), 那么阶码应为什么表示? 

\setlength{\hangindent}{30pt}
\noindent
\textbf{2.1. } 十进制数 \(-0.875\) 用 IEEE~754 \verb|float| 表示为?\\
首先其为规格数, 符号位为 \(1\), 随后是阶码 \(-1\), 即 \texttt{126} 对应表示. 尾数为 \(1 - 0.125 = 1 - 2 ^{-3}\). 
即, \texttt{1.111}. \\
故表示为 \texttt{1.01111110.11100000000000000000000}
稍微划分一下: \\ \texttt{1011.1111.0110.000000000000000\dots}. 即: \texttt{BF600000H}

\setlength{\hangindent}{30pt}
\noindent
\textbf{2.2. } 十进制数 \(-754\) 用 IEEE~754 \verb|float| 表示为? 
\(s = 1\), 阶码为 \(9\), 尾数为 \(512 + 128 +  64 +  32 + 16 + 2\) 即为 \texttt{1011110010}. 
即为 
\texttt{1.10001000.01111001000000\dots}, \texttt{1100.0100.0011.1100.1000.0000000000}, \texttt{C43C8000H}


\setlength{\hangindent}{30pt}
\noindent
\textbf{2.3. } 给定 \verb|float|: \verb|C4514000H|, 该数的十进制表示为? 
\texttt{1100.0100.0101.0001.0100.0000.0000.0000}
\(s = 1\), 阶码为 \texttt{1000.1000}, 为 \(9\), 随后, 尾数为 \texttt{1.1010.0010.1000.0000}, 为 \(1.625 + 2 ^{-7} + 2 ^{-9} = 1.634765625\). 

于是表示应为 \((-1) ^{1} \times 1.634765625 \times 2 ^{9}= -837\)


\setlength{\hangindent}{30pt}
\noindent
\textbf{4.1.} \ (1) 最大正数为 \(2 ^{2 ^{9} - 1} \times (1 - 2 ^{-21})\)  \\
(2) \(2 ^{- 2 ^{9}} \times (1/2)\) \\
(3) \(-2 ^{2 ^{9} - 1} \times (1)\)\\
(4) \(-2 ^{ - 2^{9} } \times (1/2 + 2 ^{-21})\)

\setlength{\hangindent}{30pt}
\noindent
\textbf{4.2.} \ (1) 阶码: \(5\); 尾数 \(32 - 2 - 5 = 25\) \\ 
(2) 设原码表示. 上溢: 绝对值大于 \(2 ^{32 - 1} \times (1 - 2 ^{-25})\); 下溢: 绝对值小于 \(2 ^{-31} \times (2 ^{-25})\)

\setlength{\hangindent}{30pt}
\noindent
\textbf{4.3.} \
(1) 
\texttt{10001.11001010000}\\
(2) \texttt{11111.10110110000}\\
(3) \texttt{01111.10110110000}
\end{document}
