\documentclass[12pt]{ctexart}
\usepackage{amsmath}
\usepackage{amsthm}
\usepackage{amssymb}
\usepackage{amsfonts}
\usepackage{graphicx}
\usepackage{hyperref}

\pagestyle{plain}
\theoremstyle{definition}
\newtheorem{theorem}{Def}[section]
\newtheorem{lemma}[theorem]{Lemma}
\theoremstyle{definition}
\newtheorem{definition}{定义}[section]
\newtheorem{thm}[definition]{定理}

\theoremstyle{plain} 
\newtheorem{exam}[defition]{Example}

\begin{document}
\section{RISC-V 介绍}

- 机器指令 
- 指令集 IS instruction set
- 指令集架构 ISA instruction set architecture 
也称为处理器架构

- 系列机, 基本指令系统结构相同的计算机? 
- ? 

ISA 
讲的什么几把. 

属性 
位宽, 指定了通用寄存器的宽度. 决定了寻址范围的大小, 数据运算能力的强弱. 
需要和 指令编码长度分开, 指令的编码长度是越小越好的. 

原则:
1. 简单性来自于规则性
2. 越小越快
3. 加速经常性时间.  经常使用的指令尽量短. 
4. 需要良好的折衷.

性能要求: 
1. 完备性: 指令丰富, 功能齐全, 使用方便
2. 高效性: 空间小, 速度快
3. 规整性: riscv 比较规整, 但是 x86 相反. 
4. 兼容性: 能够向上兼容
 
\subsection{}

content

- 储存器寻址
- 操作出的类型
- 控制转移类指令
- 指令格式

- 大端法和小端法
字地址, 字长度为 32 位 (riscv), 字地址为 addr.
小端法: 将字的低位放在 addr. 次低位放在 addr + 1.
大端法: 将字的高位放在 addr. 

- 对齐问题

- 寻址方式
-- 寄存器寻址
直接使用寄存器的名字
立即数寻址
间接访问, 访问寄存器的值, 根据寄存器的值访问寄存器

\end{document}
