\documentclass[a4paper, 10pt]{ctexart} %中文支持
\usepackage{float}              %防止浮动元素浮动
\usepackage{rotating}           %旋转图片用的
\usepackage[]{amsmath}          %数学公式
\usepackage{amsfonts}           %加载数学字体, 比如说\mathbb, 没有这个宏包的话可能会报错
\usepackage{amsthm}             %定义, 定理, 证明, 例子这些环境的支持
\usepackage[]{amsmath}
%使用方法:
%\newtheorem{environment name}{caption}
%比如 \newtheorem{example}{这是例子}
%效果 \begin{example} xxx \end{example} -> 这是例子 1 xxx
%proof就不需要了
\usepackage{graphicx}           %用来插入图片
\usepackage[left=1.25in,right=1.25in,top=1in,bottom=1in]{geometry}   %用来排版的
\usepackage[]{color}            %用来给部分文本上色的
\usepackage{algorithm}          %用来写伪代码的
\usepackage{algorithmic}        %同上
\usepackage{minted}
\usemintedstyle{vs}
\usepackage{extarrows}
\usepackage[]{hyperref}
%以下是宏包 amsthm 的命令, 我们使用这些环境的时候必须先对其进行一个定义
\newtheorem{theorem}{定理}
\newtheorem{example}{Example}
\newtheorem{definition}{Definition}
\newtheorem{lemma}{Lemma}
\title{算法设计与分析 第5章}
\author{毛翰翔 \and 210110531}
\begin{document}
\maketitle
1、(30分)假定我们不再一直选择最早结束的活动,而是选择最晚开始的活动,前提仍然是与之前选出的所有活动兼容。
描述如何利用这一方法设计贪心算法,并证明算法会产生最优解。

\vspace*{4ex}
$\mathbf{Solution}$: 类似地, 我们将活动按照开始时间从小到大进行排列, 并从最后一个活动开始选择. 剩下的引理其实完全类似. 记号沿用课内介绍的. 
设 $S = \left\{A_{i}:A_{i} = \left(s_{i}, f_{i}\right), s_i \le s_j \text{ if } i < j\right\}$, $A$ for activity, $s$ for start, $f$ for fin.

优化解一定包括了最后一个活动, 记为 $A_{n}$ , 对于一个没有最后一个活动的优化解, 我们将其最晚开始的活动换为 $A_{n}$ , 这个解仍是优化解. 

给定一个包含了最后一个活动的优化解, 将最后这个活动去掉, 所剩下的正是 $S$ 的子集 $S' = \left\{A_i : f_i < s_n \right\}$ 的优化解. 

这上面两点已经足够说明贪心选择性, 这点只需要归纳法就足以证明, 我们每一次加进来的活动都满足: 存在优化解包含了它. 那么到最后我们所加入的所有活动构成了优化解. 

\vspace*{4ex}
2、(30分)考虑用最少的硬币找n美分零钱的问题。假定每种硬币的面额都是整数。

a.设计贪心算法求解找零问题。假定有25美分、10美分、5美分和1美分4种面额的硬币。

b.设计一组硬币面额,使得贪心算法不能保证的到最优解。这组硬币面额中应该包含1美分,使得对每个零钱值都存在找零方案。

\vspace*{4ex}
\noindent $\mathbf{Solution}$: \\a: 对于给定的 $n$ 美分, 我们从最大面额的硬币开始选择, 如果说 $n \ge 25$ 那么我们就找 $\left\lfloor n / 25 \right\rfloor$ 枚 $25$ 美分的硬币. 
对于剩下的 $n \text{ mod } 25$ , 类似地, 能够选取多少大面额的硬币就选取多少, 以此类推.
因为能够选择 $1$ 美分的硬币, 所以一定能够找零. 

对于找零问题的一个优化解, 如果 $n \ge 25$ 则, $25$ 美分的硬币的数量一定是 $\left\lfloor n / 25 \right\rfloor$. 因为 $25$ 能够被 $10 , 5$ 找零. 
\\
b: $1 , 5, 8$ \\不能通过贪心算法得到最优解. 比如说给定 $10$, 按照贪心算法, 解应该是 $8, 1, 1$, 但是最优解是 $5  , 5$

\vspace*{4ex}
3、(40分)编程题:柠檬水找零
题目描述:
在柠檬水摊上,每一杯柠檬水的售价为 5 美元。
顾客排队购买你的产品,(按账单 bills 支付的顺序)一次购买一杯。
每位顾客只买一杯柠檬水,然后向你付 5 美元、10 美元或 20 美元。
你必须给每个顾客正确找零,
也就是说净交易是每位顾客向你支付 5 美元。
注意,一开始你手头没有任何零钱。
如果你能给每位顾客正确找零,返回 true ,否则返回 false 。
提示:
0 <= bills.length <= 10000
bills[i] 不是 5 就是 10 或是 20


\begin{minted}[mathescape,
                   linenos, 
                   numbersep=5pt,
                   gobble=2,
                   frame=lines,
                   framesep=2mm]{c++}
    #include <stdio.h>
    int coins (){
        int five = 0, ten = 0;
        // initialization
        int n = 0, current; 
        // n for the length of bills
        scanf("%d", &n);
        for (int i = 0 ; i < n ; i++){
            scanf("%d", &current);
            switch (current){
            case 5:
                five++; break;
            case 10:
                ten++;  five = five -2;
                if (five < 0) 
                    return 0 ;
                break;
            case 20:
                if (five - 3 < 0 && (ten - 1 < 0 || five - 1 < 0))
                    return 0 ;
                else if (ten -1 < 0 || five -1 < 0)
                    five = five - 3;
                else {
                    five--;
                    ten--;
                }
                break;
            default:
                break;
            }
        }
        return 1;
    }

    int main (){
        int ans = coins();
        if (ans)
            printf("true\n");
        else 
            printf("false\n");
        return 0;
    }
\end{minted}
\end{document}
