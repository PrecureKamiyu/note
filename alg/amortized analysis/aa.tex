\documentclass[a4paper, 10pt]{ctexbook} %中文支持
\usepackage{float}              %防止浮动元素浮动
\usepackage{rotating}           %旋转图片
\usepackage{hyperref}           %生成可跳转的书签
\usepackage{amsfonts}           %对某一些字体之支持
\usepackage[]{amsmath}          %数学公式
\usepackage{amsthm}             %定义, 定理, 证明, 例子环境的支持
%使用方法:
%\newtheorem{environment name}{caption}
%比如 \newtheorem{example}{这是例子}
%效果 \begin{example} xxx \end{example} -> 这是例子 1 xxx
%proof就不需要了
\usepackage{graphicx}           %插入图片
\usepackage[left=1.25in,right=1.25in,top=1in,bottom=1in]{geometry}   %用来排版的
\usepackage[]{color}            %给部分文本上色的
\usepackage{algorithm}          %写伪代码的
\usepackage{algorithmic}        %同上
\usepackage{minted}
\usepackage{amssymb}            %用来加入一些数学符号, 比如说 $\varnothing$
\usepackage{fontspec}           %不知道用来干嘛的
\setmonofont{Ubuntu Mono}       %?
\usemintedstyle{custommanni}    %设置minted插入代码的风格

\newtheorem{theorem}{定理}
\newtheorem{example}{Example}
\newtheorem{definition}{定义}
\newtheorem{lemma}{引理}
\newtheorem{proposition}{命题}
\begin{document}
\tableofcontents
In an amortized analysis, we average the time required to perform a sequence of data-structure operations over all the operations performed. With amortized analy- sis, we can show that the average cost of an operation is small, if we average over a sequence of operations, even though a single operation within the sequence might be expensive. Amortized analysis differs from average-case analysis in that prob- ability is not involved; an amortized analysis guarantees the average performance of each operation in the worst case.  The first three sections of this chapter cover the three most common techniques used in amortized analysis. Section 17.1 starts with aggregate analysis, in which we determine an upper bound T .n/ on the total cost of a sequence of n operations.  The average cost per operation is then T .n/=n. We take the average cost as the amortized cost of each operation, so that all operations have the same amortized cost.  Section 17.2 covers the accounting method, in which we determine an amortized cost of each operation. When there is more than one type of operation, each type of operation may have a different amortized cost. The accounting method overcharges some operations early in the sequence, storing the overcharge as “prepaid credit” on specific objects in the data structure. Later in the sequence, the credit pays for operations that are charged less than they actually cost.  Section 17.3 discusses the potential method, which is like the accounting method in that we determine the amortized cost of each operation and may overcharge op- erations early on to compensate for undercharges later. The potential method main- tains the credit as the “potential energy” of the data structure as a whole instead of associating the credit with individual objects within the data structure.  We shall use two examples to examine these three methods. One is a stack with the additional operation MULTIPOP, which pops several objects at once. The other is a binary counter that counts up from 0 by means of the single operation INCREMENT.  
\end{document}