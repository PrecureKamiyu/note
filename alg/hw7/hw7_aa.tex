\documentclass[a4paper, 10pt]{ctexart} %中文支持
\usepackage{float}              %防止浮动元素浮动
\usepackage{rotating}           %旋转图片
\usepackage{hyperref}           %生成可跳转的书签
\usepackage{amsfonts}           %对某一些字体之支持
\usepackage[]{amsmath}          %数学公式
\usepackage{amsthm}             %定义, 定理, 证明, 例子环境的支持
%使用方法:
%\newtheorem{environment name}{caption}
%比如 \newtheorem{example}{这是例子}
%效果 \begin{example} xxx \end{example} -> 这是例子 1 xxx
%proof就不需要了
\usepackage{graphicx}           %插入图片
\usepackage[left=1.25in,right=1.25in,top=1in,bottom=1in]{geometry}   %用来排版的
\usepackage[]{color}            %给部分文本上色的
\usepackage{algorithm}          %写伪代码的
\usepackage{algorithmic}        %同上
\usepackage{minted}
\usepackage{amssymb}            %用来加入一些数学符号, 比如说 $\varnothing$
\usepackage{fontspec}           %不知道用来干嘛的
\setmonofont{Ubuntu Mono}       %?
\usemintedstyle{custommanni}    %设置minted插入代码的风格

\newtheorem{theorem}{定理}
\newtheorem{example}{Example}
\newtheorem{definition}{定义}
\newtheorem{lemma}{引理}
\newtheorem{proposition}{命题}
\title{算法设计与分析第七章作业}
\author{毛翰翔\and 210110531}
\begin{document}
\maketitle

\paragraph{1. Solution} % (fold)
\label{par:what is this}
\textbf{聚集法: }
\\ [7pt]
对于 $n$ 次操作, 其中开销不为 $1$ 的操作次数为 $\displaystyle \left\lfloor \log _{2}n \right\rfloor$, 这些操作的开销即为
\[
\begin{aligned}
\sum_{i=1} ^{\left\lfloor \log n \right\rfloor} 2 ^{i}  \le \sum_{i=0}  ^{\left\lfloor \log _{2} n \right\rfloor} 2 ^{i}\le \frac{2 ^{\left\lfloor \log _{2} n \right\rfloor}-1}{2-1} \le n
\end{aligned}
\]
那么总的开销为开销不为 $1$ 的加上开销为 $1$ 的部分, 其中开销为 $1$ 的数量小于 $n$
\[
\sum_{i=1} ^{n} c_{i} \le n + n =2n 
\]
即, $T\left(n\right) = O\left(n\right)$, 那么 $T\left(n \right)/n = O\left(1\right)$
\\ [7pt]
\textbf{会计法: }
开销为 $1$ 的操作的摊还代价为 $3$, 开销不为 $1$ 的操作的摊还代价为 $0$. 每次进行非 $1$ 的操作时, 设这次操作是第 $i$ 次, 拿走前面 $i -  i / 2$ 次操作所给出的credits. 而每一次 $1$ 操作会给出 $2$ 的credits. 则非 $1$ 操作的实际开销为 
$$ \frac{i}{2} \cdot 2 + 0 = i$$
和题意相符, 于是有 $T\left(n\right) \le 3n$, 即 $T\left(n\right) = O\left(n\right)$
\\ [7pt]
\textbf{势能法: }
设当前操作数为 $n$ , 设 $\varPhi\left(D\right) =2\left[ n - 2^{\left\lfloor \log _{2  } n \right\rfloor}\right]$, 类似的, $1$ 操作的摊还代价为
\[
\hat{c} = 1 + \varPhi \left(D_{n+1}\right) - \varPhi \left(D_{n}\right) = 3
\] 
非 $1$ 操作的摊还开销为 
\[
\hat{c} =  i + \varPhi \left(D_{n+1}\right)  - \varPhi \left(D_{n}\right)= 0
\]
那么 $T \left(n\right) = O\left(n\right)$, 并且 $T \left(n \right) /n  = O\left(1\right)$


% paragraph what is this (end)
\paragraph{2. Solution} % (fold)
\label{par:}
\textbf{聚集法: }
当进行了 $n$ 次 \verb|flip_push| 之后, 其中的反转操作次数为 $\left\lfloor \log _{2  } n \right\rfloor$, 其中每一次反转的操作的代价是 $i$, 设反转是 $i$ 次操作
类似的, 我们有
\[
\sum_{k=1} ^{\left\lfloor  \log _{2} n  \right\rfloor}  2 ^{k} \le   n
\]
那么, 总的代价为 
\[
\sum_{i=1}^{n} c_{i} = \sum_{k=1}  ^{\left\lfloor  \log _{2}n \right\rfloor} 2 ^{k} +  \left(n -  \left\lfloor  \log _{2} n \right\rfloor\right) \le 2n
\]
即 $T\left(n\right) /n  = O\left(1\right)$
\\ [7pt]
\textbf{会计法: }
设每次进栈的操作的摊还代价是 $3$, 而反转操作的摊还代价是 $0$ , 那么每一个栈元素上的credits是 $2$. 
反转的摊还代价为 $0$. 每次进栈时, 设这次是第 $i$ 次, 拿走前面 $i -  i / 2$ 个栈元素上的credits. 则反转的实际开销为 
$$ \frac{i}{2} \cdot 2 + 0 = i$$
和题意相符, 于是有 $T\left(n\right) \le 3n$, 即 $T\left(n\right) = O\left(n\right)$
\\ [7pt]
\textbf{聚能法: }
设聚能函数 $\varPhi: S \mapsto \varPhi ( S  ) = 2 \left| S \right| $, 其中 $\left| S \right| $ 代表 $S$ 中没有被反转过的元素个数. 
那么进栈操作的摊还代价是 
\[
\hat{c} = 1 + \varPhi (S_{n+1}) - \varPhi\left(S_{n}\right) = 3
\]
反转操作的摊还代价为 
\[
\hat{c} = i + \varPhi (S_{n+1}) - \varPhi \left(S_{n}\right)  = i + 2 \left(  0  - \frac{i}{2}\right)  = 0
\]
那么有 $T\left(n\right) = O\left(n\right)$
% paragraph  (end) 



\end{document}