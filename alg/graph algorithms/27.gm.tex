\documentclass[a4paper, 10pt]{ctexart} %中文支持
\usepackage{float}              %防止浮动元素浮动
\usepackage{rotating}           %旋转图片
\usepackage{hyperref}           %生成可跳转的书签
\usepackage{amsfonts}           %对某一些字体之支持
\usepackage[]{amsmath}          %数学公式
\usepackage{amsthm}             %定义, 定理, 证明, 例子环境的支持
%使用方法:
%\newtheorem{environment name}{caption}
%比如 \newtheorem{example}{这是例子}
%效果 \begin{example} xxx \end{example} -> 这是例子 1 xxx
%proof就不需要了
\usepackage{graphicx}           %插入图片
\usepackage[left=1.25in,right=1.25in,top=1in,bottom=1in]{geometry}   %用来排版的
\usepackage[]{color}            %给部分文本上色的
\usepackage{algorithm}          %写伪代码的
\usepackage{algorithmic}        %同上
\usepackage{minted}
\usepackage{amssymb}            %用来加入一些数学符号, 比如说 $\varnothing$
\usepackage{titlesec}
\usepackage{fontspec}           %不知道用来干嘛的
%%%%%%%%%%%%%%%%%%%%%%%%%%%%%%%%%%%%%
\setmonofont{Ubuntu Mono}       %?
\usemintedstyle{custommanni}    %设置minted插入代码的风格
\titleformat*{\section}{\huge\bfseries}             %管理title的字体和大小
\titleformat*{\subsection}{\Large\bfseries}         %bfseries就是默认的字体.
\titleformat*{\subsubsection}{\large\bfseries}      % 日, content里的不还是没变? 难堪的一笔
% --------------------------------
\newtheorem{theorem}{Theorem}
\newtheorem{example}{Example}
\newtheorem{definition}{Definition}
\newtheorem{lemma}{Lemma}
\newtheorem{proposition}{Proposition}
% --------------------------------
\title{graph match}

\begin{document}
\tableofcontents
\maketitle
\section{some definition and the description of the problem}

总之我们的目的是要找出匹配边, 至于说匹配是用来干什么的? 我目前还不太清楚. Anyway, 下面给出一系列的定义. 

\begin{definition}[匹配]
匹配是图的子图, 设为 $G' = \left(V' , E'\right)$, 其中 $V' = V$ , $E ' = \left\{e \in E : e \text{互不相邻}\right\}$
\end{definition}
画图就不画了, 你可以自己画一画, latex里画图好几把麻烦. 关键在于不相邻这个条件. 
\begin{definition}[最大匹配]
最大匹配是边数最多的匹配
\end{definition}
其实还有极大匹配, 就是说当前情况, 并不能再直接加边了的匹配, 但是明显, 极大匹配不一定是最大匹配. 
\begin{definition}[匹配边, 非匹配边]
$E' $ 即为匹配边, $E - E' $ 为非匹配边.
\end{definition}
\begin{definition}[交错路径]
如果说 $p = \left< e_1, \cdots  , e_{n}\right>$ 中 $e_{i}$ 交错地是匹配边和非匹配边, i.e.  $e_{i}$ 是匹配边, 那么 $e_{i+1}, e_{i-1}$ 都是非匹配边, 那么称这个路径是交错路径.
\end{definition}
\begin{definition}[增广路径]
如果说一个路径, 是交错路径, 并且非匹配边多于匹配边, 那么这个路径是增广路径.
\end{definition}

如果我们已知一个增广路径, 我们可以将其增广, viz. 将他们匹配和非匹配的身份调换, 这样匹配边数量加一.
\section{algorithm}
有定理: 
\begin{theorem}
一个匹配是最大匹配 $\iff $ 其没有增广路径.
\end{theorem}
那么这个定理足以证明下面算法的正确性:
\begin{description}
    \item[1] 找到augmenting路径.\footnote{一个不和匹配边相邻的边, 也能称为augmenting路径} 
    \item[2] 将augmenting路径augment. 
    \item[3] 直到不存在augmenting路径.
\end{description}

\section{application}
\end{document}